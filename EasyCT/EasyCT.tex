\documentclass[nofootinbib,preprint]{revtex4-1} 
\usepackage[utf8]{inputenc}
\usepackage[fleqn]{amsmath}
\usepackage[title]{appendix}
 \usepackage[none]{hyphenat}
\usepackage{amssymb}
\usepackage{amsmath}
\usepackage[mathscr]{euscript}
\usepackage{skak}
\usepackage{capt-of}
\usepackage{afterpage}
\usepackage{placeins}
\usepackage{natbib}
\usepackage{url}
\usepackage{hyperref}
\hypersetup{colorlinks=true}
\usepackage[margin=1in]{geometry}
\usepackage{graphicx,wrapfig} 
\usepackage[percent]{overpic}
\usepackage{tabularx}
\usepackage{algorithm}
\usepackage[noend]{algpseudocode}
\usepackage{makecell}
\usepackage{setspace}
\usepackage{rotating}
%\usepackage{footmisc}
\setdisplayskipstretch{}


%\usepackage{footmisc}
%\DefineFNsymbols{mySymbols}{{\ensuremath\dagger}{\ensuremath\ddagger}\S\P
%   *{**}{\ensuremath{\dagger\dagger}}{\ensuremath{\ddagger\ddagger}}}
%\setfnsymbol{mySymbols}

\renewcommand{\arraystretch}{1.3}
%\renewcommand\footnotelayout{\fontsize{10}{12}\selectfont}

\newcommand{\tFo}[3]{\,_2F_1 \bigg[ 
\genfrac..{0pt}{}{#1}{#2}\bigg| #3 \bigg]} 

\newcommand{\tFoIn}[3]{\,_2F_1 \Big[ 
\genfrac..{0pt}{}{#1}{#2}\Big| #3 \Big]} 

\newcommand{\rev}[1]{\text{\reflectbox{$#1$}}} 


\begin{document}

\title{Geometric Interpretation of a few Hypergeometric Series}
\author{Bradley Klee}
\email{bjklee@email.uark.edu, bradklee@gmail.com} % optional
\affiliation{Department of Physics, University of Arkansas, Fayetteville, AR 72701}

\date{\today}

\begin{abstract}
Ramanujan's article ''Modular equations and approximations to $\pi$'' draws special attention 
to three curious hypergeometric series. In mathematical physics, each series determines 
periods along the elliptic level curves of an integrable Hamiltonian surface. We perform a 
combinatorial search of Hamiltonian function space and find  a diverse collection of closely
related species, some of which may have been unknown previously. The search is made rigorous 
by here defined diagnostic algorithms \texttt{ExpToODE},
\texttt{HyperelipticToODE} and \texttt{DihedralToODE}. All three take an input integral 
function and output the relevant ordinary differential equation. Examples growing out of 
Ramanujan's theory have intrinsic value, but they are not the only interesting use cases of the 
diagnostic algorithms. What we can learn from this "walk through Ramanujan's garden" will 
ultimately help us to extend systematic analysis elsewhere.  
\end{abstract}

\maketitle 

\section{History and Introduction}
The aphorism "history tends to favor those who wrote it" applies no less in particular
to history of science. Ideally, science should be an apolitical process of finding out the 
truth by reason and experimentation. Practically, we can always expect to hear another 
anecdote of unfair practice and needless exclusion, even if we only listen to the inner 
workings of the dominant western system. Eurocentrism and Anglocentrism present an even
more significant problem to the world. Western scientists can, and often do, ignore developments and 
contributions of non-western origin. Our discussion of classical science thus far repeats 
the mistake of exclusive biasing by focusing solely on European lineages 
\cite{KLEE2020Prelude,KLEE2020Pendulum}. We will now begin 
to correct for this mistake by discussing a few earlier developments of India. 

The biased\footnote{Compare with "Hsin Shin Ming", "Nagarjuna's four propositions
and zen", available at
\href{https://kansaszencenter.org/resources/}{https://kansaszencenter.org/resources/}.}
 slogan \textit{Asian ideas matter} is especially relevant when we grapple with even a small 
part of the theory of Srinivasa Ramanujan (1887-1920). From what we know about Ramanujan's heritage, 
we can easily understand that his life and work built upon thousands of years of Indian 
culture\footnote{Robert Kanigel has written a biography of Ramanujan's life, \textit{The man who knew infinity}.
The book was recently adapted to full-length motion picture under the same title.}. 
To buck Anglocentrism, we can look at Ramanujan's notebooks, and then ask, which of his basic 
ideas and techniques were already known in India before the invasion of the British East India Trading Company? Less than two decades after Ramanujan's untimely death,  A.N. Singh published an 
imminently useful, English-language review, "On the use of series in Hindu mathematics"\footnote{A 
scan is available through Erv Wilson's electronic library, 
\href{http://www.anaphoria.com/library.html}{http://www.anaphoria.com/library.html}.}\cite{SINGH1936}. 
This article clearly shows that Ramanujan \textit{was not} the first Indian scientist to take
an interest in binomial coefficients, series expansions, or the transcendental constant $\pi$. 
That being said, Ramanujan \textit{was} among the first to work directly with 
the western scientific world. Two histories need to be compared.

The binomial coefficients are an irrevocably important set of numbers denoted $\binom{n}{k}$, usually 
for integer $n$ and $k$ with $ 0 \le k \le n $. In combinatorics, $\binom{n}{k}$ determines
the number of ways to choose $k$ distinct elements from a set of cardinality $n$, and the 
identity $\binom{n}{k}=\frac{n!}{(n-k)!k!}$ counts out the answer in terms of the factorial function 
$n!=n\times(n-1)\times(n-2)\times\ldots \times 1$ and $0!=1$. Algebra gives an alternative, additive definition,
$\binom{n}{0}=\binom{n}{n}=1$ otherwise $ \binom{n}{k}=\binom{n-1}{k-1}+\binom{n-1}{k} $, which follows
from the expansion $(1+z)^n=\sum_{k=0}^{n}\binom{n}{k}z^k=(1+z)\sum_{k=0}^{n-1}\binom{n-1}{k}z^k$. 
It is quite natural to view the binomial coefficients as elements of a triangular array, as in Fig. 
\ref{fig:Meru}. When the integers are arranged in this particular way, each row determines each next row by the
addition rule. In the west, such an arrangement goes by the name "Pascal's triangle". A.N. Singh tells 
us that the same tabulation was derived in India under a poetic heading, \textit{Meru Prast$\bar{a}$ra}, 
"the Steps of Meru", possibly as early as Pingala's time circa 200 BC. In parts of the old Indian 
cosmological system, Meru plays a role similar to that of Parnassus in the old Greek cosmology\textemdash 
It is the center of the poetic universe. Pingala asked the question, given a verse of $n$ syllables divided
into $k$ \textit{lagu}, meaning light, and $n-k$ \textit{guru}, meaning heavy, how many orderings
are possible? Not only did Pingala know the answer in terms of \textit{Meru Prast$\bar{a}$ra}, he 
also knew the row sums $\sum_{k=0}^n \binom{n}{k}=2^n$\cite{SINGH1936}. This was an important discovery for Indian 
science, and thereafter Sanskrit prosody became a standard format. 


\begin{wrapfigure}{R}{6cm}
\begin{center}
\begin{overpic}[width=0.3\textwidth]{./Figures/meru.eps}
 \put (47,86) {\Large 1}
 \put (37,66) {\Large 1}
 \put (57,66) {\Large 1}
 \put (27,46) {\Large 1}
 \put (47,46) {\Large 2}
 \put (67,46) {\Large 1}
 \put (17,26) {\Large 1}
 \put (37,26) {\Large 3}
 \put (57,26) {\Large 3}
 \put (77,26) {\Large 1}

 \put (7,6)  {\Large 1}
 \put (27,6) {\Large 4}
 \put (47,6) {\Large 6}
 \put (67,6) {\Large 4}
 \put (87,6) {\Large 1}

\end{overpic}
\caption{Pingala's \textit{Meru Prast$\bar{a}$ra}.}
  \label{fig:Meru}
\end{center}
\end{wrapfigure}

Not much is known about the life of Pingala, but the fact that he wrote high-minded Sanskrit suggests 
a northern origin, possibly in one of the larger cities on the Ganges river. The implicit
allusion of \textit{Meru} to the Himalayan mountains seems to support this hypothesis; however, 
conclusive evidence is lacking. During second and third century BC, the Maurya empire included 
much of the Indian subcontinent, but did not extend all through the southern peninsula to the 
Indian ocean. Excluded territory to the far south roughly aligns with two present-day states, 
Kerala and Tamil Nadu. More is known about Pingala's follower, Halayudha. During X Century, he 
lived in Karnataka, and later moved 1,000 kilometers northward to Ujjain, where he could accomplish 
his important exposition of Pingala's earlier work on \textit{Meru Prast$\bar{a}$ra} 
(again, see \cite{SINGH1936}). 
This gives  some idea of the extent to which ideas could move on a hundred year time scale. After another 
few centuries, an important mathematical school, versed in Sanskrit prosody, would finally 
arise in the southernmost state of Kerala.

Founding of the Kerala school of Astronomy and Mathematics is associated with the life and
work Madhava of Sangamagrama (circa 1340-1425). It is now commonly accepted that Madhava originated 
the poetic definition of series expansions for sine, cosine\cite{ROY1990}. At the time, the
standard educational procedure was to transmit knowledge mainly by spoken word. The teacher would
recite his verses directly to the students, and rigor would only be done if the listeners 
could think of an objection to the master's teaching (S.P. Singh told me as much by email). 
Due to this practice, the written record is not as helpful as we would like; although, over time, 
more details became available through the notes of subsequent students working in the lineages at 
Kerala\cite{MADHAVAVID}. Their accomplishment is too profound to belittle by imposing an exogenous 
western bias. Without any notion of calculus, Indian mathematicians at Kerala preceeded the greatest 
western scientists at the central task of exactly defining the transcendental constant $\pi$ in 
terms of an infinite series expansion\cite{ROY1990}.

Given that knowledge was already sufficiently mobile and moving southward, it is entirely probable 
that Hindu science had reached Tamil Nadu by XIX Century. Ramanujan was born in Tamil Nadu on December 22, 
1887, and attended primary school there. Were benevolent Hindu nationalists able to influence 
Ramanujan's early development as a mathematician? We can only say that Ramanujan, as a 
member of the Brahmin caste, had a right to receive Hindu science, not that he actually did. At 
this time, the traditions of India became undermined by British interventionism, which included 
a commandeering of the educational system. During the same years, Mahatma Gandhi (1869-1948)
developed the strategy of \textit{Satyagraha}\footnote{In his monumental "I Have a Dream" speech, Martin 
Luther King Jr. echoed Ghandi by saying \textit{"We must not allow our creative protest to degenerate into 
physical violence. Again and again we must rise to the majestic heights of meeting physical force with soul 
force."} }, which would eventually win back Indian independence, though not during Ramanujan's short 
lifetime. Sadly, we do not know if Ramanujan was aware of Pingala or Madhava by the start of his 
research into western mathematics. Regardless of what resources he could access in the material world, 
Ramanujan became the natural successor of India's earlier number theorists. Instead of linguistic
poetry, he wrote poetic equations. His Indian notebooks are filled with thousands of cryptic 
entries, which have resisted attempts to translate or decipher. 

The story about Ramanujan's ascendance is, as Hardy called it, romantic.
It provides a stark contrast to other goings-on around Europe and the rest of the world. A few months 
after Ramanujan's arrival in England, the Archduke Franz Ferdinand of Austria was assassinated on 
28 June, 1914 by Gavrillo Princip, a murderous agitator fighting on behalf of the Bosnian 
independence movement. There certainly were Hindu nationalists who advocated violent rebellion 
against colonial forces (as did Americans during the earlier American revolution), but Ramanujan
was not one of them. Having learned from the books of Cambridge mathematicians, and with encouragement
from his countrymen, Ramanujan sought out approval from the Cambridge Dons. The Royal Society 
answered Ramanujan's letters through the person of G.H. Hardy, himself an eccentric genius of 
relatively modest upbringing. Ramanujan was invited to attend Cambridge, and after three days 
meditating at a temple in Namakkal, he received permission to accept the invitation. His life 
thereafter is more well documented. Some of his ideas were properly published, though not as 
many as could have been. Posisbly during a brief stint as a mendicant, Ramanujan is 
thought to have contracted a long lasting form of Dysentery. He later died a tragic and 
untimely death at the young age of 32. 

Regrettably, the effort to understand Ramanujan's educational circumstances started only after his death, 
with the obituary of Seshu Aiyer and Ramachandra Rao, and Hardy's short biography 
"The Indian mathematician Ramanujan"\cite{AIYARRAO1927,HARDY1937}. Neither article says anything about 
Pingala or Madhava, but both mention Euler in connection with Ramanujan's early interest in sine and 
cosine functions. Hardy's account also claims that "Carr's \textit{Synopsis}\ldots first aroused Ramanujan's 
full powers". We don't know if this Hardy assertion is true, but it sounds possibly overstated due 
to Anglocentric bias. Nevertheless, it is an important historical fact that Ramanujan studied Carr, 
because \textit{Synopsis} defines figurate numbers next to hypergeometric function $\,_2F_1$ in 
items 289-292. In fact, Item 290 gives a tabulation, in a slightly different form, of the first few 
rows of Pascal's Triangle. Carr, an Englishman, mentions neither Pascal nor Pingala. Even if Ramanujan 
was unaware of his own nation's historical contributions to science, he would have found the same ideas 
in Carr; admittedly, with wrong attribution. 

For a period of time, interest in Ramanujan languished. In the late 1970's, Bruce Berndt and coworkers 
revived the mission of editing Ramanujan's notebooks. An overview of their work is given in 
\cite{BERNDTOVERVIEW}. It shows that Ramanujan was interested in many topics, and that he paid 
quite a lot of attention to the theory of elliptic functions. According to a bibliography of Ramanujan's primary 
sources\cite{BERNDTBOOKLIST}, he had access to A.G. Greenhill's treatise on elliptic functions\footnote{Scans 
are available online via \href{https://archive.org/details/asynopsiselemen00carrgoog}{archive.org},
including \href{https://archive.org/details/asynopsiselemen00carrgoog}{Carr's \textit{Synopsis}}
and \href{https://archive.org/details/cu31924001588395}{Greenhill's \textit{Elliptic Functions}}.
}, and at first, found some interest in the period function of the simple pendulum, the complete elliptic integral of 
the first kind\footnote{We have already rigorously defined $K(\alpha)$ in \cite{KLEE2020Prelude,KLEE2020Pendulum}. 
Another certificate is given in Section IV.}, here denoted by symbol $K(\alpha)$.
Legendre found an identity, $K\big(\sin(5\pi/12)^2\big)=\sqrt{3}K\big(\sin(\pi/12)^2\big)$, which 
piqued Ramanujan's interest and led him to record a few original results in his notebooks\cite{VILLARINO2020}
(and see also item 13 of Hardy's obituary \cite{HARDY1937}). Not only did Ramanujan at times exceed the preexisting 
theory of elliptic integrals, he also broke completely free of its confines. Section 11 of Berndt's 
overview pertains to Ramanujan's theory of alternative bases, which is also described in two full 
length articles \cite{BERNDTALTERNATIVE,BARUAH2009}. An important takeaway from these accounts is that entries of 
Ramanujan's notebooks show he had already discovered existence of the three alternative theories 
\textit{while living in India}. This both does and does not help to explain how he could rapidly 
publish the article "Modular Equations and approximations to $\pi$", much less than one year 
after arriving at Cambridge University\footnote{A complete bibliography is available online at 
\href{https://www.imsc.res.in/~rao/ramanujan/}{https://www.imsc.res.in/$\scriptstyle\sim$rao/ramanujan/}.}\cite{RAMANUJAN1914}.

Following the poetry of Freeman Dyson (1923-2020), Berndt also wrote a lovely invitation 
to catalogue the "Flowers which we cannot yet see growing in Ramanujan's 
garden of hypergeometric series, elliptic functions, and q's"\footnote{Our own 
thoughts are along the lines of "Parametric curves / which cannot yet be seen to grow / in a garden of Mystery",
but we will defer to historical precedent until a better consensus can be reached.}
\cite{BERNDTFLOWERS}. 
In this work of outlook (or is it actually introspection?), mathematical loose ends 
of Ramanujan's theory of alternative bases take eminent position in section 2. There it says,
\begin{quote}
In his famous paper [Modular equations and approximations to $\pi$], Ramanujan 
records several elegant series for 1/$\pi$ and asserts "There are corresponding 
theories in which $q$ is replace by one or other of the functions" 
$$q_r:=q_r(x):=\exp\bigg(-\pi \csc(\pi/r)\frac{\,_2F_1(\frac{1}{r},\frac{r-1}{r};1;1-x)
}{\,_2F_1(\frac{1}{r},\frac{r-1}{r};1;x)} \bigg),$$ where
$r=3,4,$ or $6$ and where $\,_2F_1$ denotes the classical Gaussian hypergeometric 
function.  
\end{quote}
Neither in the original notebooks, nor in the journal article, nor in subsequent 
editing did explicit mention of elliptic curves play a very significant role. 
However, Berndt and colleagues took a significant step in this direction by changing
Ramanujan's Euler-inspired notation, 
\begin{eqnarray}
K_1 &=& 1 + \frac{1 \cdot 3}{4^2}k^2 + \frac{1 \cdot 3 \cdot 5 \cdot 7}{4^2 \cdot 8^2}k^4 
          + \frac{1 \cdot 3 \cdot 5 \cdot 7 \cdot 9 \cdot 11}{4^2 \cdot 8^2 \cdot 12^2}k^6
          + \ldots  \nonumber ,  \\
K_2 &=& 1 + \frac{1 \cdot 2}{3^2}k^2 + \frac{1 \cdot 2 \cdot 4 \cdot 5}{3^2 \cdot 6^2}k^4 
          + \frac{1 \cdot 2 \cdot 4 \cdot 5 \cdot 7 \cdot 8}{3^2 \cdot 6^2 \cdot 9^2}k^6
          + \ldots  \nonumber ,  \\
K_3 &=& 1 + \frac{1 \cdot 5}{6^2}k^2 + \frac{1 \cdot 5 \cdot 7 \cdot 11}{6^2 \cdot 12^2}k^4 
          + \frac{1 \cdot 5 \cdot 7 \cdot 11 \cdot 13 \cdot 17}{6^2 \cdot 12^2 \cdot 18^2}k^6
          + \ldots  \nonumber ,   , 
\end{eqnarray}
to that of the standard hypergeometric theory. Since we know that the hypergeometric function, 
whether it is due to Gauss or Euler, is the solution of a second-order ordinary 
differential equation(Cf. \cite{KZ2001}, Section 2), we can immediately hypothesize that 
Ramanujan's original assertion is equivalent to an assertion that 
\begin{quote}
\textit{There are corresponding theories in which the underlying elliptic curve geometry 
is replaced by one or other of the curve families $\mathcal{X}_3(\alpha)$, $\mathcal{X}_4(\alpha)$, 
$\mathcal{X}_6(\alpha)$}.
\end{quote}
In this hypothesis the unknowns $\mathcal{X}_s(\alpha)$ are Riemann tori over the complex 
numbers. As their shape varies with $\alpha$, the real and complex periods must 
be solutions of the hypergeometric differential equation\footnote{Our notation is 
different: $r$ is reserved for radius, $s$ is for signature, and $\alpha$ is the 
default expansion parameter. For a detailed working of case $s=2$, see \cite{KLEE2020Prelude} 
Section V and \cite{KLEE2020Pendulum}. },
\begin{eqnarray}
\mathcal{A}_s \circ T_s(\alpha) = 0, \;\;\text{where}\;\;\mathcal{A}_s
=(s-1)-s^2(1-2\alpha)\partial_{\alpha}-s^2\alpha(1-\alpha)\partial_{\alpha}^2. \nonumber
\end{eqnarray}
For $s=2,4$ and $6$ geometric models were known classically, and maybe a model for the difficult 
case $s=3$ was also known to a small cadre of European cognescenti(?). Nevertheless, a 
systematic exposition was missing from the literature, until the issue was taken up 
by L.C. Shen. In a series of articles \cite{SHEN1998,SHEN2013,SHEN2014,SHEN2017} 
he revealed that the relevant 
geometries could be obtained from the Chebyshev polynomials, amazing! We may not 
be too surprised to find that the mysterious garden contains even more as 
yet unseen. The unmentioned alternatives have different symmetries, different genera,
and perhaps even different fragrances, so they are also deserving of correct and 
individual diagnosis.

Presently the theory of Creative Telescoping gives us new tools for systematizing
and automating analysis of integrals taken along deformable curves\footnote{For a list of 
relevant references, trackback from the list given in \cite{KLEE2020Prelude} and see also \cite{COMPLEXITY2010}.}. 
In a most general form, as recently discussed by Bostan
and Lairez, the complications are many\cite{BOSTAN2017,LAIREZ2016}. However, when it is possible 
to reduce  a geometric integral to a univariate form, simple Hermite reduction can be used effectively.
The goal is to take a curve, say $\mathcal{X}(\alpha)=\{(p,q) : 2H(p,q)=\alpha\}$, write 
the period integral $T(\alpha)=\oint_{\mathcal{X}}dt$, and instead of evaluating $T(\alpha)$
directly, to find an annihilating operator $\mathcal{A} \in \mathbb{Q}[\![\alpha,\partial_{\alpha}]\!]$
(also called a telescoper) such that $\mathcal{A} \circ T(\alpha)=0$. Annihilation of the 
period function happens when $\mathcal{A} \circ dt -\partial_t \Xi = 0$, with certificate
$\Xi$ a function of $t$. If the annihilator and its certificate are known, the task of 
evaluating the integral $T(\alpha)$ can be changed for a similar task of solving an
ordinary differential equation. This is a much easier approach and allows us to efficiently
search for special $\mathcal{X}(\alpha)$. 

In two previous articles (or chapters), we have already discussed the standard case $s=2$ 
in detail\cite{KLEE2020Prelude,KLEE2020Pendulum}. The appropriate period function, 
$T_2(\alpha) = 4K(\alpha)=2\pi\tFoIn{\frac{1}{2},\frac{1}{2}}{1}{\alpha}$, can
be written in terms of the central binomial coefficients 
$T_2(\alpha) = 2\pi\sum_{n\ge 0} \frac{1}{16^n}\binom{2n}{n}^2 \alpha^n$. 
If we throw away the denominators $\frac{1}{16^n}$, and search the Online Encyclopedia of Integer Sequences 
(OEIS)\cite{SLOANE2020} for the integer expansion coefficients, then we find entry 
\href{https://oeis.org/A002894}{A002894}. Clicking through the cross references, we might 
also find  \href{https://oeis.org/A006480}{A006480} and \href{https://oeis.org/A000897}{A000897}, and both 
entries cross reference to \href{https://oeis.org/A113424}{A113424}. Though historically unusual\footnote{When, 
before now, have we ever made such interesting maths discoveries by clicking a hyperlink?}, this is still a great and 
easy way to find out more about Ramanujan's alternative series, 
\begin{eqnarray}
T_3(\alpha)  &=&  \tFo{\frac{1}{3},\frac{2}{3}}{1}{\alpha} 
= \sum_{n\ge 0} \frac{1}{27^n}\binom{3n}{n}\binom{2n}{n}\alpha^n, \nonumber \\ 
T_4(\alpha)  &=&  \tFo{\frac{1}{4},\frac{3}{4}}{1}{\alpha} 
= \sum_{n\ge 0} \frac{1}{64^n} \binom{2n}{n}\binom{4n}{2n}\alpha^n,  \nonumber \\ 
\text{and} \;\;\;\;\;\;T_6(\alpha)  &=&  \tFo{\frac{1}{6},\frac{5}{6}}{1}{\alpha} 
=  \sum_{n\ge 0}\frac{1}{432^n}\binom{3n}{n}\binom{6n}{3n}\alpha^n,  \nonumber 
\end{eqnarray}
in Ramanujan's notation $K_2$, $K_1$, and $K_3$ respectively. Written in this way, the 
dependence on Pingala is clear, but mystery remains. Why should these particular 
binomial products matter at all, and why only these three? These questions are motivation
enough for the present work, but a little more will be said about $\pi$ and the Kerala connection.

When calculating and storing proof data on a computer, we are not limited by paper shortages. 
Nothing but time will prevent us from recording as many results as we can. The combinatorics of 
Section II can be skipped, but it leads quite naturally into section III where most of the 
rigorous analysis is done by a few different implementations of an algorithm \texttt{EasyCT}. 
Requiring concordance between alternative implementations, not only do we easily find three 
relevant families of elliptic curves, we find that these three are somehow the lone minimal 
examples (except for a few higher-dimensional reflections). New geometric definitions allow
quick and easy proof of Legendre-style identities, which gets us into Chapter 1 of  
"$\pi$ and the AGM"\cite{BORWEIN1987}. It is tempting to keep going in a pure maths 
direction, but by the end,
we will need to steer ourselves back toward physics calculations. Again, mathematical loose
ends may be appreciated for what they are worth. 

\pagebreak

\begin{figure}[t]
\begin{center}
\begin{overpic}[width=0.9\textwidth]{./Figures/FourGeos.eps}
 \put (8.5,2) {$\mathcal{C}_2$}
 \put (43,2) {$\mathcal{C}_3$}
 \put (70,2) {$\mathcal{C}_4$}
 \put (96,2) {$\mathcal{C}_6$}
\end{overpic}
\caption{Elliptic oscillation disks, with $T_s(\alpha)=2\pi \tFoIn{\frac{1}{s},\frac{s-1}{s}}{1}{\alpha}$ for $s=2,3,4,6$.}
  \label{fig:EllDisks}
\end{center}
\end{figure}


\section{Creative Combinatorics}
We have already shown that a few different Hamiltonian functions determine the 
simple pendulum's libration behavior, and that transformation theory accounts 
for equivalence between the alternative forms\cite{KLEE2020Pendulum}. Now it is due time to continue
developing Hamiltonian mechanics by considering oscillation disks in more generality. 
As a breif reminder, an \textit{oscillation disk} is a topological disk taken from
the phase plane, which is bounded at its center by a circular point, and bounded on 
its outer edge(s) by a separatrix curve and at least one hyperbolic point\footnote{For 
more definitions and theory, refer back to \cite{KLEE2020Pendulum} section III.}. Figure 
\ref{fig:EllDisks} shows four examples with blue level curves,
\begin{eqnarray}
\mathcal{C}_s(\alpha)=\Big\{(p,q):\alpha=2H_s(p,q), \;\; s=2,3,4,\text{\;or\;}6\Big\}  
\;\;\;\;\; \text{with} \;\;\;\;\; \alpha  \in (0,1).  \nonumber
\end{eqnarray}
The four geometries differ in symmetry; however, any curve $\mathcal{C}_s(\alpha)$ 
is an elliptic curve of genus one. In fact, the curves $\mathcal{C}_2(\alpha)$ to 
far left are determined by the physicist's version of Edwards's normal form\cite{EDWARDS2007}, 
$2H_2=p^2+q^2-p^2q^2$. For other examples, the assertion of genus 1 can be proven a few different ways
using the standard theory of elliptic curves\footnote{All non-singular cubic plane 
curves are elliptic due to the existence of chord-and-tangent addition rule
\cite{SILVERMAN1992,SILVERMAN2009}. The other quartic $H_4$ is birationally equivalent 
to a cubic function, read on for more details.}. 

Let us start with the more familiar cases where kinetic energy and potential energy are 
separable, i.e. cases that have $H(p,q) = \frac{1}{2}p^2+V(q)$ in units where $m=1$. Choosing the 
potential energy function $V(q)=\frac{1}{2}q^2 + \frac{1}{2}\sum_{n=3}^{N}v_n q^n$ 
forces a circular limit, $2 H(p,q) \approx p^2 + q^2$, around the origin. The harmonic period
at the circular point is $T_0=2\pi$, so the angular frequency scale is $\omega_0=1$ for 
the entire oscillation disk. Two exceedingly simple examples of this form are the cubic and 
quartic anharmonic oscillators,
\begin{eqnarray}
2H_6(p,q) = p^2 + q^2 -\frac{2\sqrt{3}}{9}q^3, 
\;\;\;\;\;\;\;\;\;\; 
2 H_4(p,q) = p^2 + q^2 -\frac{1}{4}q^4.  \nonumber
\end{eqnarray}
Coefficients $v_3=-\frac{2\sqrt{3}}{9}$ and $v_4=-\frac{1}{4}$ are chosen so that 
$2V(\sqrt{3})=1$ for the cubic function, and $2V(\pm \sqrt{2})=1$ for the quartic. Values 
$q=\sqrt{3}$ and $q=\pm \sqrt{2}$ determine the only local maxima of either potential.
These conventions must have a separatrix curve at $\alpha = 1$ and an oscillation disk 
with domain $\alpha \in [0,1)$. To distinguish that $H_6$ and $H_4$ are not equivalent 
by canonical transformation, we need only calculate distinct period functions. 

In action-angle coordinates the Hamiltonian function of the quartic anharmonic osicllator 
is written as $2H_4(\lambda,\phi) = 2\lambda-\lambda^2\sin(\phi)^4$. 
Solving $\alpha=2H_4(\lambda,\phi)$ for $\lambda = \frac{1-\sqrt{1-\alpha \sin(\phi)^4}}{\sin(\phi)^4} $ 
then allows us to write the period integral, 
$T_4(\alpha)=\oint 2(\partial_{\alpha}\lambda) d\phi =  \oint \frac{d\phi}{\sqrt{1-\alpha \sin(\phi)^4}}$. 
We should already know how to solve this integral in series expansion, 
\begin{eqnarray}
T_4(\alpha)= \oint \frac{d\phi}{\sqrt{1-\alpha \sin(\phi)^4}}
=\sum_{n=0}^{\infty}\frac{1}{4^n}\binom{2n}{n}\oint \sin(\phi)^{4n}d\phi
=\sum_{n=0}^{\infty}\frac{2\pi}{64^n}\binom{2n}{n}\binom{4n}{2n}. \nonumber
\end{eqnarray}
As with the previous cases $E(\alpha)$ and $K(\alpha)$, the expansion 
coefficients, call them $f_n$, satisfy a hypergeometric recursion, 
\begin{eqnarray}
f_0=1, \; (n+1)^2 f_{n+1} = (n+\tfrac{1}{4})(n+\tfrac{3}{4})f_n
\iff f_n = \frac{1}{64^{n}}\binom{2n}{n}\binom{4n}{2n}, \nonumber 
\end{eqnarray}
which determines a standard expression, $T_4(\alpha) = 2\pi\tFoIn{\frac{1}{4},\frac{3}{4}}{1}{\alpha}$.
It is instructive to compare $T_4(\alpha)$ with the earlier $T_2(\alpha)=4 K(\alpha)$ by 
writing the Hadamard products, 
\begin{eqnarray}
\tFo{\frac{1}{2},\frac{1}{2}}{1}{\alpha}=\tFo{\frac{1}{2},\cdot}{\cdot}{\alpha}
\star \tFo{\frac{1}{2},\cdot}{\cdot}{\alpha}
\;\; \text{and} \;\;
\tFo{\frac{1}{4},\frac{3}{4}}{1}{\alpha}=\tFo{\frac{1}{2},\cdot}{\cdot}{\alpha}
\star \tFo{\frac{1}{4},\frac{3}{4}}{\frac{1}{2}}{\alpha}, \nonumber
\end{eqnarray}
where the $\star$ operator indicates multiplication of expansion 
coefficients\footnote{When $F$ and $G$ are both hypergeometric functions: 
join upper parameters, join lower parameters with an additional $1$, 
and finally pairwise cancel parameters occurring in both upper and lower sets. }, 
\begin{eqnarray}
F(\alpha)\star G(\alpha) = \sum_{n=0}^{\infty} f_n\;\alpha^n \star \sum_{n=0}^{\infty} g_n\;\alpha^n 
= \sum_{n=0}^{\infty} f_n \; g_n \; \alpha^n  \nonumber .
\end{eqnarray}
The shared form, $2H=2\lambda-\lambda^2 \Phi$, determines the equivalent first factor, and they 
differ on the second factor $\sum \alpha^n \oint \Phi^n d\phi$. This observation suggests 
a first combinatorial foray.

\begin{figure}[t]
\begin{center}
\begin{overpic}[width=0.9\textwidth]{./Figures/ThreeGeos.eps}
 \put (20,0) {$\mathcal{C}_3'$}
 \put (51.5,0) {$\mathcal{C}_4'$}
 \put (95,0) {$\mathcal{C}_6'$}
\end{overpic}


\phantom{\;}
\caption{Higher genus oscillation disks with $T(\alpha)=T_s(\alpha^2)=2\pi \tFoIn{\frac{1}{s},\frac{s-1}{s}}{1}{\alpha^2}$ for $s=3,4,6$.}
  \label{fig:NotEllDisks}
\end{center}
\end{figure}


The condition of separable potential and kinetic energy requires $\Phi \propto \sin(\phi)^4$. If
we loosen this condition, then $\Phi$ need only be a trigonometric polynomial of homogeneous 
degree $4$, i.e. must have the form $\Phi = \sum_{n=0}^{4} c_n \cos(\phi)^n\sin(\phi)^{4-n}$.
By searching the range of valid perturbations $\Phi$, it is possible to find at least one more 
well-related case, 
\begin{eqnarray}
2H_4'=p^2 + q^2  - \frac{1}{4}(p^2 + q^2)^2 +2 p^2 q^2 \iff 
\Phi=\cos(\phi)^4-6\cos(\phi)^2\sin(\phi)^2+\sin(\phi)^4. \nonumber
\end{eqnarray}
This case stands out because perturbing term reduces to $\Phi=\cos(4\phi)$. The oscillation 
disk takes the shape of a square with concave edges, as in the center of Fig. \ref{fig:NotEllDisks}. 
On the disk, time is measured by period function $T_4(\alpha^2)=2\pi\tFoIn{\frac{1}{4},\frac{3}{4}}{1}{\alpha^2}$. 
To see why this coincidence should occur, observe that odd powers of $\Phi$ integrate to zero, 
while $\oint\cos(4\phi)^{2n}d\phi\propto \binom{2n}{n}$. Relative to the case $\Phi=\sin(\phi)^4$, 
left and right Hadamard factors of the period function transpose\textemdash an impressive 
dance of parameters and exponents! We will encounter similar symmetry and similar maneuvers
as we continue to study simple period functions. 

The cubic oscillator is terribly more difficult to solve. In action-angle coordinates, the 
definition $2H_6= 2\lambda -\frac{4\sqrt{6}}{9}\lambda^{3/2} \cos(\phi)^3$  is a 
special case of $2H= 2\lambda -\lambda^{3/2}\Phi$. Constraint $\alpha=2H$ implies generally 
that $\alpha \Phi^2 = 2x - x^{3/2}, x=\lambda\Phi^2$. Exact root-solving is too unwieldy 
a process, so we have no better recourse than series reversion 
(cf. \href{https://oeis.org/A214377}{A214377}). Here we will drop rigor, 
skip the expansion of $\lambda$, and instead proceed directly to assert that,
\begin{eqnarray}
T_6(\alpha)=  \sum_{n=0}^{\infty}\frac{1}{8^n}\binom{3n}{n}\oint\bigg(\frac{4\sqrt{6}}{9} \cos(\phi)^3 \bigg)^{2n} d\phi
= 2\pi \sum_{n=0}^{\infty}\frac{1}{432^{n}}\binom{3n}{n}\binom{6n}{3n}. \nonumber 
\end{eqnarray}
A relatively easy proof will be given in the next section. For now, let us note the similarities
to previous examples. The standard expression $T_6(\alpha) = 2\pi\tFoIn{\frac{1}{6},\frac{5}{6}}{1}{\alpha}$ 
follows from, 
\begin{eqnarray}
f_0=1, \; (n+1)^2 f_{n+1} = (n+\tfrac{1}{6})(n+\tfrac{5}{6})f_n
\iff f_n = \frac{1}{432^{n}}\binom{3n}{n}\binom{6n}{3n}.\nonumber 
\end{eqnarray}
As before, Hadamard decomposition
$\tFoIn{\frac{1}{6},\frac{5}{6}}{1}{\alpha}=
\tFoIn{\frac{1}{3},\frac{2}{3}}{\frac{1}{2}}{\alpha}
\star \,_3F_2 \Big[ 
\genfrac..{0pt}{}{\frac{1}{6},\frac{1}{2},\frac{5}{6}}{\frac{1}{3},\frac{2}{3}}\Big| \alpha \Big]$
combines a left factor determined from the general cubic constraint, $\alpha \Phi^2=2\lambda -\lambda^{3/2}$,
with a right factor determined by integrating powers of $\Phi$. Again we can search valid
choices of $\Phi$ to find another interesting case, $2H_3=p^2+q^2-(\frac{4}{27})^{\frac{1}{2}} \; ( q^3-3 p^2 q)$,
with $\Phi=\frac{4\sqrt{6}}{9}\sin(3\phi)$. The period function can be calculated simply by changing 
the right factor of the Hadamard product. That is,  
$T_3(\alpha)=2\pi\tFoIn{\frac{1}{3},\frac{2}{3}}{1}{\alpha}$ follows from 
$\tFoIn{\frac{1}{3},\frac{2}{3}}{1}{\alpha}=
\tFoIn{\frac{1}{3},\frac{2}{3}}{\frac{1}{2}}{\alpha}
\star \tFoIn{\frac{1}{2},\cdot}{\cdot}{\alpha}$.


\begin{table}[p]
\begin{center}
\captionof{table}{Integral series $I(\alpha) =\oint\frac{d\phi}{1-\alpha\Phi}
=\sum_{n=0}^{\infty} \oint (\alpha\Phi)^n  d\phi$ must satisfy $\mathcal{A} \circ I(\alpha)=0$.}  
\label{tab:TrigPolys}
\begin{tabularx}{\textwidth}{ c | c | c l | c   | c  }
\hline \hline
 \hspace{0.1cm}s\hspace{0.1cm}  & \hspace{0.4cm} $\Phi=$ \hspace{0.4cm} & \hspace{.2cm} 
 &  $\mathcal{A}=$ \hspace{9cm} 
 & $I(z)$ &  \hspace{0.03cm} $\,_pF_q$? \hspace{0.03cm}     \\
\hline 
2           &  $P^2 Q^2$  && $2-(1-4\alpha)\partial_{\alpha}$  
            &\hspace{0.1cm} \href{https://oeis.org/A000984}{A000984} \hspace{0.1cm}
            &   yes \\
2           &  $I_2 Q^2$   && same as for $\Phi=P^2 Q^2$
            & \href{https://oeis.org/A000984}{A000984} 
            &   yes \\
4           &  $Q^4$      && $6-(1-64)\alpha\partial_{\alpha}
                             - 2 \alpha (1 - 16 \alpha)\partial_{\alpha}^2$  
            & \href{https://oeis.org/A001448}{A001448} 
            &  yes \\
4           &  $(Q_4)^2$       && same as for $\Phi=P^2 Q^2$ 
            & \href{https://oeis.org/A000984}{A000984} 
            &  yes  \\            
6           &  $(P Q^3)^2$    && $30 - 3 (1- 452 \alpha)\partial_{\alpha}
                               - 8 \alpha (4 - 243 \alpha)\partial_{\alpha}^2$ 
            & \href{https://oeis.org/A211419}{A211419} 
            &  yes \\
            & \tiny (cont.) && \hspace{0.5cm} $ -16 \alpha^2 (1 - 27 \alpha)\partial_{\alpha}^3$  &  &    \\            
            &  $P_2 P^2$    && $2(1 + 4 \alpha)^2 
                             -(1 - 32 \alpha -112 \alpha^2 - 160 \alpha^3)\partial_{\alpha}$ 
                            %  -2 \alpha (1 + \alpha) (1 + 4 \alpha) (1 - 8 \alpha)\partial_{\alpha}^2$                   
            & \href{https://oeis.org/A288470}{A288470} 
            &  no \\
                & \tiny (cont.)       && \hspace{0.5cm}  $ -2 \alpha (1 + \alpha) (1 + 4 \alpha) (1 - 8 \alpha)\partial_{\alpha}^2$                   
            & &     \\
            &  $P_3 P $    && $12 \alpha (4 + 9 \alpha)- 2 (1 -13 \alpha - 84 \alpha^2 - 270 \alpha^3) \partial_{\alpha}$ 
            & \href{https://oeis.org/A092765}{A092765} 
            &  no  \\
                & \tiny (cont.)       && \hspace{0.5cm} $ -\alpha (1 - 4 \alpha) (1 + 6 \alpha) (4 + 9 \alpha)\partial_{\alpha}^2$                   
            & &     \\
            &  $(P_3 Q)^2$     && $12 (10 - 126 \alpha + 729 \alpha^2)$  
            & \href{https://oeis.org/AXXXXXX}{nAn} 
            &  no \\
                & \tiny (cont.)       && \hspace{0.5cm} $ -6 (5 - 1093 \alpha + 7308 \alpha^2 - 13122 \alpha^3)\partial_{\alpha}$                   
            & &     \\
                & \tiny (cont.)       && \hspace{0.5cm} $ -\alpha (320 - 11043 \alpha + 85806 \alpha^2 - 69984 \alpha^3)\partial_{\alpha}^2$                   
            & &     \\
                & \tiny (cont.)       && \hspace{0.5cm} $ -2 \alpha^2 (5 - 54 \alpha) (16 - 207 \alpha + 108 \alpha^2)\partial_{\alpha}^3$                   
            & &     \\
\hline
3           &  $(Q_3)^2$      && same as for $\Phi=P^2 Q^2$
            & \href{https://oeis.org/A000984}{A000984} 
            &  yes \\
3           &  $(I_2 Q)^2$   && same as for $\Phi=P^2 Q^2$
            & \href{https://oeis.org/A000984}{A000984} 
            &  yes \\
4           &  $(P Q^2)^2$  && $24-6 (1 - 136 \alpha)\partial_{\alpha}-18 \alpha (3 - 64 \alpha)\partial_{\alpha}^2$ 
            & \href{https://oeis.org/A005810}{A005810} 
            &  yes \\
            & \tiny (cont.)       && \hspace{0.5cm} $ -\alpha^2 (27 - 256 \alpha)\partial_\alpha^3$                   
            & &     \\
6           &  $(Q^3)^2$        && $40 - 2 (1 - 904 \alpha)\partial_{\alpha} - 18 \alpha (1 - 144 \alpha)\partial_{\alpha}^2$  
            & \href{https://oeis.org/A066802}{A066802} 
            &  yes \\
            & \tiny (cont.)       && \hspace{0.5cm} $ -9 \alpha^2 (1 - 64 \alpha)\partial_{\alpha}^3$                   
            & &    \\
            &  $(P_2 Q)^2$  && $24 (1 - 16 \alpha)^2 $ 
            & \href{https://oeis.org/A005721}{A005721} 
            &  no \\           
            & \tiny (cont.)       && \hspace{0.5cm} $ -6 (1 - 248 \alpha + 2688 \alpha^2 - 9216 \alpha^3)\partial_{\alpha}$                   
            & &     \\
            & \tiny (cont.)       && \hspace{0.5cm} $ -6 \alpha (1 - 16 \alpha) (9 - 276 \alpha + 512 \alpha^2)\partial_{\alpha}^2$                   
            & &     \\
            & \tiny (cont.)       && \hspace{0.5cm} $ -\alpha^2 (1 - 16 \alpha)^2 (27 - 32 \alpha)\partial_{\alpha}^3$                   
            & &    \\
            \hline
            &  $I_2 Q^4$      && same as for $\Phi=Q^4$   
            & \href{https://oeis.org/A001448}{A001448} 
            &  yes \\
3           &  $I_2 P^2 Q^2$  && same as for $\Phi=P^2 Q^2$ 
            & \href{https://oeis.org/A000984}{A000984} 
            &  yes\\
\hline
            &  $(P^5 Q)^2$        && $7560-30(3-184504\alpha)\partial_{\alpha}-30\alpha(459-1076000\alpha)\partial_{\alpha}^2 $
            & \href{https://oeis.org/AXXXXXX}{nAn}  
            &  yes \\
            & \tiny (cont.)       && \hspace{0.5cm} $-423\alpha^2(99-80000\alpha)\partial_\alpha^3-32\alpha^3 (729 - 312500 \alpha)\partial_{\alpha}^4$                   
            & &    \\
            & \tiny (cont.)       && \hspace{0.5cm} $- 4 {\alpha}^4 (729 - 200000 \alpha)\partial_{\alpha}^5$                   
            & &    \\
6           &  $(P_2 Q_2^2)^2$    && same as for $\Phi=(P Q^2)^2$
            & \href{https://oeis.org/A005810}{A005810} 
            &  yes  \\
 \vdots   &  \vdots          &&   & \vdots  & \vdots   \\
\hline            
\end{tabularx}
\phantom{\;}
$P_n=2\cos(n\phi),\;Q_n=2\sin(n\phi),\;P=P_1,\;Q=Q_1, \;I_2=\frac{1}{4}(P^2+Q^2)=1$. 

See Appendix A for computer proofs.
\end{center}
\end{table} 

By the most expedient, intuitive analysis, we have already uncovered a secret that 
is well within the reaches of what Ramanujan himself could have known and calculated:
\begin{quote}
\textit{The three alternatives to $K(\alpha)$ all obey a Hadamard decomposition to two hypergeometric
factors. One factor is determined by the general choice of a degree, either quartic or cubic. 
The other factor is then determined by the special choice of a trigonometric polynomial, 
homogeneous in the chosen degree.} 
\end{quote}
We will never know exactly how Ramanujan found $K_1$, $K_2$, and $K_3$, but his notebooks 
do evidence a propensity for exhaustive searches. In any case, the statement is not lacking 
insight, but certainly needs more rigor. We can do incrementally better by using Creative 
Telescoping to print and verify Table \ref{tab:TrigPolys}. Here, right factors are written as 
$I(\alpha)=\oint\frac{d\phi}{1-\alpha\Phi}$ and given alongside an annihilator 
$\mathcal{A}$ such that $\mathcal{A} \circ I(\alpha)=0$. 
When annihilator $\mathcal{A}$ determines a hypergeometric $I(\alpha)$, the 
coefficient recursion can be found by the Frobenius method. We will now give 
a brief, easy example of how this works in practice.  

The most simple choice, $\Phi=4\sin(\phi)^2$, recalls earlier calculations\footnote{Cf. 
[1] sec. 1-2. In fact, up to scale of $\alpha$, the same matrix invariants can be used again.} 
of $E(\alpha)$ and $K(\alpha)$. The integral function 
$I(\alpha)=\oint\frac{d\phi}{1-4\alpha\sin(\phi)^2}$
has no square root, so details work out with even less effort. The identity,
\begin{eqnarray}
\mathcal{A}\circ \frac{dI}{d\phi}-\partial_{\phi}\Xi
=\Big(2-(1-4\alpha)\partial_{\alpha} \Big) \circ \frac{1}{1-4 \alpha \sin(\phi)^2}
-\partial_{\phi}\bigg( \frac{2\cos(\phi)\sin(\phi)}{1-4 \alpha \sin(\phi)^2}\bigg) = 0.
\nonumber
\end{eqnarray}
utilizes certificate $\Xi$ to prove that $\mathcal{A}$ almost annihilates
$\frac{dI}{d\phi}$. Again, exact differentials integrate to zero on a closed 
contour, thus $\mathcal{A}$ completely annihilates $I(\alpha)$, or $\mathcal{A} \circ I(\alpha)=0$.  
When solving for a coefficient recursion, it is also useful to view 
$\mathcal{A}$ as a coefficient matrix, $\mathcal{A}\sim \big(\;^{2}_{0}\;^{-1}_{\;4}\big)$, 
where powers of $\alpha$ increase by row, and powers of $\partial_{\alpha}$ increase by
column. The matrix encoding of $\mathcal{A}$ allows algorithmic streamlining of the Frobenius method,
\begin{eqnarray}
\frac{a_{n+1}}{a_n} = \frac{(2,4)\cdot(1,n)}{(1)\cdot(1+n)} = 4\frac{(n+\tfrac{1}{2})}{(n+1)} \implies
I(\alpha)=\oint\frac{d\phi}{1-4 \alpha \sin(\phi)^2}=\tFo{\frac{1}{2},\cdot}{\cdot}{4 \alpha}. \nonumber
\end{eqnarray}
In case it is not already clear, dots in the arguments of a $\,_2F_1$ function indicate 
canceled parameters, so the evaluation is more precisely to a $\,_1F_0$ function. 
\FloatBarrier

The main virtue of Table \ref{tab:TrigPolys} is \textit{systematism}. Rather than searching
by intuition through the space of valid $\Phi$, we develop a comprehensive list of monomials,
and use the diagnostic \texttt{ExpToODE} to distinguish cases. We include any
monomial of the variables $Q_n=2\sin(n\phi)$,  $P_n=2\cos(n\phi)$, and $I_2=\frac{1}{4}(P^2+Q^2)=1$, 
which satisfies the degree constraint that ${d=\sum \text{subscript} \times \text{exponent}}$, with $d=4$ 
or $d=3$, and summing over all multiplicands. For the quartic case, we 
find only 6 distinct cases,  and only 4 distinct cases for cubic functions. These 
appear in the first and second divisions of Table \ref{tab:TrigPolys}. Perturbations 
$\Phi$ appear with an extra square when odd powers would otherwise integrate 
to zero. This is true for all cubic perturbations, and also for a few of the quartic
perturbations, as we have already seen.

Given the data of Table \ref{tab:TrigPolys}, matrix encodings of each 
annihilator can be inspected to determine whether or not $I(\alpha)$ is a hypergeometric
function. If the matrix has non-zero values only on the central diagonal and the 
first upper diagonal, then it is hypergeometric and gets a "yes" in the last column. 
Otherwise if the matrix form of $\mathcal{A}$ contains non-zero values on $k>2$ diagonals, 
the Frobenius recursion will relate $a_n, a_{n+1},\ldots,a_{n+k-1}$, so it cannot be 
hypergeometric.\footnote{There is a sometimes a caveat about "minimal telescopers", but 
all "no" cases of Table \ref{tab:TrigPolys} have been double checked for the minimal property by guess 
and check, and by referencing with OEIS.} We should not be too surprised to find a few 
more interesting models for the series called by Ramanujan $K_1$, $K_2$ and $K_3$. 
However it is somewhat amazing that all "yes" hits lead to one of these periods. 

The asymmetric alternatives are also worth a look. Reading down the table, we first 
find the pertubation $\Phi=I_2Q^2=Q^2$, and can recognize the corresponding Hamiltonian
function, $2H_{\varphi}=(p^2+q^2)(1-\frac{1}{4}q^2)$, as the algebraic form that describes simple 
pendulum libration. The first unknown is $\Phi=(PQ^3)^2$, or in the full notation, 
$2H_6''=p^2+q^2-\frac{4\sqrt{3}}{9}p q^3$. The period function, 
$T_6(\alpha^2)=2\pi\tFoIn{\frac{1}{6},\frac{5}{6}}{1}{\alpha^2}$,
follows from the Hadamard decomposition,
\begin{eqnarray}
\tFo{\frac{1}{6},\frac{5}{6}}{1}{\alpha^2}=
\tFo{\frac{1}{4},\frac{3}{4}}{\frac{1}{2}}{\alpha^2}
\star\,_3F_2 \bigg[ \genfrac..{0pt}{}{\frac{1}{6},\frac{1}{2},\frac{5}{6}}{\frac{1}{4},\frac{3}{4}}\Big| \alpha^2 \bigg]
 \nonumber.
\end{eqnarray}
Comparison with the similar cubic period reveals more than dancing parameters,
\begin{eqnarray}
\binom{4n}{2n}\sum_{k=0}^n \binom{6 n}{k}\binom{5 n - k - 1}{n-k}=\binom{3n}{n}\binom{6n}{3n}. \nonumber
\end{eqnarray}
a seemingly improbable, nonetheless true, binomial identity! Another way to prove the identity 
is to observe that the shear transformation $p \rightarrow p+\frac{2\sqrt{3}}{9}q^3$ takes 
quartic $H_6''$ to sextic $2H_6'=p^2+q^2-\frac{4}{27}q^6$. All such shears have Jacobian 
determinant $1$, so they are canonical transformations, thus preserve periods of oscillation.
As it must, $H_6'$ also follows  from $\Phi=(Q^3)^2$ applied to the general sextic form.  
Another similar example for signature $4$ is found from $H_4''=p^2+q^2-pq^2$, by applying 
the shear $p \rightarrow p+\frac{1}{2}p^2$. This canonical transformation again obtains $H_4$,
thus $\binom{3n}{n}\binom{4n}{n}=\binom{2n}{n}\binom{4n}{2n}$, another unexpected 
identity!

The only cases left are those marked as $s=3$ in Table \ref{tab:TrigPolys}. Applying
an Abel-Wick rotation, $p\rightarrow i p, q\rightarrow \sqrt{3}-q$ to $1-H_3$ obtains 
$\widetilde{H}_3=(3p^2+q^2)(1-\frac{2\sqrt{3}}{9}q)$. Up to a factor $\sqrt{3}$ on 
the frequency scale $\omega_0$, $\widetilde{H}_3 \approx (p^2+q^2)(1-\frac{2\sqrt{3}}{9}q)$,
what we obtain by applying ${\Phi=I_2Q}$ to the general cubic form. Isoperiodicity follows
from the fact that $\alpha \rightarrow 1-\alpha$ acts invariantly on the annihilator 
$\mathcal{A}_s$. The last essential case is $\Phi=Q_3^2$, 
with Hamiltonian form, ${2H_3'=p^2+q^2-\frac{4}{27}( q^3-3 p^2 q)^2}$. 
By now, the derivation $H_3'$ and its period function $T_3(\alpha^2)$
should be quite obvious. If not, we assert that sextic anharmonic 
oscillators of the form  $\alpha = 2H(\lambda,\phi)=2\lambda-\lambda^3\Phi$ 
are measured by the period function, 
\begin{eqnarray}
T(\alpha) = \sum_{n=0}^{\infty} \frac{1}{8^n}\binom{3n}{n}\alpha^{2n} \oint \Phi^n d\phi 
= \tFo{\frac{1}{3},\frac{2}{3}}{\frac{1}{2}}{\alpha^2}\star\oint\frac{d\phi}{1-\frac{27}{32}\alpha^2\Phi} . \nonumber
\end{eqnarray}
Consequently, when a cubic function $\alpha=2H=2\lambda -\lambda^{\frac{3}{2}}\Phi$ has period $T(\alpha)$, 
the corresponding sextic function $\alpha=2H=2\lambda -\lambda^{3}\Phi^2$ has period $T(\alpha^2)$. 

Having done so much more analysis, we can now strengthen the earlier theorem:
\begin{quote}
\textit{Assuming either form, $\alpha=2H=2\lambda-\lambda^{\frac{3}{2}}\Phi$ or 
$\alpha=2H=2\lambda-\lambda^2\Phi$, a valid choice of monomial $\Phi$ determines 
a hypergeometric period $T(\alpha)$ if and only if $\mathcal{A}_s \circ T(\alpha)=0$
for $s=2,3,4$ or $6$. Canonical models $H_s$ all have elliptic level curves $\mathcal{C}_{s}(\alpha)$
for $\alpha \in (0,1)$. Except for $s=2$, each $H_s$ has a higher-genus analog, and 
a canonical set of $H_s'$ for $s=3,4,6$ can be chosen to maximize symmetry.}
\end{quote}
Take a step back and think about what the strengthened result says. Not only is it possible 
to find Ramanujan's set $K_1$,$K_2$, and $K_3$ by searching a space of geometric models;
if we place mild constraints on the search space, we will only find the set 
$K_1$,$K_2$, and $K_3$! In that sense Ramanujan's presentation is \textit{complete} if not 
\textit{comprehensive}. This curious case of good insight blurs the line between 
fortune and genius, but not between apathy and work. Don't forget, Ramanujan did not have a computer 
to generate and check results!

\begin{table}[t]
\begin{center}
\captionof{table}{Sextic Hamiltonians and their period functions, $T(\alpha) =
\,_4F_3 \Big[ \genfrac..{0pt}{}{\frac{1}{6},\frac{1}{3},\frac{2}{3},\frac{5}{6}
}{\frac{1}{4},\frac{1}{2},\frac{3}{4}}\Big| \alpha^4 \Big]
\star I(\alpha)$.}  
\label{tab:HypSextics}
{\renewcommand{\arraystretch}{2}%
\begin{tabularx}{\textwidth}{ c l | c l | c l }
\hline \hline
\hspace{0.1cm} & 
$\alpha = 2H =$ & \hspace{0.1cm} 
& $I(\alpha)=$ &\hspace{0.1cm} & $T(\alpha)=$ \\
\hline
& $p^2 + q^2 - \frac{8}{9}( p^5 q -\frac{10}{3}p^3q^3 +q^5 p)$ \hspace{0.5cm} & &
$2\pi\,_2F_1 \Big[ \genfrac..{0pt}{}{
\frac{1}{2},\cdot
}{\cdot}\Big| \alpha^4 \Big]$   & &
$2\pi\,_4F_3 \Big[ \genfrac..{0pt}{}{
\frac{1}{6},\frac{1}{3},\frac{2}{3},\frac{5}{6}
}{\frac{1}{4},\frac{3}{4},1}\Big| \alpha^4 \Big]$     \\
& $p^2 + q^2 - \frac{4 \sqrt{3}}{9} p q (p^2 - q^2)^2$  & &
$2\pi\,_3F_2 \Big[ \genfrac..{0pt}{}{
\frac{1}{4},\frac{1}{2},\frac{3}{4}
}{\frac{1}{3},\frac{2}{3}}\Big| \alpha^4 \Big]$   & &
$2\pi\,_2F_1 \Big[ \genfrac..{0pt}{}{
\frac{1}{6},\frac{5}{6}
}{1}\Big| \alpha^4 \Big]$     \\
& $p^2 + q^2 - \frac{4}{27}(p^2-q^2)^3$  & &
$2\pi\,_3F_2 \Big[ \genfrac..{0pt}{}{
\frac{1}{6},\frac{1}{2},\frac{5}{6}
}{\frac{1}{3},\frac{2}{3}}\Big| \alpha^4 \Big]$   & &
$2\pi\,_4F_3 \Big[ \genfrac..{0pt}{}{
\frac{1}{6},\frac{1}{6},\frac{5}{6},\frac{5}{6}
}{\frac{1}{4},\frac{3}{4},1}\Big| \alpha^4 \Big]$     \\
& $\big(p^2 + q^2\big)\big(1 - \frac{64\sqrt{3}}{243} p q^3\big)$  & &
$2\pi\,_3F_2 \Big[ \genfrac..{0pt}{}{
\frac{1}{6},\frac{1}{2},\frac{5}{6}
}{\frac{1}{4},\frac{3}{4}}\Big| \alpha^4 \Big]$   & &
$2\pi\,_6F_5 \Big[ \genfrac..{0pt}{}{
\frac{1}{6},\frac{1}{6},\frac{1}{3},\frac{2}{3},\frac{5}{6},\frac{5}{6}
}{\frac{1}{4},\frac{1}{4},\frac{3}{4},\frac{3}{4},1}\Big| \alpha^4 \Big]$     \\
& $p^2 + q^2 - \frac{32\sqrt{5}}{125} p q^5$  & &
$2\pi\,_5F_4 \Big[ \genfrac..{0pt}{}{
\frac{1}{10},\frac{3}{10},\frac{1}{2},\frac{7}{10},\frac{9}{10}
}{\frac{1}{6},\frac{1}{3},\frac{2}{3},\frac{5}{6}}\Big| \alpha^4 \Big]$ \hspace{0.5cm}  & &
$2\pi\,_4F_3 \Big[ \genfrac..{0pt}{}{
\frac{1}{10},\frac{3}{10},\frac{7}{10},\frac{9}{10}
}{\frac{1}{4},\frac{3}{4},1}\Big| \alpha^4 \Big]$     \\
\end{tabularx}}
\rule{\textwidth}{.01cm}
\end{center}
\end{table}

So long as we don't pay too much attention to the particulars of series reversion, 
there is no reason to cease the search at quartic degree. Given that $H_3'$ and $H_6'$ are sextic 
functions, it is natural to wonder: what happens generally for quintics and higher? 
Using \texttt{ExpToODE} we are able to perform an exhaustive search up to octic degree, 
over 200 choices of $\Phi$. Most frequently the annihilators are not hypergeometric,
but we also found many hypergeometric cases, including those of Table \ref{tab:HypSextics}. 
In the subset of hypergeometric search results, we do not find any surprising examples 
where $T(\alpha^n)=\tFoIn{a,b}{c}{\alpha^n}$ with 
$(a,b) \neq (s,\frac{s-1}{s})$ or $c \neq 1$. For $s=6$, the search uncovers another 
sextic model, though with period $T_6(\alpha^4)$. 
In a narrow search above $d=8$, we find $\alpha =2H=p^2 + q^2 - \frac{64}{27} p ^4 q^6$, 
again with period $T_6(\alpha^4)$. 

The two extra  occurrences of $T_6(\alpha)$ both entail freak-accident parameter cancellations. 
More commonly, Hadamard products involve little-to-no cancellation. For $d=7$ or $d=8$ it is 
already possible to find periods of the form $\,_{10}F_8$ or $\,_{10}F_9$. The search results 
show a clear average pattern: as degree increases, complexity of the period function increases,
and simple examples become sparsely distributed, if at all. We could continue to make
observations and hypotheses, but again these would depend on unproven assertions. 
To prove claims more surely, we will now delve deeper into the algorithmic theory 
of Creative Telescoping. 

\pagebreak

\begin{table*}
\begin{tabularx}{\textwidth}{l l}
\hline
\multicolumn{2}{l}{\textbf{Algorithm 1.} Simple Creative Telescoping via Hermite-Ostrogradsky reduction. }  \\
\hline
\textbf{Input:}  & An integrand $dI\big(\alpha,z(\alpha,t)\big)$, 
                   denominator $\rho$, and differentials $\partial_{\alpha}z$, and $\partial_{t}z$.   \\
\textbf{Output:} \hspace{0.5cm} & Annihilator $\mathcal{A}$ and certificate $\Xi$ 
                                  such that $\mathcal{A}\circ \frac{dI}{dt} - \partial_t\Xi=0$. \\
\hline
\end{tabularx}
\begin{algorithmic}[1]
\Function{EasyCT}{$dI$,$\rho$,$\partial_{\alpha} z$,$\partial_{t} z$}:
\hspace{2.6cm} \texttt{\#(Ha, ha, ha! Have Fun!)$>$\;\big(\rotatebox{-90}{\!\!\!\!\small \texttt{8D}}\big)}
\State $ d \gets \Call{Deg}{\rho}$; $\Delta \gets \Call{Deg}{\partial_{t} z}-1$; 
\State $u\gets  \sum_{n=0}^{d+\Delta-1}u_{n}z^n$; $v \gets \sum_{n=0}^{d-1}v_{n}z^n $;
\State $ \mathbf{G} \gets \Call{CoefficientMatrix}{\rho u - (\partial_{t} z)(\partial_z\rho) v,2d+\Delta }$; 
\If {$\Call{Det}{\mathbf{G}}=0$} \Return "Error: no inverse for $\mathbf{G}$.";
\Else \; $ (\mathbf{U},\mathbf{V}) \gets$ decomposition of $\mathbf{G}^{-1}$; $\dot{\mathbf{V}} \gets (\partial_t z )\partial_z\mathbf{V}$;
\EndIf
\State $n\gets 0$; $x_0\gets\frac{dI}{dt}$; $ \mathbf{x}_0 \gets \Call{CoefficientVector}{\rho x_0 ,d+\Delta}$;  
\State $\mathbf{A}\gets\Call{NullSpace}{\{\mathbf{x}_0\}^T}$;
\While{$\Call{Empty}{\mathbf{A}}$}  $n\gets n+1$; $\mathbf{x}_n=0$;
\State $x_n\gets\partial_{\alpha}x_{n-1}+(\partial_{\alpha}z)\partial_z x_{n-1} $;
\If {$!\Call{Reducible}{\rho, x_n,n}$ } 
     \State \Return "Error: $\rho/dI$ mismatch."; \EndIf
\State $\mathbf{w}=\{w_0,w_1,\ldots,w_n\} \gets \Call{PartialFractions}{x_n,\rho} $;
      \For{$m=0,1,\ldots,n$}
        \State $\mathbf{w}_m \gets \Call{CoefficientVector}{w_m,d+\Delta}$;
        \State $\mathbf{x}_n \gets \mathbf{x}_n 
        + \Call{VectorReduce}{\mathbf{U},\mathbf{V},\dot{\mathbf{V}},m,\mathbf{w}_m,d-1} $;
      \EndFor
\State $\mathbf{A}\gets\Call{NullSpace}{\{\mathbf{x}_0,\ldots,\mathbf{x}_n\}^T}$;
\EndWhile
\State $\mathbf{\Xi} \gets \Call{Reap}{\;\!}$; 
\Return $\big(\mathbf{A}\cdot\{1,\partial_{\alpha},\ldots,\partial_{\alpha}^n\},
\mathbf{A}\cdot\Call{SubTotal}{\mathbf{\Xi}}\big)$;
\EndFunction

\Function{VectorReduce}{$\mathbf{U},\mathbf{V},\dot{\mathbf{V}},k,\mathbf{w},l$}:
\While{$k>0$} 
\State $\Call{Sow}{\frac{1}{k \rho^k}\{1,z,\ldots,z^l\} \cdot \textbf{V} \cdot \mathbf{w} }$;
\State $\mathbf{w} \gets (\mathbf{U} +\frac{1}{k} \dot{\textbf{V}}) \cdot \mathbf{w}$;
\State $k \gets k-1 $;
\EndWhile
\State \Return $\textbf{w}$;
\EndFunction
\end{algorithmic}
\rule{\textwidth}{.01cm}
\end{table*}
\section{Diagnostic Algorithms}
Creative Telescoping is an algorithmic theory that aims to assist in the process 
of redefining functions according to the ordinary differential equations they satisfy.
We have already made use of \texttt{ExpToODE}, with input/output map 
$I(\alpha) \rightarrow \mathcal{A}$. Soon we will introduce two more 
algorithms, \texttt{hyperellipticToODE} and \texttt{DihedralToODE}. 
Despite differing input domains, all three algorithms follow the same pseudocode, 
as written in Alg. 1. This pseudocode is a template for a range of C.T. algorithms 
that rely on degree-bounded Hermite-Ostrogradsky reduction\cite{BRONSTEIN2005}. Such algorithms
are the simplest possible, while still meeting minimum rigor. They make a good
starting place for curious beginners. 

Hermite-Ostrogradsky reduction is essentially limited to univariate cases; however,
for the sake of versatility, it is useful to assume a chain-rule structure. The 
integral $I(\alpha) = \oint \frac{dI}{dt}dt$ requires a period $T$ such that 
$t\rightarrow t+T$ leaves $\frac{dI}{dt}$ invariant, so most likely, $t$ will not
make a good basis for reductions. We need an algebraic variable $z(\alpha,t)$. 
A good choice for $z$ is often determined by inspection of $\rho$, for typically 
$\rho \in \mathbb{Q}[\![i,\alpha,z]\!]$. In fact, the algorithm assumes $\rho$ is a 
degree-$d$ polynomial of $z$, i.e. $\rho = \sum_{n=0}^d c_n z^n$, and the $c_n$ 
themselves are usually rational functions or polynomials in the variable $\alpha$. 
Subsequently we must also require 
$\dot{\rho}=(\partial_t z) \sum_{n=0}^d n c_n z^{n-1}=\sum_{n=0}^{d+\Delta}c_n'z^n$, 
i.e. that $\dot{\rho}$ is a $z$-polynomial. We will discuss in detail the obvious 
examples $z=e^{it}$ and $z(t)=q(t)$, as well as a non-obvious, more difficult case,
$z(t)=\lambda(t)$. Since these three cases are all strongly related, it should help
to start with an overview of Alg. 1 including what it does, why it works, and when 
it can possibly fail. 

The first few lines 2-4 determine the dimensionality of the reduction process, but 
do not guarantee success. Once variables are chosen, a feasibility analysis needs to 
be performed relative to the central statement of Hermite-Ostrogradsky reduction 
from $w$ to $[w]$, 
\begin{eqnarray}
\frac{[w]}{\rho^{m}}=\Big(u-\frac{1}{m}\dot{v}\Big)\frac{1}{\rho^{m}} = 
\frac{w}{\rho^{m+1}}-\partial_t\bigg( \frac{v}{m\rho^{m}}\bigg) 
\;\;\; \iff \;\;\; 
 w = \rho u -\dot{\rho} v \nonumber.
\end{eqnarray}
That such a reduction exists is itself an improbable assumption. The variables $u$, 
$v$, and $w$ must be polynomials in the variable $z$ with consistent degrees, i.e. 
$\deg(w)=\deg(u)+\deg(\rho)$ and $\deg(w)=\deg(v)+\deg(\dot{\rho})$. Polynomials $u$ and $v$ are 
treated as unknowns with many degrees of freedom, $\text{dof}(v)=\deg(v)+1$ and 
$\text{dof}(u)=\deg(u)+1$. Solving the degree bound, $\deg(w)=\text{dof}(u)+\text{dof}(v)$,
we obtain that $\deg(u)=\deg(\dot{\rho})-1$ and $\deg(v)=\deg(\rho)-1$. We can also state that 
$\deg(u)=\deg(\dot{v})=d+\Delta-1$ in terms of $d=\deg(\rho)$ and  ${\Delta = \deg(\partial_t z)-1}$. 
This restricts all functions $w(z)$ with implicit $\deg(w)=2d+\Delta-1$ to have non-zero $w_n$ only 
from $n=0$ to $n=d+\Delta-1$. When all of these conditions are met, it is possible to proceed 
to line 5 of Alg. 1, where the first critical error could possibly occur. 

The $\mathbf{G}$ matrix is constructed relative to spanning vectors 
$\mathbf{z}=\{1,z,z^2,\ldots,z^{2d+\Delta}\}$ and  
$\mathbf{uv}=\{u_0,u_1,\ldots,u_{d+\Delta-1},v_0,v_1,\ldots,v_{d-1} \}$. 
It encodes the essentials of the right hand
side of $w=\rho u - \dot{\rho} v$. Introducing another coefficient vector $\mathbf{w}$ such that 
$w=\mathbf{z}\cdot\mathbf{w}$, we can rewrite the reduction constraint as 
$\mathbf{w}=\mathbf{G} \cdot \mathbf{uv}$. If and only if $\det(\mathbf{G})\neq0$, the linear equation 
is uniquely solveable. If $\det(\mathbf{G}) = 0$, an error is thrown on line 5 and Alg. 1 fails. For 
some choices of $z$ and $\rho$, it is possible to prove that $\det(\mathbf{G})$ never equals to zero,
and this marks an important distinction between \textit{proveable} and \textit{effective} algorithms.
Proveable algorithms are preferable because they will always work, whereas effective algorithms can
only be guaranteed case-by-case. Absolute rigor is never necessary, and as long as possible errors 
are well-understood, it is not always desirable\footnote{Our view is that extra rigor must sometimes be 
sacrificed to explore more deeply.}.

Line 5 is a major milestone for the algorithm, and passing it strongly suggests that the a positive 
result will be obtained upon halting. To explain subsequent line 6, we must split apart 
vector $\mathbf{uv}$ to $\mathbf{u}$ and $\mathbf{v}$, lengths $d+\Delta$ and $d$ respectively,
while dropping zeros of $\mathbf{w}$ to obtain $\mathbf{w}'$, another vector of length $d+\Delta$. 
Then we can define matrices $\mathbf{U}$ and $\mathbf{V}$ such that $\mathbf{u}=\mathbf{U}\cdot \mathbf{w}'$
and $\mathbf{v}=\mathbf{V}\cdot \mathbf{w}'$. Given that $\mathbf{uv}=\mathbf{G}^{-1}\cdot\mathbf{w}$,
matrices $\mathbf{U}$ and $\mathbf{V}$ must be submatrices of $\mathbf{G}^{-1}$, so the first part 
of line 6 is no more complex than matrix inversion. Once the matrix $\mathbf{V}$ is known, the 
function $v(w)$ is known, as is $\dot{v}(w)$, as is the matrix $\dot{\mathbf{V}}$. Both $\mathbf{U}$
and $\dot{\mathbf{V}}$ are square matrices, so we can finally achieve the goal of writing 
Hermite-Ostrogradsky reduction in terms of linear algebra, 
$[\mathbf{w}']=(\mathbf{U}+\frac{1}{m}\dot{\mathbf{V}})\cdot\mathbf{w}'$,
as in line 22 of Alg. 1.  The halting condition is already in sight, but there is at least 
one more significant chance of failure.


Lines 11-12 validate that $\rho$ is well-chosen with regard to integrand $dI$. As the 
length of $\mathbf{x}_0$ is $d+\Delta$, we can anticipate needing to calculate no more 
than $D=d+\Delta$ reductions, before the set $\{\mathbf{x}_0,\mathbf{x}_1,\ldots,\mathbf{x}_D\}$
must contain at least one linear dependency. Depending on symmetry, the first linear dependency
may occur for $D'+1$ vectors, with $D'<D$. For this reason, we compute and validate derivatives 
on the fly. To reach $n=D'$ and exit the loop after line 17, we must always find, for each $n \le D'$, 
that $\partial_{\alpha}^n\frac{dI}{dt}=\sum_{m=0}^n\frac{w_m}{\rho^{m+1}}$. In other words,
the partial fraction decomposition of $\partial_{\alpha}^n\frac{dI}{dt}$ must yield no more 
than $n+1$ degree-bounded numerators $w_m=\sum_{k=0}^{d+\Delta}w_{m,k}z^k$. During each
loop, if the check on line 11 passes, then the numerators $w_m$ are calculated on line 13. 
The algorithm enters a second loop on line 14, and calculates the coefficients $w_{m,k}$ on 
line 15. A third loop is entered on line 20 via line 16, and fourth loops occur on lines 
21 and 22 as matrix multiplication. When this process repeats without error, the algorithm
approaches halting on success. Eventually, on line 17, a linear dependence between the 
$\mathbf{x_n}$ will be found and set to $\mathbf{A}$. The loop breaks, and shortly 
thereafter, the function returns a positive result. 

The subfunction defined on line 19 deserves a closer look. It manages the iterative part 
of Hermite-Ostrogradsky reduction by taking a $\mathbf{w}_m\sim \frac{w_m}{\rho^{m+1}}$ from
its initial form to its minimal form  $\mathbf{w}_f\sim \frac{w_f}{\rho}$. Lines 21-22 involve
$2m$ matrix multiplications, only $m$ of which are strictly necessary. The sow step on line 
21 keeps track of partial certificates by hiding them in the memory. Later, on line 18,
they can be retrieved and summed. When subtotaled by index $n$ and dotted with 
$\mathbf{A}$, the partial certificates $\mathbf{\Xi}$ determine a total certificate 
$\Xi$ such that ${\mathcal{A} \circ \frac{dI}{dt}-\partial_t \Xi=0} $. If the reduction 
reaches large recursion depth, speed and memory usage can suffer, so sometimes we choose to
run the algorithm without calculating certificates. However, if there is any doubt or misunderstanding, 
having a certificate assures quality. 
 
In practice, we also care about the complexity of Alg. 1. How does it scale relative to the 
degree bound $D$? Naively, the answer is that Alg. 1 is polynomial-time, roughly
$\mathcal{O}(D^{3+\omega})$, where $1<\omega<3$ is the complexity cost of matrix multiplication. 
In many cases three loops are not necessary, and the figure without partial fraction
decomposition  $\mathcal{O}(D^{2+\omega})$ looks even better\footnote{Both complexity 
estimates are comparable to estimates for similar algorithms \cite{COMPLEXITY2010}.}. The problem is that 
matrix multiplication generates larger and larger polynomial data as recursion depth of 
the $k$-loop increases. This causes slow-down during the reduction phase, 
and eventually makes nullspace calculation prohibitively slow. If $2<\omega'<3$ is the complexity 
of finding a null vector, then $\mathcal{O}(D^{1+\omega'})\sim \mathcal{O}(D^{2+\omega})$, 
and we could expect this step to dominate time dynamics. In many calculations, indeed it does. 
Heuristically, the complexity of Alg. 1 is much better described by an exponential or 
super exponential figure. If necessary, more accurate estimates can follow from
testing on random inputs. Of course, this requires an implementation for some particular
choice of inputs. 

In the previous section, we have already revealed that functions $\sin(t)$ and $\cos(t)$ make 
a good basis for an application of Creative Telescoping. The choice $z(t)= \cos(t)+i \sin(t)$ 
is even better because $\partial_t z = i z$. If we allow negative powers of $z$ by requiring
that ${\rho \in Q\mathbb[\![i,\alpha,z,z^{-1}]\!]}$ then we can represent $\sin(t)$ and 
$\cos(t)$ as $\frac{1}{2i}(z-\frac{1}{z})$ and $\frac{1}{2}(z+\frac{1}{z})$ respectively.
When $\rho$ is chosen as $\rho = 1-\alpha \Phi$ with $\Phi \in Q\mathbb[\![i,z,z^{-1}]\!]$,
and when $\frac{dI}{dt}=\frac{1}{\rho}$ we can make slight edits to Alg. 1 and arrive at 
the algorithm \texttt{ExpToODE}. In fact, we need only make changes to account for negative 
powers. Assuming that $\Phi=\sum_{\tilde{d}}^{d}c_n z^n$ then $\text{bideg}(\rho)=(\tilde{d},d)$. 
As $\Delta=0$, we can set $\text{bideg}(u)=\text{bideg}(v)=\text{bideg}(\rho)$,
which implies $\text{bideg}(w)=(2\tilde{d},2d)$
and $\text{dof}(u)+\text{dof}(v)=2(d+\tilde{d})+1$ when $v$ has no constant term. Thus 
the reduction constraint $w=\rho u -\dot{\rho}v$ leads to a square matrix $\mathbf{G}$
of size $2(d+\tilde{d})+1$. After line 5 the algorithm is much the same, except that 
the degree bound is changed to $D=d+\tilde{d}+1$.

\newpage
\begin{figure}[t]
$\mathbf{G}=\begin{bmatrix}
\begin{array}{c  |c c }
   &  \phantom{\;0\;}  & \phantom{\;0\;}  \\
\hline
    \;\; \mathbf{I} \;\;\;   &   &   \\
\hline   
     &   &   
\end{array}
\end{bmatrix}
+\alpha \begin{bmatrix}
\begin{array}{c |c c }
  \phantom{\;x\;}  &  LT &   \\
\hline
   &      & \\
\hline   
   &     &   UT
\end{array}
\end{bmatrix},
\;\;\;
\mathbf{G}=
\begin{bmatrix}
\begin{array}{c |c c }
  LT &  LT   \\
\hline
 UT  &     UT 
\end{array}
\end{bmatrix}
+\alpha \begin{bmatrix}
\begin{array}{c  |c  }
 \mathbf{\;\;I\;\;}  & \phantom{\;0\;}    \\
\hline
     & 
\end{array}
\end{bmatrix}$
\begin{center}
UT for upper triangle, LT for lower triangular.
\end{center}
\label{fig:gmatrices}
\caption{Forms for $\mathbf{G}$ matrices: \texttt{ExpToODE} (left), \texttt{HyperellipticToODE} (right).}
\end{figure}

The form $\rho = 1-\alpha \Phi$ is simple enough to prove that \texttt{ExpToODE} halts
on success for any valid input $\Phi$. As in Fig. \ref{fig:gmatrices}, decompose $\mathbf{G}$ by 
$\mathbf{G}=\mathbf{G}_0+\alpha\mathbf{G}_{\alpha}$, with both $\mathbf{G}_{0}$ and
$\mathbf{G}_{\alpha}$ members of $\mathbb{Q}[\![i]\!]$. Under its first 
$\text{dof}(u)=d+\tilde{d}+1$ columns, matrix $\mathbf{G}_{0}$ contains an
identity submatrix, $\mathbf{I}$, and these are the only non-zero values of $\mathbf{G}_{0}$. 
Expanding the determinant by minors determines a bound that 
$\det(\mathbf{G})=\det(\mathbf{D})+\mathcal{O}(\alpha^{d+\tilde{d}+1})$, 
with $\mathbf{D}$ a $\mathbf{G}$-submatrix on the space complement to
that of $\mathbf{I}$. In the $\mathbf{D}$ subspace, $\mathbf{G}_{0}=0$, so 
$\mathbf{D} \subset \alpha \mathbf{G}_{\alpha}$. Matrix $\mathbf{D}$
has only one non-zero diagonal, and it is possible to identify that 
$\det(\mathbf{D}/\alpha)=d^d \tilde{d}^{\tilde{d}} a b$
when $\Phi = a z^{-\tilde{d}} + \ldots + b z^d$. Finally we obtain 
$\det(\mathbf{G})\propto \alpha^{d+\tilde{d}}$ or $\det(\mathbf{G})\neq 0$,
thus an inversion error is never thrown. As for derivatives, they can be written 
naively in terms of the central binomial coefficients 
$\partial_{\alpha}^n\frac{dI}{dt}=\binom{2n-2}{n-1}\frac{\Phi^n}{\rho^{n+1}}$. 
A valid partial fraction decomposition must always exist, and a mismatch error between 
$dI$ and $\rho$ is never thrown. 

With a little more effort, it is possible to obtain a closed form,
$\partial_{\alpha}^n\frac{dI}{dt}=\frac{n!}{\alpha^n}\sum_{k=0}^n\binom{n}{k}\frac{(-1)^{n+k}}{\rho^{k+1}}$,
for the partial fraction expansion. This particular form is an amazing fortuity because it allows 
elimination of the loop on line 14. Once we have computed the reduction of $\frac{1}{\rho^{n}}$, 
we do not need to calculate it again during subsequent iteration of the $n$ loop. In the optimized
algorithm, each $\mathbf{x}_n$ can be calculated as the reduction of $\frac{1}{\rho^n}$. Once a 
linear dependency is found, it must exist in any linear transformation $\mathbf{x}'=\mathbf{M}\cdot \mathbf{x} $
so long as $\det(\mathbf{M}) \neq 0$. In this case, $\mathbf{M}$ is a lower triangular matrix
with non-zero elements $m_{n,k}=n!\binom{n}{k}\frac{(-1)^{n+k}}{\alpha^n}$, so $\det(\mathbf{M})\neq0$
always holds true. The coefficients $\mathbf{A}$ can then be found from the nullspace of 
$\mathbf{x}'$ and returned as usual. This is a nice optimization that helps \texttt{ExpToODE} 
to be an extremely successful workhorse, as we have already seen in the preceding section.
 
The other easy choice, not so obvious, is that $z(t)=q(t)$ with $\partial_t z(t)=p(t)$,
where $q(t)$ and $p(t)$ are solutions to Hamilton's equations given a hyperelliptic
form $\alpha=2H=p^2+2V(q)$, and $V(q)=\sum_{n=1}^{d}c_n q^n$ The period is not a single 
number, but a function $T(\alpha)=\oint dt = \oint \frac{dq}{p}$, with derivatives 
$\partial_{\alpha}^n\frac{dI}{dt} = p \;\partial_{\alpha}^n \frac{1}{p}= \frac{(2n-1)!!}{(-2)^n}\frac{1}{p^{2n}}$,
as is proven using the chain rule with $\partial_{\alpha}p=\frac{1}{2p}$. Even though 
$p$ is not a polynomial of $q$ we can choose $\rho=p$ if we rewrite that,
\begin{eqnarray}
\frac{[w]}{\rho^{2n}}=\Big(u-\frac{1}{2n+1}\partial_q v\Big)\frac{1}{\rho^{2n}} = 
\frac{w}{\rho^{2n+2}}-\partial_t\bigg( \frac{v}{(2n+1)\rho^{2n+1}}\bigg) 
\;\;\; \iff \;\;\; 
 w = \rho^2 u -\dot{\rho} v \nonumber.
\end{eqnarray}
This statement is not as straightforward. After applying chain rule,
${\partial_t v = (\partial_t q) \partial_q v= \rho \partial_q v}$, the factor $\rho$ 
cancels with the extra $+1$ power in the denominator. The degree bounding behaves 
as if $\Delta=-1$, and all reductions must be carried out to the zeroth power 
of $\rho$. Also, the denominators in lines 21-22 need to be changed to replace 
$k$ with $2k+1$. Other than these few minor changes, the resulting algorithm
\texttt{HyperellipticToODE} exactly follows the template of Alg. 1.
 
For essentially the same reason as before, \texttt{HyperellipticToODE} is also a
rigorously proveable algorithm. However, in this case the decomposition 
$\mathbf{G}=\mathbf{G}_0+\alpha \mathbf{G}_{\alpha}$ has its identity submatrix
in $\mathbf{G}_{\alpha}$, consequently $\det(\mathbf{G})=\alpha^{d-1}+$
smaller powers of $\alpha$. The case $d=1$ is of no interest, and the case $d=2$
always results in $\mathcal{A} = \partial_{\alpha}$, as it must. In the quadratic
case, the Riemann surface associated to $2H=p^2+a q+b q^2$ is the genus zero 
harmonic hyperboloid, with $T(\alpha)=T_0=$ constant. For $d=3$ and $d=4$, usually
the level curves of $H$ are elliptic and $\mathcal{A}\circ T(\alpha)=0$ has 
two solutions, corresponding to orthogonal periods along real and complex time 
dimensions. For $d>4$, $\mathcal{A}$ typically contains 2$g$+1 terms, with 
genus $g=\lfloor (d-1)/2 \rfloor>1$. These cases are the proper hyperelliptic 
curves, whose Riemann surfaces admit $g$ distinct real periods and 
$g$ distinct complex periods. In this sense, the proof of algorithm 
\texttt{HyperellipticToODE} is also a reproof of Fuchs's theorem on the 
periods of hyperelliptic curves\cite{GRAY2010}, 
though with slightly different conventions.

Last but not least, we have the case of $\texttt{DihedralToODE}$, a very effective 
algorithm, but not entirely proveable. For inputs we assume a form,
\begin{eqnarray}
\alpha=2H = \bigg(\sum_{n=1}^{d_1}c_{1,n}\lambda^{n}\bigg)
+\bigg(\sum_{n=0}^{d_2}c_{2,n}\lambda^{n}\bigg)\lambda^{\frac{m}{2}}\cos(m\phi)
= h_1+h_2\cos(m\phi),  \nonumber 
\end{eqnarray}
so chosen because it can easily be solved to eliminate $\phi$ dependence,
$\cos(m\phi)=\frac{\alpha-h_1}{h_2}$. Relative to Alg. 1, the choice 
$z=\lambda$ is somewhat surprising, but it follows follows from the observation
that $\dot{\lambda}^2$ and $\ddot{\lambda}$ can both be written as $\lambda$ 
polynomials, 
\begin{eqnarray}
\dot{\lambda}^2 = \frac{-m^2}{4}(\alpha-h_1-h_2)(\alpha-h_1+h_2) \nonumber
\;\;\;\;\;\; \text{and} \;\;\;\;\;\;
\ddot{\lambda} = \frac{m^2}{4}\big((\alpha-h_1)(\partial_{\lambda}h_1)+h_2(\partial_{\lambda}h_2)\big) \nonumber.
\end{eqnarray}
Similarly, the angular derivative  
$\dot{\phi}=\big(h_2(\partial_{\lambda}h_1)+(\alpha-h_1)(\partial_{\lambda}h_2)\big)/(2 h_2)$
works out to be a rational function of $\lambda$ polynomials, and we can usually choose 
$\rho=\big(h_2(\partial_{\lambda}h_1)+(\alpha-h_1)(\partial_{\lambda}h_2)\big)$. 
Again, the basic assumption of Hermite-Ostrogradsky reduction must be rewritten,
\begin{eqnarray}
\frac{[w]}{\rho^{n}}
=\Big(u-\frac{1}{n}(\ddot{\lambda}+\dot{\lambda}^2\partial_q v)\Big)\frac{1}{\rho^{n}} 
= \frac{w}{\rho^{n+1}}-\partial_t\bigg( \frac{\dot{\lambda}v}{n\rho^{n}}\bigg) 
\;\;\; \iff \;\;\; 
 w = \rho u -\dot{\lambda}^2 v (\partial_{\lambda}\rho) \nonumber.
\end{eqnarray}
As before $\frac{dI}{dt}=1$ and subsequent derivatives are calculated as 
$\partial_{\alpha}^n\frac{dI}{dt} = \dot{\phi}\partial_{\alpha}^n\dot{\phi}^{-1}$
via the chain rule with $\partial_{\alpha}\lambda = (2\dot{\phi})^{-1}$. Degree
bounding is possible with $d=\deg(\rho)$ and $\Delta=\deg(\dot{\lambda}^2)-1$. 
Thereafter, the best practice is to check for errors while computing and 
to require \textit{post hoc} quality analysis on the outputs. When 
$\mathcal{A}\circ\frac{dI}{dt}-\partial_t \Xi=0$, minimum rigor is still achieved.

Something very interesting happens with the I/O map of \texttt{DihedralToODE}. For a
typical input with $d_1=\lfloor m/2 \rfloor$, $d_2=0$, and $d=\lfloor (m-2)/2 \rfloor$,
the ouput $\mathcal{A}$ has $2d+1$ terms. We are tempted to identify $g'=d=\lfloor(m-2)/2\rfloor$
as the genus of the Riemann surfaces associated with $\alpha=2H$. This would foolishly contradict
old and well-known theorems, which typically predict genus as a quadratic function of 
degree. Instead, what appears to happen is that the choice of dihedral symmetry gives 
the Riemann surface unexpected isoperiodicities, thus the number of distinct periods is 
fewer than the number of distinct homology classes. Since the dimension of $\mathcal{A}$ 
counts periods, it cannot yield the genus without a complementary symmetry analysis.
We do not necessarily need homology/cohomology to complete the proof of Section II, 
but will continue working toward these advanced topics anyways. 

% Finalize algorithms. Output data.

\section{Finishing the Proof}
\begin{table}[t]
\begin{center}
\captionof{table}{Period analysis of a few simple hyperelliptic Hamiltonians.}  
\label{tab:Hyperelliptic}
{\renewcommand{\arraystretch}{2}%
\begin{tabularx}{\textwidth}{ c l | c l | c l | c }
\hline \hline
\hspace{0.1cm} & 
$2V(q)=$ & \hspace{0.1cm} 
& $\mathcal{A}=$\hspace{7cm} &\hspace{0.1cm} & $T(\alpha)=$ & \;\;$s$\; \\
\hline
& $-\frac{2\sqrt{3}}{3}q^3$ \hspace{0.5cm} & &
$5 - 36 (1 - 2 \alpha)\partial_{\alpha}- 36  \alpha(1 - \alpha)\partial_{\alpha}^2$   & &
$2\pi\,_2F_1 \Big[ \genfrac..{0pt}{}{
\frac{1}{6},\frac{5}{6}
}{1}\Big| \alpha \Big]$  & $6$    \\
& $-\frac{1}{4}q^4$  & &
$3 - 16 (1 - 2 \alpha)\partial_{\alpha}- 16  \alpha(1 - \alpha)\partial_{\alpha}^2$   & &
$2\pi\,_2F_1 \Big[ \genfrac..{0pt}{}{
\frac{1}{4},\frac{3}{4}
}{1}\Big| \alpha  \Big]$  & $4$   \\
& $-\frac{6\sqrt{15}}{125}q^5$  & &
$1701 - 2000 (10 - 207 \gamma)\partial_{\gamma}$ 
  & &
$2\pi\,_4F_3 \Big[ \genfrac..{0pt}{}{
\frac{1}{10},\frac{3}{10},\frac{7}{10},\frac{9}{10}
}{\frac{1}{3},\frac{2}{3},1}\Big| \alpha^3 \Big]$ &     \\
& \;\;\;\tiny(cont.)  & & \hspace{0.5cm}
$-\ldots - \ldots - 90000 \gamma^3 (1 - \gamma) \partial_{\gamma}^4 $
  & & &     \\
& $-\frac{4}{27}q^6$  & &
$5 - 36 (1 - 2 \beta)\partial_{\beta}- 36\beta(1 - \beta)\partial_{\beta}^2$  & &
$2\pi\,_2F_1 \Big[ \genfrac..{0pt}{}{
\frac{1}{6},\frac{5}{6}
}{1}\Big| \alpha^2 \Big]$    & $6$  \\
& $-\frac{50\sqrt{35}}{2401}q^7$  & &
$12065625 - 57624 (3136 - 815625 \epsilon)\partial_{\epsilon}$  & &
$2\pi\,_6F_5 \Big[ \genfrac..{0pt}{}{
\frac{1}{14},\frac{3}{14},\frac{5}{14},\frac{9}{14},\frac{11}{14},\frac{13}{14}
}{\frac{1}{5},\frac{2}{5},\frac{3}{5},\frac{4}{5},1}\Big| \alpha^5 \Big]$ \;    \\
& \;\;\;\tiny(cont.)  & & \hspace{0.5cm}
$-\ldots - \ldots - \ldots - \ldots $
  & & &     \\
& \;\;\;\tiny(cont.)  & & \hspace{1.0cm}
$ - 4705960000\epsilon^5 (1 - \epsilon)\partial_{\epsilon}^6 $
  & & &     \\
& $-\frac{27}{256}q^8$  & &
 $945 -256 (32 - 675 \gamma)\partial_{\gamma} $ & &
$2\pi\,_4F_3 \Big[ \genfrac..{0pt}{}{
\frac{1}{8},\frac{3}{8},\frac{5}{8},\frac{7}{8}
}{\frac{1}{3},\frac{2}{3},1}\Big| \alpha^3 \Big]$     \\
& \;\;\;\tiny(cont.)  & & \hspace{0.5cm}
$-\ldots - \ldots - 36864 \gamma^3 (1 - \gamma)\partial_{\gamma}^4  $
  & & &     \\
\end{tabularx}}
\rule{\textwidth}{.01cm}
Dots indicate dropped terms. Variables are $\beta=\alpha^2, \gamma=\alpha^3, \epsilon=\alpha^5$. 
\end{center}
\end{table}

Recall that in Section II, we treated the problem of series reversion as too difficult, 
and simply accepted assertions for period function left factors. How can we be sure 
that such a factorization even exists? Consider the general form 
$\alpha=2H=2\lambda - \lambda^{d/2} \Phi$, or if you like, 
$\alpha \Phi^{\frac{2}{d-2}} = 2x+x^{d/2}$ with $x=\lambda \Phi^{\frac{2}{d-2}}$.
The formal solution is either, 
\begin{eqnarray}
x=\alpha \Phi^{\frac{2}{d-2}}\bigg(\frac{1}{2}
+\sum_{n=1}^{\infty}c_n(\alpha^{\frac{d-2}{2}} \Phi )^{n} \bigg) 
\;\;\;\; \text{or} \;\;\;\;
\lambda = \alpha \bigg(\frac{1}{2} +\sum_{n=1}^{\infty}c_n(\alpha^{\frac{d-2}{2}} \Phi )^{n} \bigg), \nonumber
\end{eqnarray}
so that chosen exponents $j=\frac{d-2}{2}$ and $k=1$ satisfy 
$\alpha^{1+j n}\Phi^{\frac{2}{d-2}+kn}=\alpha^{\frac{d}{2}+j(n-1)}\Phi^{\frac{d}{d-2}+k(n-1)}$. 
Then we can solve either constraint on coefficients of $\alpha^{\frac{d-2}{2}}\Phi$ to obtain 
$c_n$ as a function of $c_{n-1}$ with terminal $c_0$. It is possible, not guaranteed that the $c_n$
obey a hypergeometric recursion, but that property is not necessary to proceed. We only need to 
understand that the $c_n$ \textit{do not} depend on choice of $\Phi$ and that they \textit{do} 
depend on a choice of $d$. This allows us to produce a factored from for the period function, 
\begin{eqnarray}
T(\alpha) =\oint 2\partial_{\alpha}\lambda d\phi = \oint\bigg(\sum_{n=0}^{\infty} a_n(\alpha^{\frac{d-2}{2}} \Phi)^{n} \bigg)d\phi
=L_d\Big(\alpha\Big) \star I_d\Big(\alpha\Big), \nonumber  
\end{eqnarray} 
with $I_d(\alpha) = \oint (1-\alpha^{\frac{d-2}{2}} \Phi)^{-1}d\phi$. For odd powers of
$d$, odd powers of $\Phi$ integrate to zero, thus $T(\alpha)$ always expands in whole powers of $\alpha$.
The right factor is analyzable via \texttt{ExpToODE}. We have already seen that $\Phi=Q^{2}$ leads 
to hypergeometric $I_d(\alpha)$, and we can also calculate the hypergeometric parameters of the $I(\alpha)$ arising
from $\Phi=Q^{2n}$, for any $n$. This allows proof by concordance 
between \texttt{ExpToODE} and \texttt{HyperellipticToODE}. If $2H=p^2+q^2-c_d q^d$ has 
hypergeometric $T(\alpha)$, then the left factor $L$ must also be hypergeometric. We can find its 
parameters easily by comparing the parameters of $I_d(\alpha)$ and $T(\alpha)$. Admittedly, the 
strategy is convoluted, but it trades the difficulties of series reversion for a routine 
algorithmic calculation. Ultimately, the main reason to prefer this constructive, data driven 
proof is that it allows us to gain more familiarity with the algorithms of Section III. 

Now we just turn on the computer, instantiate the algorithms, calculate a few differential 
equations, and list them in Table \ref{tab:Hyperelliptic}.  They are all hypergeometric, and 
they should\footnote{Unfortunately when typing out long equations typos are often made. Write
the author if you notice any!} exactly match previously asserted forms. As an example to show how the Hadamard factoring 
actually works, we use the output of \texttt{ExpToODE}[$\sin(\phi)^{10}$] to determine that 
 \begin{eqnarray}
I_5(\alpha) = \oint \frac{d\phi}{1-\alpha^3\sin(\phi)^{10}} = 
2\pi\,_5F_4 \bigg[ \genfrac..{0pt}{}{
\frac{1}{10},\frac{3}{10},\frac{1}{2},\frac{7}{10},\frac{9}{10}
}{\frac{1}{5},\frac{2}{5},\frac{3}{5},\frac{4}{5}}\bigg| \alpha^3 \bigg]
= 2\pi \sum_{n=0}^{\infty} \frac{1}{2^{10n}}\binom{10n}{5n}\alpha^{3n}.
\nonumber
\end{eqnarray}
The complement $\{\frac{1}{10},\frac{3}{10},\frac{1}{2},\frac{7}{10},\frac{9}{10}\}/
\{\frac{1}{10},\frac{3}{10},\frac{7}{10},\frac{9}{10}\}=\{\frac{1}{2}\}$ determines
a lower parameter $\frac{1}{2}$ to go along with $\{\frac{1}{3},\frac{2}{3}\}$ from 
$T(\alpha)$. The extra parameter $1$ we get for free, which leaves the upper parameters
only. Since no cancellation occurs, they will just be the lower parameters of $I_5(\alpha)$.
Now better than assertion, for $d=5$, we derive that
\begin{eqnarray}
L_d(\alpha)= \,_4F_3 \bigg[ \genfrac..{0pt}{}{
\frac{1}{5},\frac{2}{5},\frac{3}{5},\frac{4}{5}
}{\frac{1}{3},\frac{1}{2},\frac{2}{3}}\bigg| \alpha^3 \bigg]
=  \sum_{n=0}^{\infty} \bigg(\frac{2^{2}3^3}{5^5}\bigg)^n\binom{5n}{2n}\alpha^{3n}.
\nonumber
\end{eqnarray}   
A similar explicit calculation proves for any odd $d$ that $a_n \propto \binom{d n}{2n}$ and 
for any even $d$ that $a_n \propto \binom{d n/2}{n}$. It would be nice to have a second proof  
(perhaps using series reversion) especially as calculating Annihilators becomes time intensive
as $d$ increases. However, it is no bother to do the work up to $d=8$. We have done this 
work, so we can assure the wide results of Section II.

\begin{table}[t]
\begin{center}
\captionof{table}{Period analysis of a few simple dihedral Hamiltonians.}  
\label{tab:Dihedral}
{\renewcommand{\arraystretch}{2}%
\begin{tabularx}{\textwidth}{ c | c | c l | c l | c }
\hline \hline
\;$m$\; &  \;\;\;$c_m$\;\;\; & \hspace{0.1cm} 
& $\mathcal{A}=$\hspace{7.7cm} &\hspace{0.1cm} & $T(\alpha)=$ & \;\;$s$\;\; \\
\hline
\multicolumn{7}{c}{$\alpha = 2H = 2\lambda - c_m \lambda^{m/2}\cos(m\phi) $ } \\
\hline
3 & $\frac{4\sqrt{6}}{9}$ & &
$2 - 9 (1 - 2 \alpha)\partial_{\alpha}- 9  \alpha(1 - \alpha)\partial_{\alpha}^2$   & &
$2\pi\,_2F_1 \Big[ \genfrac..{0pt}{}{
\frac{1}{3},\frac{2}{3}
}{1}\Big| \alpha \Big]$  & 3    \\
4 & $1$  & &
$3 - 16 (1 - 2 \beta)\partial_{\beta}- 16  \beta(1 - \beta)\partial_{\beta}^2$   & &
$2\pi\,_2F_1 \Big[ \genfrac..{0pt}{}{
\frac{1}{4},\frac{3}{4}
}{1}\Big| \alpha^2  \Big]$  & 4   \\
5 & $\frac{24\sqrt{30}}{125}$  & &
$216 - 250 (5 - 108 \gamma)\partial_{\gamma}$ 
  & &
$2\pi\,_4F_3 \Big[ \genfrac..{0pt}{}{
\frac{1}{5},\frac{2}{5},\frac{3}{5},\frac{4}{5}
}{\frac{1}{3},\frac{2}{3},1}\Big| \alpha^3 \Big]$ &     \\
 & \tiny(cont.)  & & \hspace{0.5cm}
$-\ldots - \ldots - 5625 \gamma^3 (1 - \gamma) \partial_{\gamma}^4 $
  & & &     \\
6 & $\frac{32}{27}$  & &
$40 - 9 (27 - 680 \delta)\partial_{\delta}$  & &
$2\pi\,_4F_3 \Big[ \genfrac..{0pt}{}{
\frac{1}{6},\frac{1}{3},\frac{2}{3},\frac{5}{6}
}{\frac{1}{4},\frac{3}{4},1}\Big| \alpha^4 \Big]$   &   \\
 & \tiny(cont.)  & & \hspace{0.5cm}
$-\ldots - \ldots - 1296 \delta^3 (1 - \delta) \partial_{\delta}^4 $
  & & &     \\
7 & \;$\frac{400\sqrt{70}}{2401}$\;  & &
$450000 - 8232 (343 - 93750 \epsilon)\partial_{\epsilon}$  & &
$2\pi\,_6F_5 \Big[ \genfrac..{0pt}{}{
\frac{1}{7},\frac{2}{7},\frac{3}{7},\frac{4}{7},\frac{5}{7},\frac{6}{7}
}{\frac{1}{5},\frac{2}{5},\frac{3}{5},\frac{4}{5},1}\Big| \alpha^5 \Big]$ \;    \\
 & \tiny(cont.)  & & \hspace{0.5cm}
$-\ldots - \ldots - \ldots - \ldots $
  & & &     \\
 & \tiny(cont.)  & & \hspace{1.0cm}
$ - 73530625 \epsilon^5 (1 - \epsilon)\partial_{\epsilon}^6 $
  & & &     \\
8 & $\frac{27}{16}$  & &
 $25515 -160 (1024 - 341901 \zeta)\partial_{\zeta} $ & &
$2\pi\,_6F_5 \Big[ \genfrac..{0pt}{}{
\frac{1}{8},\frac{1}{4},\frac{3}{8},\frac{5}{8},\frac{3}{4},\frac{7}{8}
}{\frac{1}{6},\frac{1}{3},\frac{2}{3},\frac{5}{6},1}\Big| \alpha^6 \Big]$     \\
 & \tiny(cont.)  & & \hspace{0.5cm}
$-\ldots - \ldots - \ldots - \ldots  $
  & & &     \\
 & \tiny(cont.)  & & \hspace{1.0cm}
$ - 5308416\zeta^5 (1 - \zeta)\partial_{\zeta}^6 $
  & & &     \\ \hline 
\multicolumn{7}{c}{$\alpha = 2H = 2\lambda - c_m \lambda^{m}\big(1+\cos(2m\phi)\big) $ } \\  
\hline
2 & $\frac{1}{2}$  & &
$1 - 4 (1 - 2 \alpha)\partial_{\alpha}- 4  \alpha(1 - \alpha)\partial_{\alpha}^2$   & &
$2\pi\,_2F_1 \Big[ \genfrac..{0pt}{}{
\frac{1}{2},\frac{1}{2}
}{1}\Big| \alpha \Big]$  & $2$    \\
3 & $\frac{16}{27}$  & &
$2 - 9 (1 - 2 \beta)\partial_{\beta}- 9  \beta(1 - \beta)\partial_{\beta}^2$   & &
$2\pi\,_2F_1 \Big[ \genfrac..{0pt}{}{
\frac{1}{3},\frac{2}{3}
}{1}\Big| \alpha  \Big]$  & $3$   \\
4 & $\frac{27}{32}$  & &
$27 - 8 (16 - 351 \gamma)\partial_{\gamma}$ 
  & &
$2\pi\,_4F_3 \Big[ \genfrac..{0pt}{}{
\frac{1}{4},\frac{1}{2},\frac{1}{2},\frac{3}{4}
}{\frac{1}{3},\frac{2}{3},1}\Big| \alpha^3 \Big]$ &     \\
 & \tiny(cont.)  & & \hspace{0.5cm}
$-\ldots - \ldots - 576 \gamma^3 (1 - \gamma) \partial_{\gamma}^4 $
  & & &     \\
5 & $\frac{4096}{3125}$  & &
$384 - 375 (5 - 128 \delta)\partial_{\delta}$ 
  & &
$2\pi\,_4F_3 \Big[ \genfrac..{0pt}{}{
\frac{1}{5},\frac{2}{5},\frac{3}{5},\frac{4}{5}
}{\frac{1}{4},\frac{3}{4},1}\Big| \alpha^4 \Big]$ &     \\
 & \tiny(cont.)  & & \hspace{0.5cm}
$-\ldots - \ldots - 10000 \delta^3 (1 - \delta) \partial_{\delta}^4 $
  & & &     \\
\end{tabularx}}
\rule{\textwidth}{.01cm}
Dots indicate dropped terms. Variables are 
$\beta=\alpha^2, \gamma=\alpha^3, \delta=\alpha^4, 
\epsilon=\alpha^5, \zeta=\alpha^6$. 

See Appendix A for computer proofs.
\end{center}
\end{table}


The outstanding case $\alpha=2H=p^2+q^2-\frac{64}{27}p^{4}q^{6}$ should not 
be forgotten. It involves a difficult decic perturbation, which we will deal with 
circuitously. Again, start instead with a simple hyperelliptic decic, 
$\alpha =2H=p^2+q^2-\frac{256}{2135}q^{10}$. The computer tells us via 
\texttt{HyperellipticToODE} that the period function is annihilated by
\begin{eqnarray}
\mathcal{A}=189 - 125 (15 - 368 \delta)\partial_{\delta} - 125 \delta (335 - 1144 \delta)\partial_{\delta}^2 
- 10^4 \delta^2 (5 - 8 \delta)\partial_{\delta}^3 - 10^4 \delta^3(1 - \delta)\partial_{\delta}^4, \nonumber 
\end{eqnarray}
with $\delta=\alpha^4$, and we can check the certificate if doubt about $\mathcal{A}$ 
persists\footnote{The certificate sums terms up to $q^{35}$ in its numerator,
so a computer will probably be necessary.}. The hypergeometric parameters of $T(\alpha)$ are $\{\frac{1}{10},\frac{3}{10},\frac{7}{10},\frac{9}{10}\}$ upper and 
$\{\frac{1}{4},\frac{3}{4},1\}$ lower. Meanwhile, the parameters of right
factor $I_{10}$ are the same as $I_5$, so by cancellation, we calculate
parameters for the shared left factor $L_{10}$. They are 
$\{\frac{1}{5},\frac{2}{5},\frac{3}{5},\frac{4}{5}\}$ upper and
$\{\frac{1}{4},\frac{1}{2},\frac{3}{4}\}$ lower. According to 
$\texttt{ExpToODE}$ with $\Phi=(z+\frac{1}{z})^5 z^3$ these parameters also 
determine the $L_{10}$ expansion coefficients $a_n\propto\binom{5n}{n}$, 
up to choice of scale for $\alpha$. Writing the right factor as,
\begin{eqnarray}
I(\alpha) = \oint \frac{d\phi}{1-\alpha^4(2^{10}\cos(\phi)^6\sin(\phi)^4)}
=\sum_{n=0}^{\infty}\sum_{k=0}^{4n} \binom{6n}{n+k}\binom{4n}{k}(-1)^k \alpha^{4n}, \nonumber
\end{eqnarray}\FloatBarrier 
\noindent allows us to uncover yet another unexpected binomial identity,
\begin{eqnarray}
\binom{3n}{n}\binom{6n}{3n}
=\binom{5n}{n}\sum_{k=0}^{4n} \binom{6n}{n+k}\binom{4n}{k}(-1)^k. \nonumber
\end{eqnarray}
This identity essentially explains why $\alpha=2H=p^2+q^2-\frac{64}{27}p^{4}q^{6}$ 
should have period $T_6(\alpha^4)$. If there is any doubt about veracity, 
the identity can be double checked using Zielberger's algorithm for 
hypergeometric summations. 

\begin{table}[t]
\begin{center}
\captionof{table}{Canonical models $H_s$ for which 
$\mathcal{A}_s=(s-1)-s^2(1-2\alpha)\partial_{\alpha}-s^2\alpha(1-\alpha)\partial_{\alpha}^2 $.}  
\label{tab:Canonical}
{\renewcommand{\arraystretch}{2}%
\begin{tabularx}{\textwidth}{c  l  l | c l  l | l}
\hline \hline
\;\;\;&$2H_2=$ & $p^2+q^2-p^2 q^2$ & \;\;\;\; & $T_2(\alpha)=$ &
$2\pi\,\tFoIn{\frac{1}{2},\frac{1}{2}}{1}{\alpha}$ \hspace{1.0cm} & \;\;\; genus 1   \\
\;\;\;&$2H_3=$ & $p^2+q^2-(\frac{4}{27})^{\frac{1}{2}} \; ( q^3-3 p^2 q)$ \hspace{1.0cm} 
&& $T_3(\alpha)=$ &
$2\pi\,\tFoIn{\frac{1}{3},\frac{2}{3}}{1}{\alpha}$  & \;\;\; genus 1 \\
\;\;\;&$2H_4=$ & $p^2 + q^2 -\frac{1}{4}q^4$ 
&& $T_4(\alpha)=$ &   
$2\pi\,\tFoIn{\frac{1}{4},\frac{3}{4}}{1}{\alpha}$  & \;\;\; genus 1 \\
\;\;\;&$2H_6=$ & $p^2 + q^2 -\frac{2\sqrt{3}}{9}q^3$ 
&& $T_6(\alpha)=$ &
$2\pi\,\tFoIn{\frac{1}{6},\frac{5}{6}}{1}{\alpha}$  & \;\;\; genus 1 \\
%\end{tabularx}}
%{\renewcommand{\arraystretch}{2}%
%\begin{tabularx}{\textwidth}{ l  l | l   l }
\hline
\;\;\;&$2H_3'=$ & $p^2+q^2-\frac{4}{27}( q^3-3 p^2 q)^2$ 
&& $T_3(\alpha^2)=$ &
$2\pi\,\tFoIn{\frac{1}{3},\frac{2}{3}}{1}{\alpha^2}$  & \;\;\; genus 4 \\
\;\;\;&$2H_4'=$ & $p^2 + q^2  - \frac{1}{4}(p^2 + q^2)^2 +2 p^2 q^2 $ 
&& $T_4(\alpha^2)=$ &
$2\pi\,\tFoIn{\frac{1}{4},\frac{3}{4}}{1}{\alpha^2}$  & \;\;\; genus 3 \\
\;\;\;&$2H_6'=$ & $p^2+q^2-\frac{4}{27}q^6$ 
&& $T_6(\alpha^2)=$ &
$2\pi\,\tFoIn{\frac{1}{6},\frac{5}{6}}{1}{\alpha^2}$   & \;\;\; genus 2
\end{tabularx}}
\rule{\textwidth}{.01cm}
Transforming $\alpha \rightarrow \beta=\alpha^2$ reduces 
$\mathcal{A}_s'= 4(s-1)-s^2(1-3\alpha^2)\partial_{\alpha}-s^2\alpha(1-\alpha^2)\partial_{\alpha}^2 
\rightarrow \mathcal{A}_s $. 
\end{center}
\end{table}


The general strategy of concordance is effective though convoluted. We would 
like to have more direct proofs, especially for the seven cases depicted in {Fig. \ref{fig:EllDisks}} and Fig. \ref{fig:NotEllDisks}. These cases divide neatly 
into two classes: simple hyperelliptic or simple dihedral. Table \ref{tab:Hyperelliptic} 
already contains three annihilators with identifiable signature. This is no surprise, 
as we already know that $H_6$, $H_4$, and $H_6'$ have the required hyperelliptic form.
The remaining $H_2$, $H_3$, $H_3'$ and $H_4'$ all have simple dihedral symmetry, 
so their period functions can be proven more directly using 
\texttt{DihedralToODE}. We instantiate an implementation, map 
across a small search space, and in Table \ref{tab:Dihedral} list the four 
hits alongside a few other negatives. Since \texttt{DihedralToODE} is only 
an effective algorithm, we also list certificates, 
\begin{eqnarray}
\Xi_3=\frac{4\sqrt{6}\lambda^{5/2}\sin(3\phi)}{(3\alpha-2\lambda)^3}, 
\;\;\Xi_4'=\frac{\lambda^{3}\sin(4\phi)}{4(\alpha-\lambda)^3}, 
\;\;\Xi_2=\frac{\lambda^{3}\sin(4\phi)}{8(\alpha-\lambda)^3}, 
\;\;\text{and}\;\; 
\Xi_3'=\frac{32\lambda^{4}\sin(6\phi)}{3(3\alpha-4\lambda)^3},
\nonumber
\end{eqnarray}
for $H_2$, $H_4'$, $H_2$ and $H_3'$ respectively. The results of 
\texttt{DihedralToODE} listed in Table \ref{tab:Canonical} can be checked 
against these certificates. By judicious use of the chain rule, we simply by 
calculate a zero value, $\mathcal{A}_s \circ dt \pm \partial_t \Xi_s=0$, where 
choice of either $+$ or $-$ accounts for the possibility of a sign error (
whether accidental or not, sign errors do sometimes happen). The other three 
results of \texttt{HyeperellipticToODE} involve longer 
certificates, here omitted. Those annihilation relations can also 
be checked post computation if necessary (cf. Appendix A).

The last step of the proof involves verifying the genus figures quoted in 
Table \ref{tab:Canonical}, column three. According to well-known genus degree 
bounds, we expect $g=\lfloor (d-1)/2 \rfloor$ for hyperelliptic cases and 
$g=\frac{1}{2}(d-1)(d-2)$ for dihedral cases. The upper bound is only 
met for $H_3$, $H_4$, $H_6$, $H_4'$, and $H_6'$. The other models, $H_2$, 
and $H_3'$, involve hidden symmetries, which require a corrective term 
$\Delta = -\sum_{n}\frac{1}{2}r_n(r_n+1)$. The correction $\Delta$ characterizes 
behavior of singular points at infinity. As early as 1884, Max Noether gave a 
procedure for calculating $\Delta$\cite{POPESCU2016}. The same basic idea of 
counting genus as $g=\frac{1}{2}(d-1)(d-2)+\Delta$ is nowadays built into standard algorithms 
of algebraic geometry\cite{GENUS1988}. In this case, we do not need to reinvent the wheel, 
so simply input $2H_s-\frac{1}{2}=0$ or $2H_s'-\frac{1}{2}=0$ into Singular
software, and use the built-in genus function to output the correct integer\cite{SINGULAR2020}. 
In principle, this I/O approach only tells us the genus of a curve 
$\mathcal{C}_s(\frac{1}{2})$ or $\mathcal{C}_s'(\frac{1}{2})$.
The domain of the oscillation disk is unobstructed by critical points, so it 
is safe to assume\footnote{Perhaps this point could use more rigor eventually, 
but presently we take it as common sense.} that 
$g\big(\mathcal{C}(\frac{1}{2})\big)=g\big(\mathcal{C}(\alpha)\big)$ when 
$\alpha \in (0,1)$. The strong statement of Section II stands true.

For most purposes, the first four rows of Table \ref{tab:Canonical} are a 
good-enough starting place. If not, a next step into the territory of higher 
genus takes into consideration the models listed in the final three rows. 
We have already found more exotic species, and are not quite done with the 
task of cataloging. One glaring omission on our part is that we have yet to mention 
${2H=p^2 + 12 q^2 + 8 q^3 -36 q^4 -48 q^5 -16 q^6}$, a genus 2 hyperelliptic model 
with period $T(\alpha) \propto T_3(\alpha)$. It belongs to an infinite class
of hyperelliptic models, indexed by degree $d=3,4,5...$, each having a potential
 ${V_d(q)=\frac{1}{4}\big(1-\mathcal{T}_d(q)\big)}$, where $\mathcal{T}_d(q)$ stands for the 
$d^{th}$ Chebyshev polynomial of the first kind\footnote{There are many different kinds of 
Chebyshev polynomials, here we mean $\mathcal{T}_n\big(\cos(x)\big)=\cos(nx)$.}. 
Using \texttt{HyperellipticToODE} it is possible to prove that 
$\mathcal{A}_d=\frac{d^2-4}{4d^2}-(1-2\alpha)\partial_{\alpha}
-\alpha(1-\alpha)\partial_{\alpha}^2$, say for the first ten or twenty cases. 
It would be interesting to determine a $d$-dependent form for certificates
$\Xi_d$, but not strictly necessary since one proof of $\mathcal{A}_d$ has 
already been given. This family of Chebyshev potentials is truly amazing in 
terms of hidden symmetry, but it doesn't yield any new cases with $g=1$ 
after $d=4$, nor does it yield any more cases of identifiable signature 
after $d=6$.

\begin{figure}[t]
\begin{overpic}[width=0.9\textwidth]{./Figures/ExtraGeos.eps}
 \put (12.25,-2.5) {$s=4$}
 \put (47.5,-2.5) {$s=6$}
\end{overpic}
\end{figure}
\begin{table}[t]
\begin{center}
\captionof{table}{A few more use cases for \texttt{DihedralToODE}.}  
\label{tab:MoreCases}
\begin{tabularx}{\textwidth}{ c l | c l | c }
\hline \hline
\multicolumn{2}{c|}{Input: $2H=$} & \multicolumn{2}{c|}{Output: $\mathcal{A}=$}  & \;\;\; $s$ \;\;\; \\
\hline
\;\;\;& $\lambda^2-\frac{1}{8}\lambda^4\big(1-\cos(8\phi)\big)$ 
& \;\;\;\; & 
$(4 - 15 \alpha) + 16 \alpha (2 -     3 \alpha)\partial_{\alpha}$ \hspace{1.0cm} &  4   \\
& &   & \hspace{2.0cm}  $ +16 \alpha^2 (1 - \alpha) \partial_{\alpha}^2$  &  \\
\;\;\;& $\lambda^2-\frac{2\sqrt{3}}{9}\lambda^3\cos(6\phi)$  
&&
$(9-32\alpha)+36\alpha(2-3\alpha)\partial_{\alpha}$  &  6 \\
& &   & \hspace{2.0cm}  $ +36 \alpha^2 (1 - \alpha) \partial_{\alpha}^2$  &  \\
\;\;\;& $2\lambda-\frac{32}{27}\lambda^2+\frac{16}{729}\lambda^3\big(9-\cos(6\phi)\big)$
\hspace{0.5cm} 
&&    
$8(18 - 25 \alpha) 
- 9 (27 - 104 \alpha + 75 \alpha^2)\partial_{\alpha}$ \hspace{0.6cm} &  \\
& &   & \hspace{2.0cm}  $ -9 \alpha (1 - \alpha) (27 - 25 \alpha)\partial_{\alpha}^2$  &  \\
\end{tabularx}
\rule{\textwidth}{.01cm}
\end{center}
\end{table}

One last construction is worth mentioning. Both $H_3'$ and $H_4'$
feature a factorizable separatrix that decomposes to a product of hyperbolas. Generalizing 
on this form and performing analysis via \texttt{DihedralToODE}, we find a few more interesting 
cases to take note of, as in Table \ref{tab:MoreCases} and depicted above. The first two
rows list Hamiltonians with identifiable signature. The corresponding curve geometries 
are not oscillation disks in the strict sense because leading term $\lambda^2$ does not 
allow for a harmonic limit at the origin. Yet period functions still exist,
and they can be written as $T(\alpha)=T_s(\alpha)/\sqrt{\alpha}$ for $s=4$ or $6$.
In the third row, the annihilator is not hypergeometric, but it does have the special 
property\footnote{Compare with the "well-integrable" geometries of the Prospectus \cite{KLEE2020Prelude}.} 
that $\partial_{\alpha}(\alpha(1-\alpha)(27-25\alpha))=(27-104\alpha+75\alpha^2)$.  
Remarkably, this operator also shows up in a completely different combinatorial 
search performed by Bostan et al. \cite{DIAGONALS2015}. 

Even more waits to be discovered and explained. It would be nice to develop 
a constructive theory for explaining all higher-genus models in terms of the genus 1 
examples given in Table \ref{tab:Canonical}, but we cannot do so presently. Instead of 
digging deeper into questions of existence and equivalence, next we will return to the 
practical task of using period functions to calculate numerical values. 

\section{Periods and Solutions}
Physicists recognize hyperelliptic Hamiltonians without any qualms because the assumed 
form requires separability, with kinetic energy $\frac{1}{2}p^2$. Indeed, 
the simplest cases, $H_6$ and $H_4$, are textbook examples often found under the heading
of anharmonic oscillation or perturbation theory\cite{LL1960}. As models for data, $H_6$ and $H_4$
work well in situations where a period varies linearly around a harmonic limit. In this
context, local analysis of $H_6$ and $H_4$ generalizes the simple harmonic approximation 
to the \textit{simple anharmonic approximation}. If the data is precise enough, more terms 
can be added to the potential $V(q)$ until the corresponding function $T(\alpha)$ sufficiently 
matches data\footnote{Although, there is one caveat. Period function $T(\alpha)$ does not 
uniquely determine potential $V(q)$ \cite{LL1960}.}. Sometimes it may be possible to measure a period 
function over an entire oscillation disk. These circumstances motivate us to develop an entire 
theory for solving period functions. An important question is: what can we exclude?

Our own prejudices may tempt us to discard any functions not conforming to the seperable 
form $H=\frac{1}{2}p^2 + V(q)$. This would be a crime of oversimplification and an unnecessary 
handicap to analysis in general. Hamiltonian $H_2$ is a perfect example in support of inclusiveness,
because the corresponding oscillation disk exactly models the libration motion of a simple pendulum.
The role of transformation theory can not be ignored. It allows us to change a seemingly 
malformed perturbing term, $p^2 q^2$, into something more familiar, $V(q)=\sin(q)^2$. We have 
no catalogue of all such transforms, but expect more to be discovered soon. The point is that 
even though dihedral models look funny compared to their hyperelliptic counterparts, they may 
eventually facilitate analysis. As nature produces a variety of forms, so should we.

When formulating a space of viable models for anharmonic oscillation, a permissive attitude allows us to 
include any oscillation disk that we can manage to integrate. Algorithms of the previous sections 
allow a great many integral period functions to be calculated as solutions to output ordinary differential 
equations. Instead of evaluating an integral every time we need a value of function $T(\alpha)$, we only 
need to calculate integrals for a small set of initial data, say  
$T_1=T(\alpha_1), \; T_2=T(\alpha_2),\;\ldots,\; T_n=T(\alpha_n)$.
When $n$ equals to the order of $\mathcal{A}$, the system of equations 
should be uniquely solvable for $T(\alpha)$ as a proper function. Our preference
is for piecewise series solutions, but completely numerical solutions also work. 

To show how solutions are carried out in practice, we will now solve $T_s(\alpha)=\partial_{\alpha}S_s(\alpha)$
for each of $s=2,3,4$ and $6$. The four cases are similar enough to proceed with variable  $s$,
except when determining initial data. Recall that action function $S_s(\alpha)$ has a natural geometric 
interpretation as the area enclosed by curve $\mathcal{C}_s(\alpha)$. For this reason, $S_s(\alpha)$ makes 
a better starting place than $T_s(\alpha)$, but first we need to transform the period annihilator
into an action annihilator. Assume that $\mathcal{A}_S \circ S(\alpha)=0$, then 
$\partial_{\alpha} \circ \big(\mathcal{A}_S \circ S(\alpha)\big)=0$ can be rewritten as 
$\mathcal{A}_T \circ T(\alpha)=0$, and the identity $\mathcal{A}_T=\mathcal{A}_s$ determines that 
$\mathcal{A}_S=(s-1)-s^2\alpha(1-\alpha)\partial_{\alpha}^2$. It is again hypergeometric, 
but the missing middle term causes lower parameter $c$ to equal zero. The usual coefficient 
recursion would have $f_1 \propto f_0/0$. This solution does not work, so another must 
be found. In general, $\mathcal{A}_S \circ S_s(\alpha)=0$ is solved 
by\footnote{Recall \cite{KLEE2020Prelude} Sec. IV-V, or see again: \href{http://mathworld.wolfram.com/HypergeometricFunction.html}{Hypergeometric Function}, 
\href{http://mathworld.wolfram.com/Second-OrderOrdinaryDifferentialEquationSecondSolution.html}{Second-Order ODE Second Solution}. },
\begin{eqnarray}
S_s(\alpha) &=& C_1 \alpha \tFo{\frac{1}{s},\frac{s-1}{s}}{2}{\alpha}
- C_0 \alpha \tFo{\frac{1}{s},\frac{s-1}{s}}{2}{\alpha}
\int  \bigg(\alpha \tFo{\frac{1}{s},\frac{s-1}{s}}{2}{\alpha}\bigg)^{-2}d\alpha  \nonumber \\
 &=& C_0\bigg(1 +\Big(\frac{s-1}{2\;s^2}\Big)\alpha\bigg)
+ \bigg(C_1+C_0\Big(\frac{s-1}{2\;s^2}\Big)\log(\alpha) \bigg) \alpha 
\tFo{\frac{1}{s},\frac{s-1}{s}}{2}{\alpha}
+ \sum_{n>1}C_n\;\alpha^n. \nonumber  
\end{eqnarray}
The first line is perfectly valid, but may be doubted. The second line gives an Ansatz,
which is already correct up to a set of undetermined coefficients. The condition,
\begin{eqnarray}
0&=&\mathcal{A}_S \circ S_s(\alpha) = \sum_{n>0}a_n \alpha^n = \Big((s-1)(s^2-s+1)C_0-2s^4 C_2\Big)\alpha  \nonumber \\ 
&&\hspace{2.0cm}
+ \Big( (s-1)^2(5-5s+8s^2)C_0 + 12 s^4 (1 + s) (2s -1)C_2 -72 s^6 C_3\Big)\alpha^2 + \ldots, \nonumber
\end{eqnarray}
sets up a recursion on the coefficients $a_n$. As the summation continues to 
higher powers of $\alpha$, the pattern continues, and the first appearance 
of $C_n$ always occurs in coefficient $a_{n-1}$. The system of equations can 
be solved sequentially to determine $C_n \propto C_0$, but they are just 
the expansion coefficients of the integral on the preceding line. 

The expansion around $\alpha=0$ requires $C_0=0$, for 
$\partial_{\alpha}(\alpha\log(\alpha))=1+\log(\alpha)$, but $T(\alpha)$ is finite 
valued at the origin. The harmonic limit $T(0)=2\pi$ also determines $C_1=\pi$, 
so we need only one initial condition rather than two. With the solution 
$S_s(\alpha)=\pi \alpha \tFoIn{\frac{1}{s},\frac{s-1}{s}}{2}{\alpha}$,
very slow convergence becomes a problem as $\alpha$ approaches 1. 
Fortunately, symmetry allows for a second expansion, which converges rapidly where 
the first diverges and \textit{vice versa}.
Annihilator $\mathcal{A}_S$ transforms invariantly by 
$\alpha \rightarrow \rev{\alpha}=1-\alpha$, so we can write a reversed 
expansion $\rev{S_s(\alpha)}$ around $\alpha=1$ by simply taking $S_s(\alpha)$
and reversing $C_n \rightarrow \rev{C}_n$ and $\alpha \rightarrow \rev{\alpha}$.
The identity $\mathcal{A}_S \circ \rev{S_s(\alpha)} = 0$ holds through reversal 
of $\alpha$, thus the solution is correct up to determination of $\rev{C}_0$ 
and $\rev{C}_1$. These initial data depend on the behavior near $\alpha=1$, which in 
turn depends on the choice of $s$. The zeroth coefficient is just the area enclosed by the separatrix, 
$ \rev{C}_0 = \text{area}\big(\mathcal{C}_s(1)\big)$. Determination of 
$\rev{C}_0$ requires case by case integration. Curves $\mathcal{C}_2(1)$ 
and $\mathcal{C}_3(1)$ bound equilateral polygons, so the corresponding 
interior areas can be calculated proportional to squared side lengths.
Separatrices  $\mathcal{C}_4(1)$ and $\mathcal{C}_3(1)$ are not polygon 
boundaries, but the area can be calculated by $2\int p\; dq$ across
the maximum width of the oscillation disk. When $\rev{C}_0$ is known exactly,
then $\rev{C}_1$ can be determined to arbitrary precision by solving 
$S_s(\alpha)=\rev{S_s(\alpha)}$ at the midpoint $\alpha=\rev{\alpha}=\frac{1}{2}$.
Table \ref{tab:InitDat} collects the results of our numerical calculations. 


\begin{table}[t]
\begin{center}
\captionof{table}{Initial data of reversed action functions.}  
\label{tab:InitDat}
\begin{tabularx}{0.9 \textwidth}{ c |  c l | c l  c l }
\hline \hline
 \;\;\;$s$\;\;\;&\multicolumn{2}{l|}{ \; $\rev{C}_0=$} & \multicolumn{2}{l}{\; $\rev{C}_1=$} 
 &  \\
\hline
2 &\;\;\;\;& $4$ & \;\;\;\; 
         & $-4.2725887222397812\ldots$ &  & $ = -\frac{3}{2} - 4 \log(2)$   \\
3 &\;\;\;& $9\frac{\sqrt{3}}{4}$ \hspace{0.5cm} & \;\;\; 
         & $-4.1533165583656958\ldots$ &  & $ = -\frac{3}{4}\sqrt{3}\big(1+2\log(3)\big)$   \\
4 &\;\;\;& $8\frac{\sqrt{2}}{3}$ & \;\;\; 
         & $-4.0014346021854628\ldots$ &  & $ = -\frac{3}{4}\sqrt{2}\big(1+4\log(2)\big)$   \\
6 &\;\;\;& $\frac{18}{5}$ & \;\;\; 
         & $-3.7842127941220551\ldots$ & \;\;\;\; & $ = - \frac{1}{4}\big(3+8\log(2)+6\log(3)\big)$
\\ \hline
\end{tabularx}
\end{center}
\end{table}

The two solutions are equivalent, so they can be stitched together at $\alpha=\rev{\alpha}=\frac{1}{2}$ 
to form a piecewise solution $\underset{^{pw}}{S_s}(\alpha)$, which equals $S_s(\alpha)$ 
when $0 \le \alpha \le \frac{1}{2}$, otherwise it equals $\rev{S_s(\alpha)}$ when 
$\frac{1}{2}<\alpha \le 1$. As a \textit{computable function}, $\underset{^{pw}}{S_s}(\alpha)$ 
implicitly relies on a truncation parameter $N$, which controls how many terms are summed in 
both expansions. Choosing $N=60$ is good enough to guarantee double precision, according to the error bound
$\mathcal{A}_s \circ \underset{^{pw}}{S_s}(\alpha) < 10^{-16}$ when $\alpha \in [0,1]$.
Precision can always be increased by retaining more summation terms, but don't forget,  
a reconstruction of $\underset{^{pw}}{S_s}(\alpha)$ with more terms also requires 
recalculation of the coefficient $\rev{C}_1$. This is only a slight annoyance, one we 
must live with for now. We still don't know of a reasonable method to obtain exact solutions 
for the four values $\rev{C}_1$, but strongly suspect the guesses\footnote{The computer program RIES
accepts numerical values as input and returns probable closed forms as output. It is 
free software and available online at  \href{https://mrob.com/pub/ries/}{https://mrob.com/pub/ries/}.}
in Table \ref{tab:InitDat} are correct. As error reaches a maximum at the boundary, a less 
conservative error estimate is obtained by subtracting the numerical value from the (assumed 
correct) exact value. When this is done for the $N=60$ approximation, we can estimate 
the error of $\underset{^{pw}}{S_s}(\alpha)$ less than $10^{-20}$ on the entire 
oscillation disk.  Having defined a set of computable functions with error bounds, we 
can now begin to calculate. 


\begin{figure}[t]
\begin{overpic}[width=0.95\textwidth]{./Figures/SFuns.eps}
 \put (-0.2,-2.5) {$0$}
 \put (27.0,-2.5) {$0$}
 \put (54.1,-2.5) {$0$}
 \put (81.1,-2.5) {$0$}


 \put (17.5,-2.5) {$1$}
 \put (44.8,-2.5) {$1$}
 \put (72,-2.5) {$1$}
 \put (99,-2.5) {$1$}

 \put (0.8,29)  {$S_2$}
 \put (28.0,29) {$S_3$}
 \put (55.1,29) {$S_4$}
 \put (82.2,29) {$S_6$}

 \put (-2.0,3.1)   {$\frac{1}{2}$}
 \put (-2.0,10.7)  {$\frac{3}{2}$}
 \put (-2.0,18.2)  {$\frac{5}{2}$}
 \put (-2.0,25.9)  {$\frac{7}{2}$}


 \put (17.7,1.3) {$\alpha$}
 \put (45,1.3) {$\alpha$}
 \put (72.2,1.3)   {$\alpha$}
 \put (99.2,1.3)   {$\alpha$}

\end{overpic}

\phantom{\;}

\caption{Root solving computable functions, 
$\underset{^{pw}}{S_s}(\alpha_n)
=\frac{1}{4}(n+\frac{1}{2})\underset{^{pw}}{S_s}(1)$ 
with $n=0,1,2,$ and $3$.}  
\label{fig:SFuns}
\end{figure}

\begin{table}[t]
\begin{center}
\captionof{table}{Semiclassical quantum values for $h=S_s(1)$ and $N=4$, as in Fig. \ref{fig:SFuns}.}  
\label{tab:QVals}
\begin{tabularx}{0.95 \textwidth}{ c |  c l | c l | c l | c l }
\hline \hline
 \;\;\;$s$\;\;\;
 &  \multicolumn{2}{l|}{ \; $\alpha_0=$} 
 &  \multicolumn{2}{l|}{ \; $\alpha_1=$} 
 &  \multicolumn{2}{l|}{ \; $\alpha_2=$} 
 &  \multicolumn{2}{l}{ \; $\alpha_3=$} 
    \\
\hline
2 &\;\;\;& $0.15592230\ldots$ \;\;\; & \;\;\;  & $0.44694844\ldots$ \;\;\; &  
   \;\;\;& $0.70571106\ldots$ \;\;\; &  \;\;\; & $0.92129983\ldots$ \;\;\; \\  
3 &\;\;\;& $0.15232502\ldots$ & \;\;\; & $0.43916833\ldots$ &  
   \;\;\; & $0.69785371\ldots$ &  \;\;\; & $0.91761719\ldots$ \\  
4 &\;\;\;& $0.14788266\ldots$ & \;\;\; & $0.42934933\ldots$ &  
   \;\;\; & $0.68764827\ldots$ &  \;\;\; & $0.91260116\ldots$ \\  
6 &\;\;\;&  $0.14176761\ldots$ & \;\;\; & $0.41544690\ldots$ &  
   \;\;\; & $0.67263195\ldots$ &  \;\;\; & $0.90470726\ldots$ \\
\hline
\end{tabularx}
\end{center}
\end{table}



Plots of functions $S_s(\alpha)$, as in Fig. \ref{fig:SFuns}, allow us to see 
with our eyes how similar the alternatives are. All four have the same 
small-amplitude limit ${S_s(\alpha) \approx \pi \alpha}$, but begin to diverge
from one another as $\alpha$ approaches $1$. Maximum difference occurs at $\alpha=1$.
An easy bound, $\frac{2(4-18/5)}{(4+18/5)}=\frac{2}{19}$, says that, at most, 
the value of one action function differs from the value of another by 
about $10\%$. In addition to the graphs, we also want to calculate a few 
special "quantum values"\footnote{It is left as an exercise:
find matrices with comparable eigenvalues, see also \cite{KLEE2020Prelude} Sec. VI.}. 
In principle, the action functions could be inverted to find energy  
${\alpha = S_s^{-1}(S)}$ as a function of action $S$.  However, when only 
a few values $\alpha_n$ are needed, it is much more efficient to simply root 
solve $\underset{^{pw}}{S_s}(\alpha_n)-h_n=0$, say by Newton's method. Calculating 
the so-called quantum values requires special choice of a density $N/h$, which 
in turn determines $h_n=\frac{h}{N}(n+\frac{1}{2})$ for $n=0,1,\ldots,N-1$. In Figure \ref{fig:SFuns} we set the action scale as $h=\rev{C}_0$, choose $N=4$. The values 
of Table \ref{tab:QVals} then determine the blue curves of Fig. \ref{fig:EllDisks}. 
In those graphs, the area contained by any central $n=0$ blue curve is 
$\frac{1}{8}\rev{C}_0$, as is the area between any $n=3$ curve and the separatrix. 
The area between any two consecutive blue curves equals to $\frac{1}{4}\rev{C}_0$.
Thus the total area enclosed by the separatrix is 
$\big(\frac{1}{8}+\frac{1}{8}+3\frac{1}{4}\big)\rev{C}_0=\rev{C}_0$ as necessary. 
The quantum theory will contribute more significantly to followup articles. Before
then, we have a few other classical calculations to discuss.

\begin{figure}[t]
\begin{overpic}[width=0.9\textwidth]{./Figures/ToricSections.eps}
 \put (75,25) {$s=2,\;\;\;3,$}
 \put (83.5,22) {$4,\;\;\;6.$} 
\end{overpic}
\caption{Toric cross sections with real/complex contours in blue/green. }
\label{fig:ToricSections}
\end{figure}


Computable instances of the alternative period functions, allow us to calculate
alternative nomes, the $q$'s mentioned in the introduction. Yet the $q$'s seem to 
stand for \textit{questions unanswered}, and 
we are left to wonder. What should we make of them? How are they used in the theory of 
elliptic curves? They are not the $q$'s of Hamiltonian mechanics, rather, they are \textit{nomes}, 
which conform to the general expression $q=e^{i \tau}$, with 
$\tau = T_{\mathfrak{I}}/T_{\mathfrak{R}}$,
the ratio between real and complex periods. Thus far, we have only dealt with 
real periods, but for the set of basic $H_s$ the real periods are the complex periods,
up to a change of Harmonic frequency and inversion of the energy scale $\alpha \rightarrow 1-\alpha$. 
The annihilator $\mathcal{A}_s$ transforms invariantly by $\alpha \rightarrow 1- \alpha$, so we 
can write $T_{\mathfrak{I}}=i(\omega_s/\widetilde{\omega}_s)T_s(1-\alpha)$. To obtain the frequency ratios, 
simply apply an Abel-Wick rotation\footnote{Refer back to \cite{KLEE2020Pendulum} Sec. IV. A rotational axis must be chosen to 
intersect a hyperbolic point on the separatrix.} from $H_s \rightarrow \widetilde{H}_s$,
as in Fig. \ref{fig:ToricSections}, and find the harmonic frequency around a circular point of $\widetilde{H}_s$. 
For signatures $s=2,3,4,6$, we calculate the corresponding complex-time frequencies $\widetilde{\omega}_s$ as 
$\widetilde{\omega}_2 = 2$, $\widetilde{\omega}_3 = \sqrt{3}$, $\widetilde{\omega}_4 = \sqrt{2}$, 
and $\widetilde{\omega}_6=1$. By convention, $\omega_s = 1$, then follows 
$\omega_s/\widetilde{\omega}_s = \frac{1}{2}\csc(\pi/s)$, as quoted in the introduction. 
In the case $s=2$, we have already shown how the nome $q$ contributes to an exact solution
using Harold Edwards's alternative theory \cite{EDWARDS2007}. More work needs to be done 
using other $q$'s to solve the time parameterization problem on other $H_s$, especially case $s=3$.

\pagebreak 
\section{A Few Binomial Series for $\pi$}
A very special feature of $S_s(\alpha)$ and $T_s(\alpha)$ is that they satisfy 
a Legendre-style identity, 
$S_s(\alpha)T_s(\rev{\alpha})+S_s(\rev{\alpha})T_s(\alpha)=2\pi\;\rev{C}_0$. 
It follows from $T_s(\alpha)=2\partial_{\alpha}S_s(\alpha)$ and 
$\mathcal{A}_S \circ S_s(\alpha)=0$, after taking either limit 
$\alpha \rightarrow 0$ or $\alpha \rightarrow 1$. First observe,
\begin{eqnarray}
 \partial_{\alpha}\big(S_s(\alpha)T_s(\rev{\alpha})+S_s(\rev{\alpha})T_s(\alpha)\big)=
 -S_s(\alpha) S_s''(\rev{\alpha})+S_s(\rev{\alpha})S_s''(\alpha) =  0, \nonumber
\end{eqnarray}
because solving either $\mathcal{A}_S \circ S_s(\alpha) = 0$ or 
$\mathcal{A}_S \circ S_s(\rev{\alpha})=0$ 
produces $S_s''(z)=\frac{s-1}{s^2\alpha(1-\alpha)}S_s(z)$ with 
either $z=\alpha$ or $z=\rev{\alpha}$. Next take the limit, 
\begin{eqnarray}
\lim_{\alpha \rightarrow 0} S_s(\alpha)T_s(\rev{\alpha})+S_s(\rev{\alpha})T_s(\alpha)
= \text{constant} \times \lim_{\alpha \rightarrow 0} \big( \alpha  
\times  \log(1-\alpha)  \big)
+ \rev{C}_0 \times 2 \pi = 2 \pi \; \rev{C}_0,  \nonumber 
\end{eqnarray}
and the identity is already proven, no problem! Choosing $\alpha=\rev{\alpha}=\frac{1}{2}$ 
and dividing both sides by $2\pi^2$, we obtain a few summations for $\rev{C}_0/\pi$,
\begin{eqnarray}
\frac{4}{\pi} &=& \sum_{n=0}^{\infty}\sum_{k=0}^{n}
\frac{1}{32^n}\frac{1}{(k+1)}\binom{2(n-k)}{n-k}^2\binom{2k}{k}^2, \nonumber \\
\frac{9\sqrt{3}}{4\pi} &=&  \sum_{n=0}^{\infty}\sum_{k=0}^{n}
\frac{1}{54^n}\frac{1}{(k+1)}\binom{3(n-k)}{n-k}\binom{2(n-k)}{n-k} \binom{3k}{k} \binom{2k}{k}, 
   \nonumber  \\
\frac{8\sqrt{2}}{3\pi} &=&  \sum_{n=0}^{\infty}\sum_{k=0}^{n}
\frac{1}{128^n}\frac{1}{(k+1)}\binom{2(n-k)}{n-k}\binom{4(n-k)}{2(n-k)} \binom{2k}{k} \binom{4k}{2k}, 
   \nonumber  \\
\frac{18}{5\pi} &=&  \sum_{n=0}^{\infty}\sum_{k=0}^{n}
\frac{1}{864^n}\frac{1}{(k+1)}\binom{3(n-k)}{n-k}\binom{6(n-k)}{3(n-k)} \binom{3k}{k} \binom{6k}{3k}. 
   \nonumber     
\end{eqnarray}
After dropping denominators, the corresponding integer sequences, 
\begin{eqnarray}
s=2 &:& 1, 6, 56, 620, 7512, 96208, 1279168, 17471448,243509720, 3447792656, \ldots  \nonumber  \\
s=3 &:& 1, 9, 138, 2550, 51840, 1116612, 24999408, 575368596,13518747000, \ldots  \nonumber  \\
s=4 &:& 1, 18, 632, 27300, 1306200, 66413424, 3515236032, 191434588488, \ldots  \nonumber  \\
s=6 &:& 1, 90, 20280, 5798100, 1854085464, 632693421360, 225235329359040, \ldots  \nonumber  
\end{eqnarray}
are not found in OEIS as of June 16, 2020. These are not exactly what Ramanujan had in 
mind for Section 14 of \cite{RAMANUJAN1914}, but they appear similar. Perhaps they are too straightforward 
or not rapidly convergent enough, but at least they are easy to derive. We would like to carry 
this idea farther, but simply do not have time or space. Interested readers are refered to 
Jonathan and Peter Borwein's "Pi and the AGM"\cite{BORWEIN1987}.


\section{Conclusion}
Famously, G.N. Watson (1886-1965) compared Ramanujan with the Italian renaissance
by saying, 
\begin{quote}
\textit{Ramanujan's formula gave me a thrill which is indistinguishable from the thrill
which I feel when I enter the Sagrestia Nuova of the Capille Midicee and see 
before me the austere beauty of the four statues representing Day, Night, 
Evening, and Dawn, which Michelangelo has set over the tombs of the Medicis.} 
\end{quote}
We don't want to take away Watson's imminent and lordly right to a personal opinion. 
This sort of high praise is appropriate, but is it another example of 
Eurocentric bias from a leader of an Anglocentric society? Even to this day, 
the question of how we should view Ramanujan and his work remains persistently 
difficult. An antithesis is to compare with the temple architecture at Namakkal. This 
could be more appropriate considering the profound importance of place in Ramanujan's 
life. Such a comparison misses the point of Watson's appraisal, which is to 
say that Ramanujan became a solid contributor to Western culture.

There is a middle way and a synthesis of perspectives, perhaps not known only 
to the author, where Ramanujan's works are described as a type of music that 
came into existence before it could even be properly heard. Although Pandit 
Ravi Shankar and Ustad Zakir Hussain do not hail from Tamil Nadu, they are 
also natural Indian citizens who used their work to transcend national 
boundaries. In Watson's day the idea of a world harmony was 
only a promise, one made amidst the machinations of World War. Later it 
did come to pass that Ravi Shankar and Zakir Hussain helped to invent the 
audible genre of world music. The rest is on records, tapes, CDs, 
DVDs and probably a few youtube videos\footnote{Listen 
also: \textit{West Meets East Volumes I \& II}, 
Ravi Shankar and Yehudi Menuhin; \textit{Passages}, Ravi Shankar and 
Phillip Glass; \textit{Remember Shakti Live at Jazz a Vienne}, Zakir Hussain, 
John McLaughlin et al. And here's one more, 
\textit{Tala Matrix}, Zakir Hussain, Bill Laswell, et al.}.

The musicality of Ramanujan's mathematics bears some relation to the prosody of earlier Indian
scientists. First the Pingala wrote sayings for binomial 
coefficients, then Madhava for trigonometric functions, and after Ramanujan, now there is even a saying 
for elliptic integrals. Is Ramanujan's famous assertion of alternatives $K_1$, $K_2$, and $K_3$ 
a product of the East or of the West? And whose hearing is it meant to reach? The chosen language
is not Sanskrit, nor Latin. Nor are alternatives $K_1$, $K_2$, and $K_3$ otherwise hidden from 
lower classes or less privileged castes. Mahatma Ghandi and his followers won their engagements worldwide. 
The future is not the exclusive property of any trading company, and the Brahmin's exclusive right 
to practice Hindu science has also been released. It has never been easier for anyone to find out about 
Ramanujan's theory, or any theory thereafter! 

How can we all come to appreciate Ramanujan? We don't know, but continue 
to spend our efforts unfolding what we can from ideas he left behind. The ideas are difficult,
but as we have seen, computing power makes them more readily accessible. Already,
billions and trillions of digits of $\pi$ have been computed using his insight. 
Too bad we do not have time to develop more about $1/\pi$, but it is not the only
place where Ramanujan's presence can be felt. The preceding analysis of geometric 
models means to open up a whole new plot in the Mysetrious garden. What else 
grows in this pleasant and peaceful space? Are there perhaps beetles and 
butterflies practicing symbiosis with the flora? Poetics aside, at least we have a program for 
extending systematic analysis\footnote{Again refer back to the Prospectus of \cite{KLEE2020Prelude}.}.  

Science is a process of developing bias, and in the end, science will defeat 
itself unless the practice of \textit{branching out} is maintained. One weakness
of the present article is that we have focused mostly on hypergeometric cases 
in order to suggest that only four achieve a minimal form. This point is at once 
worthwhile and irrelevant. We should, and we will, spend more time analyzing geometries 
such as case 3 of Table \ref{tab:MoreCases}. It \textit{is not} a hypergeometric
case, so the Hadamard factor analysis does not even apply! Yet the periods 
of this geometry, and of other similar geometries, are still differentiably-finite.
That is, the period and action functions are the solutions of ordinary differentiable
equations. The analysis is not much more difficult, and we can make more progress
easily.

In Section V, Table \ref{tab:QVals} is really more profound than it seems. The 
semiclassical eigenvalues do more than enable harmonic proportions in drawn 
figures. In physics, they provide a bridge between the classical theory and 
the quantum theory. Similar values can be calculated by matrix methods, but 
tunneling between states also needs to be accounted for. This is one focus
of a planned followup article, where we will also include a brief explanation
as to how differential equations can be used for classical data analysis.  
This effort will build upon algorithms defined in this article, so we have 
already completed much of the work necessary. Physics and math, practice 
non-duality and the way is clear.

\pagebreak

\section*{Acknowledgements}
Neil Sloane and Joerg Arndt were persistent opponents whose antagonism 
helped to improve rigor. Both of them work \textit{pro bono publico}, 
and correspond through [seqfan] and [mathfun] mailing lists. Bill Gosper 
knew the phrase "singular modulus", leading to reference \cite{VILLARINO2020}.
Peter Paule provided the author access to the RISC Mathematica implementation 
of Zeilberger's algorithm for hypergeometric summation, which is packaged 
with Koutschan's Holonomic Functions. In applicable cases, these tools 
have been used to double check results obtained by \texttt{EasyCT} 
implementations. Other calculations and explanations herein are the 
author's own work (with the aid of Mathematica and Singular). 

\bibliographystyle{unsrtnat}
\bibliography{biblio}

\pagebreak 

\begin{appendices}
\section{Supporting Documents}
The three implementations for \texttt{EasyCT} discussed above were given ahead of schedule 
as \href{https://github.com/bradklee/Dissertation/tree/master/Prelude/notebooks}{supplementary material} 
for the dissertation prelude, and in fact, all results of Table \ref{tab:Canonical} have been 
proven correct since the release of the Wolfram Demonstration \href{https://demonstrations.wolfram.com/AFewMoreGeometriesAfterRamanujan/}{A few more geometries after Ramanujan}. Similarly, analysis of right 
factors has been possible since the earlier release of, 
\href{https://demonstrations.wolfram.com/ApproximatingPiWithTrigonometricPolynomialIntegrals/}{Approximating 
Pi with Trigonometric-Polynomial Integrals}. For Section II, a bit of extra programming was done to 
create a large search space and analyze results found. Another notebook containing this search 
is available via github. The search is fast enough to rerun in under an hour(?). We have also created
an extensive set of I/O records, each of which is trivial to validate using the CheckCert notebook. For 
section IV, more searching was done using \texttt{HyperellipticToODE} and \texttt{DihedralToODE}.
Those searches are also given (with certificate checking turned on) in another supplementary notebook at
github.  

As is typically the case, this article was slightly delayed by another branching-out opportunity that we 
decided to take seriously. An argument on the [math-fun] mailing list eventually led us all to consider
how simple knots such as the trefoil might be represented as algebraic varieties. Following a suggestion of 
Cris Moore, we quickly found a seemingly minimal solution for the trefoil, and also managed to integrate 
its natural period function, essentially using an \texttt{EasyCT} algorithm. The certified result, along 
with a few plots, was then published on Wolfram Demonstrations under the title,
 \href{https://demonstrations.wolfram.com/AlgebraicFamilyOfTrefoilCurves/}{Algebraic Family of Trefoil Curves}.
It is relatively easy to generalize the calculation to other similar torus knots, so there is certainly 
room for growth and more comparative analysis. After all, why should we spend all of our thoughts and efforts on 
integrating unknots?   
\end{appendices}

\end{document}

