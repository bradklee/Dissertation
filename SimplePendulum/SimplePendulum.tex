\documentclass[nofootinbib,preprint]{revtex4-1} 
\usepackage[utf8]{inputenc}
\usepackage[fleqn]{amsmath}
\usepackage[title]{appendix}
 \usepackage[none]{hyphenat}
\usepackage{amssymb}
\usepackage{amsmath}
\usepackage[mathscr]{euscript}
\usepackage{skak}
\usepackage{capt-of}
\usepackage{afterpage}
\usepackage{placeins}
\usepackage{natbib}
%\usepackage[noend]{algpseudocode}
\usepackage{algorithm2e}
\usepackage{url}
\usepackage{hyperref}
\hypersetup{colorlinks=true}
\usepackage[margin=1in]{geometry}
\usepackage{graphicx,wrapfig} 
\usepackage[percent]{overpic}
\usepackage{tabularx}
\usepackage{makecell}



%\usepackage{footmisc}
%\DefineFNsymbols{mySymbols}{{\ensuremath\dagger}{\ensuremath\ddagger}\S\P
%   *{**}{\ensuremath{\dagger\dagger}}{\ensuremath{\ddagger\ddagger}}}
%\setfnsymbol{mySymbols}

\renewcommand{\arraystretch}{1.3}

\newcommand{\pFq}[3]{\mathcal{F}\bigg[ \begin{array}{c}
#1\\ #2 \end{array} \bigg| #3 \bigg]} 
\newcommand{\tFo}[3]{\,_2\mathcal{F}_1\bigg[ \begin{array}{c}
#1\\ #2 \end{array} \bigg| #3 \bigg]} 

\begin{document}
\title{An Alternative Theory of Simple Pendulum Libration}
\author{Bradley Klee}
\email{bjklee@email.uark.edu, bradklee@gmail.com} % optional
\affiliation{Department of Physics, University of Arkansas, Fayetteville, AR 72701}

\date{\today}

\begin{abstract}
Integration along the transcendental Hamiltonian function $\alpha=2H_{\vartheta}(p,q)=p^2+\sin(q)^2$ 
predicts the time dependence of simple pendulum libration when $\alpha \in (0,1)$. By 
equivalence via canonical transformation, the same can be said for the 
algebraic Hamiltonian function,
$$\alpha=2H_{\varphi}(p,q)=\Big(p^2+q^2\Big)\Big(1-\frac{1}{4}q^2\Big),$$ 
a close relative to Harold Edwards's normal form for elliptic curves. Combining real and 
complex transformation theory with Edwards's theory of elliptic curves and elliptic 
functions, we derive an exact solution of the simple pendulum's librational equations of 
motion.
\end{abstract}

\maketitle 

\section{History and Introduction}

\begin{wrapfigure}{R}{7cm}
\begin{center}
\begin{overpic}[width=0.35\textwidth]{./Figures/TComp.eps}
 \put (4,90) {\Large$T$}
 \put (58,4) {\Large$\alpha$}
 \put (-5,32) {\Large$1$}
 \put (-5,65) {\Large$2$}
 \put (-1,-6) {\Large$0$}
 \put (32,-6) {\Large$1$}
\end{overpic}
\caption{Two period functions.}
  \label{fig:TwoPeriods}
  \phantom{space}
\end{center}
\end{wrapfigure}

The universal law of gravitation\footnote{For an introduction, try \textit{Feynman 
Lectures on Physics}, Vol. I, Ch. 7 \cite{FEYNMAN1963}.} guarantees an analogy between the 
libration of a pendulum around its fixed axis and the orbital motion of 
a planet around its central star. The entire analogy goes deeper than to simply
describe a planet and a pendulum bob as massive objects affected by 
gravitational forces. In both cases, the equations of motion have oscillating
solutions, which return to their initial conditions after a fixed interval of 
time. In both cases, the recurrence periods depend on an amplitude parameter
that enters through the initial conditions. Here the analogy reveals its great 
fallacy. The two period functions do not share very many similarities, 
nor are they equally difficult to solve or to measure. The discovery and 
resolution of this dichotomy plays across many interesting chapters from the 
history of science, and leads into the subject matter of this current work.

About a century before the advent of 
integral calculus, Galileo (1564-1642) made the simple observation that two 
pendulums of the same characteristic length will undergo isochronous 
oscillations. The observation does not hold true outside the small angle limit, 
as Galileo may very well have known\footnote{As Bill Gosper speculates
in an email from Sunday July 7, 2019, 05:33:49 UTC.}. A counterpoint to this 
story occurs in the researches of Kepler (1571-1630), who first discovered the three 
laws of planetary motion. The third law asserts that the square of a planet's yearly 
period is proportional to the cube of the orbital ellipse's semi-major axis length. 
In a sense, Kepler succeeded where Galileo could not. He was able to correctly 
ascertain the functional dependence of an oscillation period. Meanwhile, significant 
obstacles in theory and experiment stood between renaissance scientists and a better 
understanding of pendulum dynamics.  

Though the planets can not be controlled in a laboratory as can a pendulum bob, the 
planetary period function is sometimes easier to measure than that of a simple 
pendulum\footnote{For a popular discussion of Galileo, Kepler, and early measurements, 
try \textit{Infinite Powers} \cite{STROGATZ2019}.}. 
Using only XVI century implements, an astronomer can chart the day-to-day motion of a 
planet at a resolution of tens or hundreds of data points per yearly cycle. 
Without digital acquisition of data, a similar sampling of pendulum motion at $10$ to 
$100$ Hertz is practically impossible. Dissipative losses introduce uncertainty to 
measured values and compound upon difficulties due to short timescale. Even if renaissance 
experimental scientists could have accurately measured changes of the pendulum's period on 
the order of a few percents, the analytic task of extracting a functional form would have been 
an impossible prospect before the theoretical advances of Euler (1707-1783). The period 
function in question is not algebraic, nor is it immediately easy to characterize by fitting 
only a few free parameters. 

Among hundreds of other topics, Euler initiated the general study of elliptic integrals and 
hypergeometric series, and subsequently solved exactly for the simple pendulum's amplitude 
dependence\footnote{Cf. Euler Archive \cite{EULER2019}:  
\href{http://eulerarchive.maa.org/pages/E028.html}{E028}, 
\href{http://eulerarchive.maa.org/pages/E366.html}{E366},
\href{http://eulerarchive.maa.org/pages/E503.html}{E503}.}. Careful 
examination of publication records reveals yet another example of Arnold's principle that 
"discoveries are rarely attributed to the correct person". Euler could already understand 
the hypergeometric function $\,_2 F_1(a,b;c;z)$ in terms of its characteristic differential 
equation, 
\begin{eqnarray}
a \;b\; F - \big(c-(a+b+1)z\big) \partial_{z}F - z(1-z)\partial_{z}^2 F = 0,  \nonumber
\end{eqnarray}
yet most authors credit the later Gauss (1777-1855), who gave a more extensive treatment 
and standardized notation, decades after Euler's exploratory work \cite{DUTKA84}. In any case, 
the hypergeometric function eventually came to be seen as a crown accomplishment of classical 
function theory and a \textit{sine qua non} of period analysis \cite{KZ2001}. 

Up to a scale degree of freedom or an initial value, the two ordinary 
differential equations,
\begin{eqnarray}
3\;T-2\;\alpha \;\partial_{\alpha}T=0 \;\;\;\;\;\;\;\; \text{vs.} \nonumber
 \;\;\;\;\;\;\;\; T-\partial_{\alpha}(4\;\alpha\;(1-\alpha)\;\partial_{\alpha}T)=0,
\end{eqnarray}
define the periods of planetary and pendulum motion respectively, as seen in Figure ~\ref{fig:TwoPeriods}.  
Integers $2$ and $3$ appearing in the first equation are the exponents already known to Kepler. 
Integer $4$ appearing in the second equation is not so easy to explain. Choosing hypergeometric parameters 
$(a,b,c)=(1/s,(s-1)/s,1)$, the general equation reduces to a factored form,
\begin{eqnarray}
(s-1)F-\partial_{z}\big(s^{2}\;z\;(1-z)\;\partial_{z}F\big)=0. \nonumber
\end{eqnarray}
Usually the line $a+b=1$ intersects hyperbola $a\;b=(s-1)/s^2$ at two distinct points, 
while preserving reflection symmetry across the line $a=b$. Choosing $s=2$, line $a+b=1$ 
lays tangent to hyperbola $a\;b=1/4$ at the symmetry point $a=b=1/2$, as in Fig.
~\ref{fig:abHypLin}. This choice also recovers the pendulum's characteristic 
differential equation, thus the appearance of integer $4$ acquires a special meaning 
through the hypergeometric theory. The decomposition of $4$ to $(a,b,c)=(1/2,1/2,1)$ 
produces all the necessary parameters to construct a series solution of the simple 
pendulum's period (Cf. Ref. \cite{GRAHAM1994}, Ch. 5).



\begin{wrapfigure}{R}{7cm}
\begin{center}
\begin{overpic}[width=0.35\textwidth]{./Figures/abHypLin.eps}
 \put (4,97) {\Large$a$}
 \put (97,4) {\Large$b$}
 \put (-5,44) {$\frac{1}{2}$}
 \put (-5,88) {$1$}
 \put (44,-7) {$\frac{1}{2}$}
 \put (88,-7) {$1$}
 \put (36,88) {\underline{\;\;\;\;\Large$s=$\;\;\;\;\;\;\;\;\;\;\;\;\;\;}}
 \put (45,79) {$2, 3, 4, 6,$}
 \put (58,70) {$9, 14, 22, 45.$}
\end{overpic}
\caption{Mapping between parameters. }
  \label{fig:abHypLin}
  \phantom{space}
\end{center}
\end{wrapfigure}

Euler and Gauss developed a fine characterization of the gross dynamics of a simple pendulum, 
but left exact solution of the time parameterization problem mostly an open question 
for the next few generations of researchers. The answering involved an honor roll 
of European patriarchs\textemdash Legendre, Riemann, Jacobi, Weierstrass and many others. Especially 
through the efforts of Abel (1802-1829), a central theme emerged that the time parameter 
can also take on complex values\footnote{For another closely related historical account,
try \textit{What is the Genus?} \cite{POPESCU2016}.}. Subsequently during XX 
century the pendulum's exact solution in terms of standard, doubly-periodic 
elliptic functions became a lauded final product of the classical era \cite{WW1902,WHITTAKER1904}.

\begin{figure*}[ht] 
\begin{center}
\begin{overpic}[width=.9\textwidth]{./Figures/Sommerfeld.jpg}
\end{overpic}
\caption{Sommerfeld's phase plane geometries. (Source: Public Domain via archive.org)}
\label{fig:Sommerfeld}
\end{center}
\end{figure*}

\begin{wrapfigure}{R}{7cm}
\begin{center}
\begin{overpic}[width=0.35\textwidth]{./Figures/PendulumPortrait.eps}
 \put (39,98) {\Large$p$}
 \put (73,45) {\Large$q$}
\end{overpic}
\caption{Pendulum Phase Portrait.}
  \label{fig:PendPortrait}
\end{center}
\end{wrapfigure}  

The XX century also saw the advent of Quantum Mechanics. During this time period, precise spectroscopic 
measurement of atoms and molecules confounded preexisting theories; however, physicists did ultimately 
discover that classical oscillations have 
somewhat odd counterparts in the quantum regime. Initial attempts to describe quantum oscillations 
built upon the concept of phase space. The mathematical prehistory of phase space goes back to Poincar\'{e} 
(1854-1912), but the work of Ehrenfest (1880-1933) marks the first occurrence of the the Deutsch 
word \textit{Phasenraum} \cite{NOLTE2010}. Thereafter Sommerfeld (1868-1951) included early depictions 
of two equivalent phase plane geometries in his famous text \textit{Atombau Und Spektrallinien} \cite{SOMMERFELD1921}. 
To the uninitiate, the two drawings\, reproduced here in Fig. \ref{fig:Sommerfeld} 
are only curious works of abstract art. It is not immediately apparent that they describe harmonic 
oscillation, nor that they relate by a special coordinate transformation, nor that they can be deformed 
to account for non-linearity. These few basic facts of Hamiltonian Mechanics were widely understood by 
the last generation of classical analysts, and they are still important 
today\footnote{Especially in the development of \textit{The Semiclassical Way} \cite{HELLER2018}.}. 

Modern texts follow and expound upon the figure drawing of Sommerfeld and others, making Hamiltonian
mechanics an attractive and delightful subject, especially for visual learners and abstract 
free-thinkers. The pendulum phase portrait, here Fig. \ref{fig:PendPortrait}, is an iconic standard. 
Both popular and specialist accounts\footnote{For example: \cite{STROGATZ2019} Chapter 11; \cite{STROGATZ2018} 
Chapter 6; \cite{HARTER2019} Unit 2, Chapter 7, \cite{LL1982} Chapter III, etc.} usually include such a figure when developing the 
visual language. The more demanding references typically ask the student not only to recreate the figure 
drawing, but also to measure its dimensions in terms of a period integral. Strangely enough, the typical 
solution does not usually involve much geometry, instead shows that a pendulum's dimensionless falling/rising 
velocity $\dot{w}$ along the vertical coordinate $w$ satisfies an algebraic constraint, 
$\dot{w}^2=\frac{1}{4}w(1-w)(1-\alpha w)$. Change of variables by $w=\sin(\phi)^2$ then yields the 
textbook form, 
\begin{eqnarray}
T(\alpha) = 2 \int_{0}^{1} \frac{ dw}{\sqrt{w(1-w)(1-\alpha w)}} =
\oint \frac{ d\phi}{\sqrt{1-\alpha \sin(\phi)^2}} . \nonumber
\end{eqnarray}
Avoiding mimicry, we will use geometric methods to find, and to prove valid, a canonical 
Hamiltonian formulation where $q/p=\tan(\phi)$ and $\dot{\phi}=\sqrt{1-\alpha\sin(\phi)^2 }$. 
Discovery (or rediscovery) of the algebraic Hamiltonian function $2H_{\varphi}(p,q)=(p^2+q^2)(1-\frac{1}{4}q^2)$ 
leads not only to quick derivation of the period integral, but also to an exact solution of the time 
parameterization problem via the Harold Edwards theory of elliptic curves and
functions.  

The article "A normal form for elliptic curves" testifies to Abel's inspirational genius, and stands 
as a paragon specimen of how mathematicians and scientists can use writing to promote alternative 
perspectives, fair attribution, and historical continuity \cite{EDWARDS2007}. History leads Edwards 
and his readers to a thoughtful and self-consistent revision of elliptic function theory. Simplification 
by symmetry is an important theme in this work of pure mathematics, but the results are not solely the 
product or possession of a leisure class. Interdisciplinary applications are part of the history and 
its followings. In cryptography, the simplified addition rule helps to optimize implementations of 
the widely-used Diffie-Hellman key-exchange protocol \cite{DBTL2007}. Thus computer scientists have been among the 
first to accept and utilize the alternative paradigm. Edwards's normal form also presents physicists 
with an opportunity to break free from the confines of disciplinary boundaries and standard 
formulae\footnote{As hinted during 
\href{https://media.ccc.de/v/31c3_-_6369_-_en_-_saal_1_-_201412272145_-_ecchacks_-_djb_-_tanja_lange}{"ECCHacks" @ 31C3}, 
Chaos Communication Congress 2014 \cite{DBTL2014}.}. 
Perhaps the future will have a better role for "Hamilton-Abel theory" than that of a clever joke or 
an idle dream. As a first step to actualizing H.A. theory, we will develop a synthesis between real-valued 
Hamiltonian mechanics and the sort of complex-valued calculus present in Harold Edwards's original 
masterpiece.\pagebreak

\begin{figure*}[t] 
\begin{center}
\begin{overpic}[width=0.75\textwidth]{./Figures/ExperimentalGeometry.eps}
 \put (8,33) {\Large$s$}
 \put (94,2) {\Large$x$}
 \put (24,33) {\Large$g$}
 \put (33,7) {\Large$m$}
 \put (43,17) {\Large$\theta$}
 \put (62,27) {\Large$\theta_0$}
 \put (87,18) {\Large$a_0$}
 \put (46,43) {\Large$z$}
\end{overpic}
\caption{Simple Pendulum Geometry.}
\label{fig:PendulumGeo}
\end{center}
\end{figure*}


\section{Preliminary Analysis}
\begin{wrapfigure}{Rt}{6cm}
\begin{center}
\captionof{table}{More Parameters.} 
\label{tab:PQList}
\begin{tabular}{ c | c  }
\hline \hline
\;Symbol\; & \;\;\;\; Dimension \;\;\;\; \\
\hline
$l$ & $[\;L\;]$  \\
$a_0$ & $[\;L\;]$  \\
$g$ & $[\;L\;]\;[\;T\;]^{-2}$  \\
$m$ & $[\;M\;]$ 
\end{tabular}
(\;\underline{L}ength, \underline{M}ass, \underline{T}ime\;)
\end{center}
\end{wrapfigure}


The simple pendulum consists of a massive bob attached by a solid rod, assumed 
massless, to an axle as in Fig.~\ref{fig:PendulumGeo}. Gravity acts on the bob with 
vertical force $mg$, and the attachment applies a response force. The rod feels 
extensive and compressive stresses, but is assumed to respond with zero strain.
As time elapses the bob swings and undergoes periodic motion 
along a circular trajectory of radius $l$. In \textit{librational motion}
the signs of angular coordinate $\theta$ and angular velocity $\dot{\theta}$ alternate 
while the pendulum reaches extremum deflection at regular intervals throughout the 
experiment. The time of one complete oscillation is called the \textit{period} and 
denoted by symbol $T$. In the absence of frictional damping a time series starting
from time $t=0$ at initial angle $\theta_0$ would have that $\theta(t)=\theta_0$ whenever 
$t/T\in \mathbb{N}$. More realistically, one period $T_n$ separates successive 
maxima $\theta_n$ and $\theta_{n+1}$, with a non-zero frictional loss $\delta\theta_n = 
\theta_{n+1}-\theta_n>0$. Only in the limit where $\delta\theta_n$ approaches zero, 
period $T_n=T_{n+1}$ becomes an exactly integrable observable. We shall work out the theory
assuming that $\delta\theta_n = 0$ regardless of $n$, while postponing worries about
fidelity to a followup article on experiment and data analysis.
\pagebreak

The simple design of Fig.~\ref{fig:PendulumGeo} involves a few free 
parameters, collected in Table ~\ref{tab:PQList}. None of the initial parameters 
have dimension $[\;T\;]$ for time, but $[\;T\;]$ does occur as a factor in the 
dimensions of $g$. Without loss of generality, the $[\;L\;][\;M\;][\;T\;]$ dimensional 
system allows us to fix three scale degrees of freedom, i.e. to choose the base units
of length, mass, and time. The obvious choice $l=m=1$ scales dimensions $[\;L\;]$ and 
$[\;M\;]$. Another sensible choice, that $g=1$, forces $l/g=1$, so also sets the 
$[\;T\;]$-scale. By intuition, we can easily guess period $T$ proportional 
to fixed time $\sqrt{l/g}=1$. Such a guess pays no regard to the dimensionless 
amplitude parameter,
\begin{eqnarray}
\alpha = a_0/l = \frac{1}{2}\big( 1 - \cos(\theta_0) \big) =  \sin(\theta_0 / 2)^2. \nonumber
\end{eqnarray}
Parameter $\alpha$ presents a difficulty to dimensional analysis by allowing that $T$ 
is a function and not just a single number. For any integer $n$, the quantity 
$\sqrt{l/g}\;\alpha^n$ has time dimension, as does any quantity $T \in \sqrt{l/g}\;\mathbb{Q}[\![\alpha]\!]$. 
An Ansatz for the period function, 
\begin{eqnarray}
T(\alpha)=\sqrt{\frac{l}{g}}\sum_{n\ge 0}c_n \alpha^n, \nonumber
\end{eqnarray} 
includes infinitely many undetermined coefficients $c_n \in \mathbb{Q}$. These coefficients 
are not constrained by dimensional analysis. The task of determining them calls for a 
stronger approach, one predicated upon a combination of physical principle and
integral calculus.

Let the simple pendulum move in the $xz$ plane along the arc of a unit circle. Choosing 
the center at $(x,z)=(0,1)$ requires that  $x^2+(z-1)^2=1$, and that $x\;\dot{x}=(1-z)\;\dot{z}$.
By these two equations, the bivariate expression for conservation of energy,
$2(2\alpha-z)=\dot{x}^2+\dot{z}^2$, reduces to a univariate form, $\dot{z}^2=2z(2\alpha-z)(2-z)$. 
Re-scaled variable 
$w=z/(2\;\alpha)$ enables neat expression of the rising/falling velocity, but even greater 
simplicity follows from the choice of $w=\sin(\phi)^2$. The period integral then takes 
the form given in the introduction. Quarter-period integral
\begin{eqnarray}
K(\alpha)=T(\alpha)/4=\int_{0}^{\frac{\pi}{2}} \frac{d\phi}{\sqrt{1-\alpha\;\sin(\phi)^2}}, 
\nonumber
\end{eqnarray}
is such a famous standard that it has a long-winded name, \textit{the complete elliptic 
integral of the first kind} \cite{LL1982}. By formally verifying that $T(\alpha)=4K(\alpha)$, 
part $(a)$ of the typical textbook exercise is already completed. More challenging part 
$(b)$ asks for a closed form for the coefficients $c_n$. Term-by-term integration of 
the series expansion\footnote{Using that $(1-4\Phi)^{-1/2} = \sum \binom{2n}{n}\Phi^n $ 
and that $\oint (2\sin(\phi))^{2n} = \binom{2n}{n}$. Cf. OEIS \cite{SLOANE2019},  
\href{https://oeis.org/A000984}{A000984}.} 
yields a closed form for each $c_n$,
\begin{eqnarray}
T(\alpha)= \oint \frac{d\phi}{\sqrt{1-\alpha \sin(\phi)^2}}
=\sum_{0}^{2\pi} \frac{1}{4^n}\binom{2n}{n} \alpha^n \oint\sin(\phi)^{2n}
= 2\pi \sum_{0}^{2\pi} \frac{1}{16^n}\binom{2n}{n}^2 \alpha^n. \nonumber 
\end{eqnarray}
If both parts $(a)$ and $(b)$ are too easy, part $(c)$ asks for proof that 
$c_n=\frac{2\pi}{16^n}\binom{2n}{n}^2$ satisfies ${(n+1)^2 c_{n+1}=(n+1/2)^2c_n}$, and 
consequently that $K(\alpha)$ satisfies a particular hypergeometric differential 
equation\footnote{We leave this as a worthwhile exercise for the reader. Cf. 
OEIS \cite{SLOANE2019}, \href{https://oeis.org/A002894}{A002894}.}. Unfortunately textbook 
answers to parts $(a)$, $(b)$, and $(c)$, do not contribute at all to intuition for 
the mysterious variable $\phi$. They are not so much answers, but instead a distraction
from deeper inquisition. Why does change of variables from $z$ to $\phi$ 
work so well? Does simplicity indicate hidden meaning? Is $\phi$ actually
the angle of some particular geometric figure? 

Recall that each libration cycle involves an alternating pattern of extrema,
\begin{eqnarray}
\max(\dot{\theta}) \xrightarrow{\;\;\;\;K\;\;\;\;} \max(\theta) 
 \xrightarrow{\;\;\;\;K\;\;\;\;} \min(\dot{\theta})
 \xrightarrow{\;\;\;\;K\;\;\;\;} \min(\theta) 
 \xrightarrow{\;\;\;\;K\;\;\;\;} \max(\dot{\theta}) \nonumber,
\end{eqnarray} 
separated by equal time intervals of length $K(\alpha)$. This notation suggests that the 
abstraction $\phi$ relates somehow to the phase between hanging angle $\theta$ and its angular 
velocity $\dot{\theta}$. A simple hypothesis makes $\phi$ the angle of a polar coordinate
system where $\theta=r\sin(\phi)$ and $\dot{\theta}=r\cos(\phi)$. If correct, the hypothesis should 
enable another derivation of the exact same integrand $d\phi/\dot{\phi}$. Gravitational force 
$mg=1$ downward along the vertical results in a partial force along the tangent of motion, 
which affects an angular acceleration, and by Newton's laws, $\ddot{\theta}=-\sin(\theta)$. 
The phase angular velocity,
\begin{eqnarray}
\dot{\phi} &=& \cos(\phi)^2 \frac{d}{dt} \tan(\phi) 
= \frac{\dot{\theta}^2}{\theta^2+\dot{\theta}^2}\frac{d}{dt}\bigg(\frac{\theta}{\dot{\theta}}\bigg)
=\frac{\dot{\theta}^2-\theta\ddot{\theta} }{\theta^2+\dot{\theta}^2}, \nonumber \\
&=& \cos(\phi)^2+\frac{\sin(\phi)}{r}\sin\big(r \sin(\phi)\big) \nonumber 
 = 1+\sum_{n>0} \frac{(-1)^n}{(2n+1)!} r^{2n} \sin(\phi)^{2n+2}, \nonumber
\end{eqnarray}
looks hopelessly complicated. It is about to get even worse. Conservation of 
energy, ${4 \alpha=\dot{\theta}^2+4 \sin(\theta/2)^2}$, 
determines $r$ as a function of $\sin(\phi)$ and $\alpha$ by reversion of the series,
\begin{eqnarray}
4\alpha = r^2 + 2\sum_{n>0}\frac{(-1)^n }{(2n+2)!} \big(r \sin(\phi)\big)^{2n+2} . \nonumber
\end{eqnarray}
Fortunately, the current argument depends only on the 
linear variation of $T(\alpha)$. The inverse series takes a form $r^2=4\alpha + \mathcal{O}(\alpha^2)$, 
and substitution to $\dot{\phi}$ partially solves $T(\alpha)$, 
\begin{eqnarray}
T(\alpha)= \oint \frac{d\phi}{\dot{\phi}} = \oint d\phi\bigg(1+\frac{2}{3}\alpha\sin(\phi)^4 + \mathcal{O}(\alpha^2) \bigg).\nonumber
\end{eqnarray}
Integral identity $\oint d\phi \sin(\phi)^4=3/8$ ensures the correct value $c_1=1/4$. More 
importantly, the expansion shows by contradiction that $\tan(\phi) \neq \theta/\dot{\theta}$ 
because the linear term of $T(\alpha)$ is an integral of $\sin(\phi)^4$ rather than $\sin(\phi)^2$.
The false hypothesis is not entirely a loss. Similarity between alternative calculations of 
$T(\alpha)$ supports the idea that $\phi$ could be a phase angle. To find and prove the correct 
geometry, we will next give a short development of Hamiltonian mechanics, including a few basic 
facts from transformation theory.

%\begin{wrapfigure}{R}{6.5cm}
%\begin{center}
%\begin{overpic}[width=0.4\textwidth]{./Figures/PendulumHpq.eps}
% \put (60,-1) {$\phi$}
% \put (58,80) {$\alpha$}
% \put (80,2) {$p$}
% \put (15,7) {$q$}
%\end{overpic}
%\caption{An energy surface. }
%  \label{fig:PendulumHpq}
%  \phantom{space}
%\end{center}
%\end{wrapfigure}



\section{Phase Plane Geometry}

The \textit{phase plane} is a two-dimensional, Euclidean vector space spanned by 
Cartesian $(p,q)$ variables. These variables measure the state of a test mass as it 
undergoes classical motion along one dimension. A choice of coordinates involves infinitely
many hidden degrees of freedom, so it is usually not true that $p$ stands for momentum and 
$q$ for position. Hamiltonian mechanics reserves the letters $p$ and $q$ for those 
\textit{canonical coordinates}, which are defined to satisfy Hamilton's equations of motion,
\begin{eqnarray}
\frac{d}{dt}\Big(p,q\Big)= \Big(\dot{p},\dot{q}\Big) 
= \bigg(-\frac{\partial H }{\partial q}, \frac{\partial H}{\partial p} \bigg). \nonumber
\end{eqnarray}
Additional variables $H$ and $t$ stand for the Hamiltonian energy function and the special
time parameter, respectively. Again, we will assume $2H(p,q)=\alpha$ 
with $\dot{\alpha} = 0$, thus the Hamiltonian $H$ uniquely determines the conserved total 
energy at any state-point $(q,p)$. A three-dimensional visualization of function $H(q,p)$ 
superposes a surface,
\begin{eqnarray}
\mathcal{H} = \{ \; (p,q,\alpha) \in \mathbb{R}^3 \; : \; \alpha = 2\;H(p,q) \; \}, \nonumber
\end{eqnarray}
above the phase plane. Level sets of surface $\mathcal{H}$ project to phase 
curves,
\begin{eqnarray}
\mathcal{C}(\alpha) = \{ \; (p,q) \in \mathbb{R}^2 \; : \; \alpha = 2\;H(p,q), \;\; (\dot{p},\dot{q})\neq(0,0) \; \}. \nonumber
\end{eqnarray}
The tangent expansion of phase curve $\mathcal{C}(\alpha)$ at energy $\alpha$, 
\begin{eqnarray}
d\alpha = 0 = \frac{\partial H}{\partial p}\;dp+\frac{\partial H}{\partial q}\;dq = \dot{q}\;dp-\dot{p}\;dq, \nonumber
\end{eqnarray}
requires the existence of an \textit{invariant differential}, $dt = dp/\dot{p} = dq/\dot{q}$,
which solves the linear constraint equation. The integral $t=\int dt$ determines the time parameter 
$t$ of $\mathcal{C}(\alpha)$ and, by integral inversion, solutions $q(t)$ and $p(t)$ to the equations of motion. 
At critical points where $(\dot{p},\dot{q})=(0,0)$, a tangent does not exist. Instead, the solution,  
$p(t)=p(0)$ and $q(t)=q(0)$, follows from the constant value theorem. 

\begin{wrapfigure}{R}{7cm}
\begin{center}
\begin{overpic}[width=0.35\textwidth]{./Figures/Paraboloid.eps}
 \put (62,-2) {\Large$p$}
 \put (78,12) {\Large$q$}
 \put (45,97) {\Large$\alpha$}
\end{overpic}
\caption{A Circular Paraboloid $\mathcal{H}_{\circ}$.}
  \label{fig:CircPara}
\end{center}
\end{wrapfigure}
 
In a case-by-case exposition of Hamiltonian mechanics, harmonic oscillation usually shows within 
the first few examples. The Hamiltonian, 
\begin{eqnarray}
2H_0(p,q)=\kappa_p \; p^2 + \kappa_q \; q^2,  \;\;\;\;\;\; \Big(\kappa_p>0,\;\kappa_q>0\Big)\nonumber
\end{eqnarray}
describes a paraboloid $\mathcal{H}_0$. The phase curves $\mathcal{C}_0(\alpha)$ 
are concentric ellipses when $\alpha>0$. At the point $(p,q)=(0,0)$, the system reaches a stable minimum 
with $\alpha=0$, positive principal curvatures, $\kappa_p$ and $\kappa_q$, and harmonic 
frequency $\omega = \sqrt{\kappa_p \;\kappa_q}$. Up to an initial condition $\phi_0$, trigonometric 
functions,
\begin{eqnarray}
p(t) &=& \sqrt{\alpha/\kappa_p} \; \cos(\omega \; t + \phi_0)   \nonumber \\
%\;\;\;\;\;\;\;\;\; \text{and} \;\;\;\;\;\;\;\;\;
q(t) &=& \sqrt{\alpha/\kappa_q} \; \sin(\omega \; t + \phi_0),  \nonumber 
\end{eqnarray}
solve Hamilton's equations, $(\dot{p},\dot{q})=(-\kappa_q \;q,\kappa_p \;p)$. A guess-and-check 
strategy works fine when the derivatives of sine and cosine are already known; however, repeating
the solution in polar coordinates helps to develop better insight.

Choosing dimensions $\kappa_p = \kappa_q=\omega$ puts paraboloid $\mathcal{H}_0$ into a most 
symmetrical, circular form, as in Fig. ~\ref{fig:CircPara}. In polar coordinates, 
${\big(p,q\big)=\big(r\;\cos(\phi),r \;\sin(\phi)\big)}$,
the Hamiltonian $2H_{\circ}(r)=\omega r^2$ depends only the radial coordinate and leaves the phase angle 
unconstrained. Conservation of energy sets radius $r=\sqrt{\alpha/\omega}$. Again from 
$\tan(\phi) = q/p$ it follows that $\dot{\phi}=(p \dot{q}-\dot{p}q)/r^2=\omega(p^2+q^2)/r^2=\omega$ 
and that $\phi = \omega t + \phi_0$. This calculation verifies the time parameterization, but more 
importantly allows proof, $\partial_r H_{\circ}(r) \neq \omega$, that $r$ and $\phi$ do not make a 
pair of canonical coordinates. Instead, the canonical choice is $\lambda = r^2/2$, and then 
Hamilton's equations work perfectly well, 
\begin{eqnarray}
\dot{\phi}=\partial_{\lambda} H_{\circ}(\lambda) = \omega 
\;\;\;\;\;\;\; \text{and} \;\;\;\;\;\;\; 
\dot{\lambda}=-\partial_{\phi} H_{\circ}(\lambda) = 0 . \nonumber
\end{eqnarray}
This observation unlocks the secret of Sommerfeld's second drawing. It is a particularly
plain picture of the harmonic oscillator phase curves in \textit{action-angle}
coordinates $(p,q)=(\lambda,\phi)$. Action-angle coordinates are an extremely useful
tool, and worthy of a proper introduction.

When stated seriously, the changing of a phase ellipse into circular form gives a seminal 
example of canonical transformation theory. Squeeze transformation of the phase plane,
$(p_i,q_i) \rightarrow (p_f,q_f)=(p_i\;k,q_i/k)$, recovers $H_{\circ}(p_f,q_f)$ from the 
earlier $H_{0}(p_i,q_i)$, when  ${k=(\kappa_p/\kappa_q)^{1/4}}$. 
Hamilton's equations remain invariant, 
\begin{eqnarray}
\Big(\dot{p}_i,\dot{q}_i\Big) 
= \bigg(-\frac{\partial H }{\partial q_i}, \frac{\partial H}{\partial p_i} \bigg) \nonumber
\longrightarrow
\Big(\dot{p}_f/k,\dot{q}_f\;k\Big) 
= \bigg(-\frac{\partial H }{\partial q_f}/k, \frac{\partial H}{\partial p_f}\;k \bigg),
\end{eqnarray}
as the extra factors $k$ cancel. Similarly, factors of $k$ cancel after transforming
the area two-form, $dp \wedge dq$. This is no coincidence. When coordinate $q$ holds 
constant, the tangent geometry requires that $dp/d\alpha = (2\dot{q})^{-1}$, and similar for 
$dq/d\alpha$ with $p$ constant. After applying Stokes's theorem, the time integrand 
obtains a profound form, $dt=2\partial_{\alpha} \int_H dp\wedge dq$. We can now ask a fundamental
question: which coordinate transformations leave $dt$ invariant, or equivalently, which 
transformations leave Hamilton's equations invariant?

Under a general transformation $(p_i,q_i) \rightarrow \big(p_f,q_f\big)$ the area form,
\begin{eqnarray}
dp_f \wedge dq_f = \bigg(\frac{\partial p_f}{\partial  p_i}dp_i+\frac{\partial p_f}{\partial q_i}dq_i\bigg)
\wedge\bigg(\frac{\partial q_f}{\partial p_i}dp_i+\frac{\partial q_f}{\partial q_i}dq_i\bigg)
=\det(J) \; dp_i \wedge dq_i,  \nonumber
\end{eqnarray}
scales according to the Jacobian matrix and its determinant,
\begin{eqnarray}
J= \begin{bmatrix}
\frac{\partial p_f}{\partial p_i} & \frac{\partial p_f}{\partial q_i} \\
\frac{\partial q_f}{\partial p_i} & \frac{\partial q_f}{\partial q_i}
\end{bmatrix} \;\;\;\;\;\;\; \text{and} \;\;\;\;\;\;\; 
\det(J) = \frac{\partial p_f}{\partial p_i}\frac{\partial q_f}{\partial q_i}
 - \frac{\partial p_f}{\partial q_i}\frac{\partial q_f}{\partial p_i}. 
\nonumber 
\end{eqnarray}
After characterizing the inverse transform $(p_f,q_f) \rightarrow \big(p_i,q_i\big)$ 
with an analogous matrix $\widetilde{J}$, action on the tangent vectors,
\begin{eqnarray}
(dp_f,dq_f)= J \cdot (dp_i,dq_i) 
\;\;\;\;\;\;\;\; \text{and} \;\;\;\;\;\;\;\;
(dp_i,dq_i)= \widetilde{J} \cdot (dp_f,dq_f). \nonumber
\end{eqnarray}
ensures a product to identity, $\widetilde{J}J=\mathbb{I}$. Hamilton's equations 
transform accordingly,
\begin{eqnarray}
\big(\dot{p}_f,\dot{q}_f \big) = \bigg(-\frac{\partial H}{\partial q_f},\frac{\partial H}{\partial p_f}\bigg)   
\longrightarrow J \cdot \big(\dot{p}_i,\dot{q}_i \big) = 
P\cdot \widetilde{J} \cdot P \cdot \bigg(-\frac{\partial H}{\partial q_i},\frac{\partial H}{\partial p_i}\bigg),
\nonumber 
\end{eqnarray}
with permutation matrices $P$ such that $P \cdot J^{-1} \cdot P = J/\det(J)$. When 
$\det(J)=1$, the area form and Hamilton's equations remain invariant
\footnote{This proof is adapted from \cite{RALSTON1989}.}. The change of coordinates is 
then said to be a \textit{canonical transformation}. For example, the pair of Jacobians,
\begin{eqnarray}
J =  \begin{bmatrix} p_i & q_i \\
-q_i/(p_i^2+q_i^2) & p_i/(p_i^2+q_i^2)
\end{bmatrix}  \;\;\;\;\;\;\;\; \text{and}  \;\;\;\;\;\;\;\; 
\widetilde{J} = \begin{bmatrix} \cos(q_f)/\sqrt{2p_f} & -\sqrt{2p_f}\sin(q_f) \\
\sin(q_f)/\sqrt{2p_f} & \sqrt{2p_f}\cos(q_f) \end{bmatrix}\nonumber,
\end{eqnarray}
satsify $\det(J)=\det(\widetilde{J})=1$, thus action-angle coordinates 
$(p_f,q_f\big)=(\lambda,\phi)$ are proven canonical\footnote{This fact 
also follows quite obviously from the simple calculation, 
$d\lambda \wedge d\phi = r dr\wedge d\phi$, because the right hand side 
reproduces the well-known area form of a polar coordinate system.} relative 
to position-momentum coordinates $(p_i,q_i)=(p,q)$.


\begin{wrapfigure}{R}{6.5cm}
\begin{center}
\begin{overpic}[width=0.35\textwidth]{./Figures/Helicoid.eps}
 \put (62,-2) {\Large$p$}
 \put (78,12) {\Large$q$}
 \put (45,95) {\Large$t$}
 \put (77,69) {\Large\rotatebox[origin=c]{90}{$\longrightarrow$}}
 \put (81,66) {\Large$T$}
 \put (77,48) {\Large\rotatebox[origin=c]{90}{$\longrightarrow$}}
 \put (81,45) {\Large$T$}
 \put (77,27) {\Large\rotatebox[origin=c]{90}{$\longrightarrow$}}
 \put (81,24) {\Large$T$}
 
\end{overpic}
\caption{A Helicoid Flow $\mathcal{F}_{\circ}$.}
  \label{fig:Helicoid}
  \phantom{space}
  \phantom{space}
  
\end{center}
\end{wrapfigure}


Jacobian matrices simplify validation of coordinate transformations, but they are no 
substitute for Hamilton's equations. The integral form,
\begin{eqnarray}
t = \int_{0}^{t} dt' = 2\partial_{\alpha}\int\!\!\!\!\int_{H} dp\wedge dq
=2 \partial_{\alpha} S(\alpha,p,q), \nonumber
\end{eqnarray}
tells more, that time $t$ depends on the area $S(\alpha,p,q)$
interior to curve $\mathcal{C}(\alpha)$, between a valid initial and final 
condition, $(p_0,q_0)$ and $(p,q)$ respectively. The Hamiltonian flow, 
\begin{eqnarray}
\mathcal{F} = \big\{\big(p,q,t\big) \in \mathbb{R}^3 : t= 2 \partial_{\alpha} S(\alpha,p,q)  \big\}, \nonumber
\end{eqnarray}
equates time evolution with a solid geometry in phase-space-time 
(Deutsch: \textit{phasenraumzeit}). For harmonic oscillation, we can choose initial conditions such that 
$2 S_{\circ}(\alpha,\lambda,\phi) = (\phi/\omega)\;\alpha$, and then the flow goes along a helicoid spiral over 
circular phase curves. In Fig. \ref{fig:Helicoid}, the helicoid flow repeats after a vertical time-translation 
of ${T_{\circ}=2\pi/\omega}$, because any closed curve $\mathcal{C}_{\circ}(\alpha)$ bounds an entire area, 
${S_{\circ}(\alpha)=\pi r^2=(\pi/\omega)\alpha}$. Amplitude independence of period $T_{\circ}$ characterizes 
harmonic oscillation. More generally, \textit{anharmonic oscillation} along curve $\mathcal{C}(\alpha)$ 
returns to initial condition $(p_0,q_0)$ after a period $T(\alpha)$, which characteristically 
\textit{must vary} with energy $\alpha$. Usually amplitude dependence cannot be avoided, 
and then the harmonic equations of motion are, at best, valid and useful only as an approximation 
in the infinitesimal limit. 


The most interesting phase plane geometries involve a spatial admixture of qualitatively different 
critical points. In local coordinates $(p,q)$ around a critical point at ${(p,q)=(0,0)}$, assume the 
Hamiltonian becomes approximately quadratic, ${2H(p,q)\approx \kappa_p p^2+\kappa_q q^2}$. Then the sign 
of $\kappa_p\kappa_q$ distinguishes between circular points with $\kappa_p\kappa_q>0$ and hyperbolic 
points with $\kappa_p\kappa_q < 0$. We have already seen that circular points imply local harmonic 
oscillation, but have yet to encounter hyperbolic points or the separatrix curves they pair with. 
Choosing coordinates where $2\widetilde{H}_{\circ}(p,q)=p^2-q^2$, hyperbolic trigonometric functions, 
${p(t) = \sqrt{\alpha}\cosh(t)}$ and ${q(t) = \sqrt{\alpha}\sinh(t)}$ solve Hamilton's equations, 
$(\dot{p},\dot{q})=(q,p)$. Proof does not differ significantly from the earlier case 
of Harmonic oscillation. In fact, this similarity between $H_{\circ}$ and $\widetilde{H}_{\circ}$ 
is the first sign of an even deeper transformation theory\footnote{The general case, 
$H\rightarrow \widetilde{H}$, is discussed thoroughly in Section IV. Incorporating Complex Time.}.

Around a hyperbolic point, the Hamiltonian factors, $2\widetilde{H}_{\circ}(p,q)=(p+q)(p-q)$. Lines determined
by $p+q=0$ and $p-q=0$ intersect at the hyperbolic point. A separatrix curve, with $\alpha=0$, 
extends from those two lines outwards to the wider area of the phase plane. When the Hamiltonian 
contains terms higher than quadratic, a line leaving the hyperbolic point can possibly change 
direction. Then the separatrix segment either goes off to infinity, or it may return along 
another direction to the initial hyperbolic point, or it may even approach a second, distinct 
hyperbolic point. Taking a union over disjoint segments, the \textit{separatrix} curve is so 
named because, globally, it separates the phase plane into qualitatively different, non-intersecting 
subsets. For example, the red separatrix curve of Fig. \ref{fig:PendPortrait} separates the phase 
portrait into regions where the pendulum rotates either clockwise or counterclockwise, and another 
central region where libration occurs. 

A general \textit{oscillation disk} is here defined as a topological disk within the phase plane, 
bounded by a separatrix at energy $\alpha_1$, and containing exactly one critical point, 
a circular point at energy $\alpha_0$. This domain also contains anharmonic oscillation within the energy range 
$\alpha \in (\alpha_0 , \alpha_1)$. Level curves around the circular point gradually deform away from a circular 
shape, but must retain loop topology. When a curve accumulates non-constant variation of curvature, 
the phase angular velocity $\dot{\phi}$ measures change of shape, as in Fig.~\ref{fig:HeatMap}. 
Most obviously, the heatmaps show slowing around the hyperbolic points. On a general oscillation disk, 
the limit $\alpha \rightarrow \alpha_1$ causes at least one small interval of the curve $\mathcal{C}({\alpha})$ 
to pinch into a corner near a hyperbolic point. Such an extreme deformation forces $\dot{\phi}$ to approach zero 
locally, and the integral period function diverges. 

\begin{figure*}[t] 
\begin{center}
\begin{overpic}[width=0.9\textwidth]{./Figures/HeatMap.eps}
 \put (33,0) {$\dot{\phi}=$}
 \put (38,0) {$\{0$}
 \put (59.5,0) {$1\}$}
 \put (0,10) {$2H_1=p^2+q^2-p^2 q^2$}
 \put (2,5) {$\dot{\phi}=1-\frac{2\;p^2 q^2}{p^2+q^2}$}
 \put (33.5,10) {$2H_{\varphi}=(p^2+q^2)(1-\frac{1}{4} q^2)$}
 \put (35.5,5) {$\dot{\phi}=1-\frac{1}{2}q^2$}
 \put (74,10) {$2H_{\vartheta}=p^2+\sin(q)^2$}
 \put (76,5) {$\dot{\phi}=\frac{p^2+q\cos(q)\sin(q)}{p^2+q^2}$}
\end{overpic}
\caption{A Few Oscillation Disk Heatmaps, Colored by $\dot{\phi}$.}
\label{fig:HeatMap}
\end{center}
\end{figure*}
\pagebreak 

The pendulum phase portrait of Fig. \ref{fig:PendPortrait} includes an oscillation disk at low energy,
which appears again in the right of Fig. ~\ref{fig:HeatMap}. Choosing canonical coordinates 
$(p,q)=(\dot{\theta},\theta)/2$, the pendulum's Hamiltonian energy function is written,  
\begin{eqnarray}
\alpha=2H_{\vartheta}(p,q)=p^2+\sin(q)^2, \nonumber
\end{eqnarray}
over energy domain $0<\alpha<1$. Within one period, $\pi \le 2q \le \pi$, a circular point occurs at 
$(q,p)=(0,0)$, and hyperbolic points occur at $(p,q)=(0,\pm \pi/2)$. The period function must increase 
from its harmonic limit at $\alpha=0$ to an infinite divergence at $\alpha=1$. Limiting analysis 
recapitulates what we should already know from preliminary analysis and laboratory experiments. At 
small amplitudes, the pendulum oscillates harmonically around the stable minimum. Larger amplitude 
oscillations eventually reach an inverted configuration where the force of gravity has only a small component 
along the direction of motion. Thus slowing occurs on approach to the unstable equilibrium point.  

The mere existence of singular divergences suggests that ordinary differential equations 
could be a useful tool for rigorously defining the pendulum's period function. At first 
we will prefer an easier, more direct analysis. Up to a second power of the action variable 
$\lambda$, the pendulum's Hamiltonian may be approximated as 
$\alpha/2=H(\lambda,\phi)=\lambda - \frac{2}{3}\lambda^2\sin(\phi)^4 $. 
Quadratic root solving yields twice the action $2\lambda=\frac{3}{2}\frac{1}{\sin(\phi)^4}
\Big(1-\sqrt{1-\frac{4}{3}\alpha \sin(\phi)^4}\Big)$, 
and by the chain rule, 
\begin{eqnarray}
T(\alpha) = \oint \frac{d\phi}{\dot{\phi}} =2 \oint d\phi\; \partial_{\alpha}\lambda 
= \oint \frac{d\phi}{\sqrt{1-\frac{4}{3}\alpha \sin(\phi)^4}} = 2\pi\bigg(1 + 
\frac{1}{4}\alpha + \frac{35}{192}\alpha^2 + \ldots \bigg) \nonumber .
\end{eqnarray}
This is a nice and easy trick! It gets the linear term correct, without any need for series reversion. 
However, the sine function is bounded by $\pm 1$, so the denominator goes to zero when $\alpha=3/4$,
and the critical points fall short at $|q|=\sqrt{3/2}<\pi/2$. After $c_0$ and $c_1$, all 
coefficients $c_n$ with $n>1$ are apparently overestimates\footnote{It should be possible to
prove this assertion by induction on $n$.}. Instead, let us introduce an arbitrary 
trigonometric polynomial $\Phi$ and replace $(4/3)\sin(\phi)^4 \rightarrow \Phi $.
When $\Phi=\sin(\phi)^2$, the root solving procedure reproduces the correct period 
function. 

As $\Phi$ multiplies $\lambda^2$ in the Hamiltonian, any valid perturbing term needs $\Phi$ to 
be a homogeneous quartic polynomial in the variables $P=\cos(\phi)$, $Q=\sin(\phi)$. The quadratic 
choice $\Phi=Q^2$ obviously does not satisfy the validity constraint, but a workaround is available 
via the Pythagorean theorem, $P^2+Q^2=1$. Multiplication of $\Phi$ by the algebraic unit $P^2+Q^2$ 
raises the polynomial degree by $+2$. This allows a quartic choice $\Phi=(Q^2+P^2)Q^2$, 
which may reduce to $\Phi=Q^2$, but also identifies with a valid perturbing term. The algebraic Hamiltonian 
$2H_{\varphi}(p,q)=(q^2+p^2)(1-\frac{1}{4}q^2)$ transforms canonically to its action-angle form, 
$H_{\varphi}(\lambda,\phi)=\lambda - \frac{1}{2}\lambda^2\sin(\phi)^2 $, again quadratic in the variable $\lambda$.
This geometry includes an oscillation disk in the energy range $0<\alpha<1$, which appears in 
the center of Fig. ~\ref{fig:HeatMap}. By design, the quarter period of $H_{\varphi}$ is $K(\alpha)$.

Isoperiodicity suggests that the two Hamiltonian models, $H_{\vartheta}$ and $H_{\varphi}$, will equate, 
$H_{\vartheta}(p_i,q_i)=H_{\varphi}(p_f,q_f)$, by a canonical transformation. A plausible guess that 
$p_i \;\dot{q}_i = p_f \;\dot{q}_f$ gives a second constraint equation in two sets of two unknowns. 
The non-linear system can be solved for either, 
\begin{eqnarray}
(p_i,q_i) &=& \bigg(\frac{1}{2}\;p_f\;\sqrt{4-q_f^2}, \;
\arcsin\Big(\frac{1}{2}q_f \sqrt{4-q_f^2}\Big)\bigg)
 \text{\;\;\;\;\;\;\;\;\;\;\;\;\;\;\;\;or}  \nonumber \\ 
(p_f,q_f) &=& \pm \bigg(\csc(q_i) \;p_i\; \sqrt{2-2\cos(q_i)}, \; \sqrt{2-2\cos(q_i)} \bigg), \nonumber 
\end{eqnarray}
and the Jacobian matrices are written out as,
\begin{eqnarray}
J = \pm \begin{bmatrix} \sqrt{2-2\cos(q_i)} \csc(q_i) & \frac{p_i \tan(q_i/2)^2}{\sqrt{2-2\cos(q_i)}} \\
0 &  \frac{\sin(q_i)}{\sqrt{2-2\cos(q_i)}}
\end{bmatrix}  \;\;\;\;\; \text{and}  \;\;\;\;\; 
\widetilde{J} = \begin{bmatrix} \frac{\sqrt{4-q_f^2}}{2} & -\frac{p_f q_f}{2\sqrt{4-q_f^2}} \\
0 & \frac{2}{\sqrt{4-q_f^2}} \end{bmatrix}\nonumber,
\end{eqnarray}
while requiring that $|q_f| \le \sqrt{2}$. The domain restriction is permissive enough to map 
between entire oscillation disks. According to the determinants, $\det(J)=\det{\widetilde{J}}=1$,
the angle $\phi$ is indeed a canonical coordinate of the pendulum Hamiltonian. Thus solution of 
the pendulum equations of motion follows from solution of the time parameterization problem on
the phase curves of either $H_{\vartheta}$ or $H_{\varphi}$. We will now pursue one particular 
solution in even more detail.

\section{Incorporating Complex Time}

Harold Edwards's theory of elliptic curves and elliptic functions starts with an 
alternative normal form, $2G=(x^2+y^2)-a^2 x^2 y^2-a^2$, which determines a family
of elliptic curves, $\mathcal{C}(a)=\{(x,y):2G(x,y,a)=0 \;\&\; a^4 \neq 0 \text{ or } 1\}$, 
all non-singular according to the extra condition on modulus $a^4$. With $a$-dependence swept into 
$\tau$, the crucial elliptic function,
\begin{eqnarray}
\psi(t)= \chi(t+\tfrac{1}{2})= \frac{2\sum_{n=1}^{\infty}e^{i \pi(2n-1)^2 \tau/2}\cos\big((2 n-1)\pi t \big) 
}{1+2\sum_{n=1}^{\infty} e^{i \pi(2n)^2 \tau/2}\cos\big(2 n \pi t \big)},\nonumber
\end{eqnarray}
allows a solution, $x(t)=\chi(t)$ and $y(t)=\psi(t)$, of the coupled differential 
equations\footnote{Edwards's normalization of the time domain requires chain rule, 
$\dot{\psi}=(2/T) d\psi/dt  $, with real period $T$.}, ${\dot{x}=y(1-a^2 x^2)}$ and 
$\dot{y}=-x(1-a^2 y^2)$, as derived from the $\mathcal{C}(a)$ tangent geometry. 
Edwards already proves this statement in Part III of the original article \cite{EDWARDS2007}. 
Translation of his solution to the simple pendulum context encounters two difficulties. 
First, a  transformation is needed to go from math coordinates $(x,y)$ to physical coordinates 
$(p,q)$\textemdash but this follows quickly from the previous developments. Second,
and with more effort, calculation of the correct value for the period ratio $\tau$ 
depends upon ascertainment of the double-periodicity notion. To this end, the Wick rotation, 
$t \rightarrow i t$, helps to extend transformation theory,
and to show relation between real and complex periods. 

In a sense, function $G$ is Hamiltonian with canonical coordinates $x$ and $y$. The analogy can be 
improved. Rewriting $(G,a,x,y) \rightarrow (H_1,\alpha,p,q)=(a^2 G,a^4,a\;y,a\;x)$
produces a Hamiltonian form, $\alpha=2H_1(p,q)=p^2+q^2 - p^2 \; q^2$, with an exact parameterization 
${p(t)=\alpha^{1/4} \;\psi(t)}$, ${q(t)=\alpha^{1/4} \;\chi(t)}$. The transformation does not leave 
the energy scale invariant from $G$ to $H$, so cannot be canonnical \textit{per se}. Instead, it is a 
\textit{covariant transformation}, $(\alpha_i,t_i,p_i,q_i) \rightarrow (\alpha_f,t_f,p_f,q_f)$, 
which requires \textit{balanced} scale change\footnote{The more liberal 
condition on $|J|$ follows again from transforming Hamilton's equations. }, 
$|J| = \frac{\partial\alpha_f}{\partial \alpha_i}\frac{\partial t_f}{\partial t_i}$. In action angle 
coordinates, the Hamiltonian becomes $H_1(\lambda,\phi) = \lambda-\frac{1}{2}\lambda^2\sin(2\phi)^2$.
Similarity between $H_{1}$ and $H_{\varphi}$ suggests another covariant transformation, 
$(t_i,\phi_i)\rightarrow (t_f,\phi_f)=2(t_i,\phi_i)$.  Angle doubling transfers the exact 
solution from $H_{1}$ to $H_{\varphi}$, where, 
\begin{eqnarray}
\lambda(t) = \frac{\sqrt{\alpha}}{2}\Big(\chi(\tfrac{t}{2})^2 + \psi(\tfrac{t}{2})^2\Big), \;\;\;\;\; 
\phi(t)   =  \arctan\big(\chi(\tfrac{t}{2})/\psi(\tfrac{t}{2})\big),  \nonumber 
\end{eqnarray}
Time parameterization for pendulum variables then follows by walking back $(\lambda,\phi)$ through 
the sequence of canonical transformations\textemdash first to the Cartesian coordinates of $H_{\varphi}$,
\begin{eqnarray}
p(t)   =  \frac{ \; \psi(\tfrac{t}{2})^2-\chi(\tfrac{t}{2})^2 
}{\Big(1+\chi(\tfrac{t}{2})^2  \psi(\tfrac{t}{2})^2\Big)^{\frac{1}{2}}},  \;\;\;\;\; 
q(t)  = \frac{2 \;  \chi(\tfrac{t}{2})\;\psi(\tfrac{t}{2}) 
}{\Big(1+\chi(\tfrac{t}{2})^2 \psi(\tfrac{t}{2})^2\Big)^{\frac{1}{2}}}, \nonumber 
\end{eqnarray}
then to the Cartesian coordinates of $H_{\vartheta}$,
\begin{eqnarray}
p(t)   = \tfrac{1}{2} \dot{\theta}(t) = \frac{\psi(\tfrac{t}{2})^2 - \chi(\tfrac{t}{2})^2
}{1+\chi(\tfrac{t}{2})^2 \psi(\tfrac{t}{2})^2 }, \;\;\;\;\; 
q(t)   = \tfrac{1}{2} \theta(t) =  \arcsin\bigg(\frac{2\;\chi(\tfrac{t}{2})\; \psi(\tfrac{t}{2})
}{1+\chi(\tfrac{t}{2})^2 \psi(\tfrac{t}{2})^2 }\bigg), \nonumber 
\end{eqnarray}
and finally to $\theta$ and $\dot{\theta}$, after multiplying by a factor of two. The expression 
for $\theta(t)$ reduces again according to a special case of the $X$ 
addition rule, proven in Part II of \cite{EDWARDS2007}. The addition rules act linearly along 
the time dimension, so the choice of $x'=x=\chi(\tfrac{t}{2})$ and 
$y'=y=\psi(\tfrac{t}{2})$ sets $X=\chi(t)$, and also, 
\begin{eqnarray}
\theta(t) = 2 \arcsin\Big(\alpha^{1/4}\;\chi\big(t\big) \Big).  \nonumber
\end{eqnarray}
Yet this formal solution is worthless until we have defined the period ratio $\tau$ as a 
function of the energy parameter $\alpha$. In context, the correct answer is that 
$\tau = i \frac{K(1-\alpha)}{2K(\alpha)}$, but again, why?


\begin{figure*}[t] 
\begin{center}
\begin{overpic}[width=0.85\textwidth]{./Figures/REPsi.eps}
 \put (2,20) {\Large$0$}
 \put (49,20) {\Large$1$}
 \put (96,20) {\Large$2$}
 \put (98,14) {\Large$t$}
 \put (3,37) {\Large$\psi$}
 \put (-7,34) { $\psi(0)$}
 \put (-3,17.5) {\Large$0$}
\end{overpic}
\caption{Real-Valued Slices of the Elliptic Function $\psi(t)$;
$\mathfrak{I}(t)=0$ and 
$\tau = \big\{ i , \tfrac{i}{3},  
\tfrac{i}{9},  \tfrac{i}{27} \big\} $.}
\label{fig:REPsi}
\end{center}
\end{figure*}

The real period,
\begin{eqnarray}
T(\alpha) = 4K(\alpha) = \oint \frac{d\phi}{\sqrt{1-\alpha\sin(2\phi)^2}},\nonumber
\end{eqnarray}
gives a first hint of double-periodicity, because it satisfies a second order differential 
equation. A proof of the assertion depends upon an annihilator $\widehat{\mathcal{A}}$ and its certificate 
function $\Xi$, 
\begin{eqnarray}
\widehat{\mathcal{A}}=1-4(1-2\alpha)\partial_{\alpha}-4\alpha(1-\alpha)\partial^2_{\alpha}  
 \;\;\;\;\;\;\; \text{and} \;\;\;\;\;\;\;
\Xi_n = \frac{\sin(2 \; n \; \phi)}{2\;n\;\big(1-\alpha \sin(n\;\phi)^2\big)^{3/2}}. \nonumber
\end{eqnarray}
The zero sum,
\begin{eqnarray}
\widehat{\mathcal{A}}\circ \bigg(\frac{1}{\sqrt{1-\alpha\;\sin(n\;\phi)^2}}\bigg) 
-\partial_{\phi}\Xi_n =0 \nonumber, 
\end{eqnarray}
is trivial to verify by applying chain rule and reducing trigonometric terms. 
Integration of any exact differential along a complete cycle yields a zero, 
including $\oint \; d\phi \; \partial_{\phi} \Xi_n=0$, thus the claim, 
$\widehat{\mathcal{A}}\circ T(\alpha)=T(\alpha)-\partial_{\alpha}(4\alpha(1-\alpha)\partial_{\alpha}T(\alpha))=0$, 
stands true. Transformation $\alpha \rightarrow 1-\alpha$ leaves $\widehat{\mathcal{A}}$ invariant, 
so the general solution of the second order differential equation can be written as 
$T(\alpha)=c_1 K(\alpha)+ c_2 K(1-\alpha)$. This solution space includes two special 
solutions, the real and complex periods, $T_{\mathfrak{R}}(\alpha)=4\;K(\alpha)$ and 
$T_{\mathfrak{I}}(\alpha)=i 2K(1-\alpha)$, whose ratio reproduces the suggested 
form $\tau=T_{\mathfrak{I}}(\alpha)/T_{\mathfrak{R}}(\alpha)$. 
The hint seems to have paid off, but leaves room for doubt and suspicion. 

We do not yet have a physical reason to complexify the time variable by extending from a one-dimensional
real axis to a two-dimensional complex plane; however, mathematical function theory after Abel calls ahead of 
schedule for just such an abstraction. To account for additional complex degrees of freedom, we will build 
out a Riemann surface\footnote{The nomenclature "Kleinian Surface" may be better. Jeremy Gray claims that 
Felix Klein (1849-1925) was the first to think of a complexified algebraic curve as "a closed surface in 
its own right" \cite{GRAY2010}. }, 
\begin{eqnarray}
\mathcal{R}(\alpha) = \{(p,q) \in \mathbb{C}^2: \alpha=2H(p,q) \} 
\simeq \{(v,w,x,y) \in \mathbb{R}^4: \alpha=2H(x+i v,y+i w) \},\nonumber
\end{eqnarray}
around each phase curve $\mathcal{C}(\alpha)$. If transformation of canonical $p$ and $q$ 
variables introduces determinant $|J| \in \mathbb{C}/\{0,1\}$, then neither the two-form 
$dp \wedge dq$, nor the time scale $dt$, nor Hamilton's equations will preserve. However, 
it is possible to cancel the extra factor $|J|$ by scaling time covariantly, 
$t_i \rightarrow t_f = |J|\;t_i$. Any transformation with unit modulus $|J|$ acts as 
rotation on the complex $t$-plane. In particular, the Wick rotation,  
$t_i\rightarrow t_f = i \; t_i$, goes by $\pi/2$ radians through the time plane, 
thus permutes real and complex axes. The simplest covariant coordinate transformation
having $|J|=i$ takes $p_i\rightarrow p_f=i\;p_i$ and rotates $\mathcal{R}(\alpha)$  by 
$\pi/2$ radians through a four dimensional space $\mathbb{C}^2 \simeq \mathbb{R}^4$.
Covariant Wick rotation is another neat trick, which ultimately allows calculation of 
the complex period on doubly-periodic $\mathcal{R}(\alpha)$ by integration along a 
real-valued level curve $\widetilde{\mathcal{C}}(\alpha)$. Before moving ahead to non-trivial 
topology, it is worthwhile to work through an easy example.

Real-valued theory distinguishes between circular points and hyperbolic points, while 
complex-valued theory does not. Any transformation such as $(p_i,q_i) \rightarrow (p_f,q_f)=(i\;p_i,q_i)$ 
changes constraints, from circular $H_{\circ}$ to hyperbolic $\widetilde{H}_{\circ}$,
and \textit{vice versa}. It is possible to embed the entire \textit{harmonic hyperboloid} 
into three dimensions,
\begin{eqnarray}
\mathcal{R}_{\circ}(\alpha)&=&\{(p,q)\in \mathbb{C}^2: \alpha =2H_{\circ}(p,q) \}  \nonumber \\ 
&&\longrightarrow  \mathcal{S}_{\circ}(\alpha)=\{(x,y,z) \in \mathbb{R}^3: \alpha + z^2 =x^2+y^2 \}, \nonumber
\end{eqnarray}
while losing only a phase degree of freedom via the map $(v,w) \rightarrow z=\pm \sqrt{v^2+w^2}$.
More explicitly, $\mathcal{S}_{\circ}(\alpha)$ is obtained by considering rotational symmetry in the $(p,q)$-plane. 
Coordinates $x=w=0$ and $z=v$ may be chosen such that 
$\widetilde{\mathcal{C}}_{\circ}(\alpha)=\{(y,z): \alpha=-z^2+y^2 \}$ is a hyperbola connected to the base circle 
$\mathcal{C}_{\circ}(\alpha)=\{(x,y): \alpha=x^2+y^2 \}$ at the turning points ${(p, q)=(0,\pm \sqrt{\alpha})}$. 
Any other point $(p,q)=(x_0,y_0)$ along $\mathcal{C}_{\circ}(\alpha)$ is yet again a turning point; however, 
in a rotated coordinate system. Point $(x_0,y_0)$ connects circular $\mathcal{C}_{\circ}(\alpha)$ to a 
\textit{rotated copy} of hyperbolic $\widetilde{\mathcal{C}}_{\circ}(\alpha)$, which
falls in the hyperplane determined by  $(x_0 \;u+y_0 \;v)(x_0 \;y-y_0 \;x)=0$.

\begin{figure*}[t] 
\begin{center}
\begin{overpic}[width=.25\textwidth]{./Figures/HyperboloidCuts.eps}
 \put (17,-4) {\Large$x$}
 \put (68,-1) {\Large$y$}
 \put (68,50) {\Large$v$}
 \put (17,50) {\Large$w$}
\end{overpic}\;\;\;\;\;\;\;
\begin{overpic}[width=.225\textwidth]{./Figures/HyperboloidPhase.eps}
 \put (6,32) {\Large$x$}
 \put (53,32) {\Large$y$}
 \put (32,95) {\Large$z$}
\end{overpic}\;\;\;\;\;\;\;
\begin{overpic}[width=.35\textwidth]{./Figures/RiemannSphere.eps}
 \put (52,72) {\Large$Z$}
 \put (8,20) {\Large$X$}
 \put (93,20) {\Large$Y$}
\end{overpic}
\caption{A Few Depictions of a Genus Zero Riemann Surface.}
\label{fig:GenusZero}
\end{center}
\end{figure*}


Two equivalent parameterizations,  
\begin{eqnarray}
&\mathcal{R}_{\circ}(\alpha) =& \bigg\{
(p,q)=\sqrt{\alpha} \Big(\sin(t),\cos(t)\Big) \in \mathbb{C}^2
: t \in \mathbb{C} \bigg\},  \nonumber  \\
\text{and} \;\;\;\;\;
&\widetilde{\mathcal{R}}_{\circ}(\alpha) =& \bigg\{
(p,q)=\sqrt{\alpha}\Big(\sinh(u),\cosh(u)\Big)  \in \mathbb{C}^2
: u \in \mathbb{C} \bigg\},  
 \nonumber
\end{eqnarray}
of the harmonic hyperboloid follow from the solution of Hamilton's equations, 
around a circular point and a hyperbolic point respectively. Wick rotation $t\rightarrow u= i\;t$ 
equates $\mathcal{R}_{\circ}(\alpha)$ and $\widetilde{\mathcal{R}}_{\circ}(\alpha)$ up to 
a $\pi/2$-radian rotation through the complex $p$-plane. While the usual trigonometric functions 
are periodic, their hyperbolic counterparts certainly are not. Taken together, periodicity along 
$\mathfrak{R}t$ and non-periodicity along $\mathfrak{R}u$ implies genus zero. 
This identification introduces some cognitive dissonance, because the sphere is 
usually given as the standard form of a genus zero surface. In three dimensions, 
the hyperboloid,
\begin{eqnarray}
&\mathcal{S}_{\circ}(\alpha) =& \bigg\{
(x,y,z)=\sqrt{\alpha} \Big(\sin(t)\cosh(u), \cos(t)\cosh(u), \sinh(u)\Big) \in \mathbb{R}^3
: (t,u) \in \mathbb{R}^2 \bigg\},  \nonumber
\end{eqnarray}
transforms into the Riemann sphere by stereographic projection,
\begin{eqnarray}
(x,y,z) \rightarrow \bigg(X,Y,Z\bigg)=\bigg(\frac{2\sqrt{\alpha}\;x}{\alpha+x^2+y^2},
\frac{2\sqrt{\alpha}\;y}{\alpha+x^2+y^2}, \pm \frac{(\alpha-x^2-y^2)}{(\alpha+x^2+y^2)}\bigg),
 \nonumber
\end{eqnarray}
with $+$ chosen for $z \ge 0$ and $-$ chosen otherwise. 

\begin{figure*}[t]
\begin{center}
\begin{overpic}[width=0.3\textwidth]{./Figures/HypHomology.eps}
\end{overpic}\;\;\;\;\;\;\;\;\;\;\;\;
\begin{overpic}[width=0.55\textwidth]{./Figures/UniformHomology.eps}
 \put (3,65) {$\mathfrak{I}(t)$}
 \put (95,29.5) {$\mathfrak{R}(t)$}
 \put (49,35) {$2\pi$}
 \put (95,35) {$4\pi$}
 \put (2,35) {$0$}
 \put (-2,33) {$0$}
% \put (8.5,42) {$\mathcal{N}$}
% \put (49,20) {$\mathcal{C}$}
 \put (-5.6,17.5) {$-2$}
 \put (-2,47) {$2$}
\end{overpic}
\caption{Genus Zero Harmonic Hyperboloid and Singly-Periodic Uniformization.}
  \label{fig:GZeroHom}
\end{center}
\end{figure*}

\begin{figure*}[t] 
\begin{center}
\begin{overpic}[width=0.55\textwidth]{./Figures/ToricSlices.eps}\end{overpic}
\;\;\;\; \begin{overpic}[width=0.3\textwidth]{./Figures/SecKey.eps}
 \put (18,86) {$2H_1=p^2+q^2-p^2 q^2$}
 \put (10,68) {\Large\rotatebox[origin=c]{90}{$\longleftarrow$}}
 \put (15,68) { Wick Rotation}
  \put (18,49.5) {$  2\widetilde{H}_1 =  -p^2+q^2+p^2 q^2 $}
 \put (18,31.5) {$   2\widetilde{H}_1 = -p^2+q^2 - \tfrac{1}{4}(p^2 + q^2)^2 $}
\end{overpic}
\caption{Toric Cross Sections via Wick Rotation.}
\label{fig:ToricSecs}
\end{center}
\end{figure*}

By having two periods rather than just one, a genus one elliptic curve differs qualitatively 
from the genus zero harmonic hyperboloid. In general, the two periods may describe the 
boundaries of a \textit{period parallelogram} in the complex $t$-plane; however, in Edwards's normal 
form, a choice of real $\alpha \in (0,1)$ amounts to choosing a \textit{period rectangle} with entirely 
complex $\tau$ and $\mathfrak{I}(\tau) \in (0,\infty)$. To calculate $\tau$ explicitly, we transform 
by a Wick rotation, $H_1 \rightarrow \widetilde{H}_1$, as in Fig. \ref{fig:ToricSecs}. Both transforms 
introduce a factor $|J|=i$ relative to the initial choice of coordinates\footnote{
Easy Exercise: Prove $|J|=i$ explicitly, then prove the two alternatives canonically equivalent.}.
Constraint $\alpha=2\widetilde{H}_1$, for either $\widetilde{H}_1$, cuts out a real-valued curve 
$\widetilde{\mathcal{C}}(\alpha)$. Hamilton's equations determine the complex period, 
up to a missing factor $i$. One choice leads to an easier integral,
\begin{eqnarray}
T_{\mathfrak{I}}(\alpha) = i \oint dt = i \int_{-\infty}^{\infty} \frac{dp}{\sqrt{(1+p^2)(\alpha+p^2)}},  \nonumber 
\end{eqnarray}
on a non-compact contour, while the other offers compactness with a harmonic limit 
$\omega=2$ as $\alpha \rightarrow 1$. Integral value $T_{\mathfrak{I}}(1) = i \pi$ also 
implies that $\omega = 2\pi i/T_{\mathfrak{I}}(1) =2$.  More importantly, the zero sum, 
\begin{eqnarray}\widehat{\mathcal{A}} \circ \frac{1}{\sqrt{(1+p^2)(\alpha+p^2)}} 
+ \partial_p \; \widetilde{\Xi} = 0, \;\;\;\text{with}\;\;\; 
\widetilde{\Xi} = \frac{p\;(1+p^2)^{1/2}}{(\alpha+p^2)^{3/2}},  \nonumber
\end{eqnarray}
certifies that $\widehat{\mathcal{A}} \circ T_{\mathfrak{I}}(\alpha)=0$, because
$\int_{-\infty}^{\infty} dp \; \partial_{p} \widetilde{\Xi}=0$. The harmonic 
limit determines the constants of integration, and $T_{\mathfrak{I}}(\alpha) = i 2K(1-\alpha)$,
as desired. Depending on the reader's naivet\'{e}, it is either astounding or mundane that 
both real and complex solutions should satisfy the same differential equation. The 
cohomological theory of algebraic varieties gives a standard mathematical explanation
as to \textit{why}\footnote{Along any homology $1$-cycle, integral $\oint dt$ is rational, so
satisfies a "Picard-Fuchs type" ODE \cite{LAIREZ2016}.}, but for now we are more concerned 
with the ever-pressing \textit{what for}.

\begin{figure*}[t] 
\begin{center}
\begin{overpic}[width=0.45\textwidth]{./Figures/ToricRibbons.eps}\end{overpic}
\;\;\;\;\;\;\;\;\;\;\;\;
\begin{overpic}[width=0.4\textwidth]{./Figures/TorusDomain.eps}
 \put (43,95) {$t \in \mathbb{C}$}
 \put (-5,-1) {$0$}
 \put (-7,45) {$2\tau$}
 \put (-7,92) {$4\tau$} 
 \put (-1,-6) {$0$}
 \put (46,-6) {$2$}
 \put (92,-6) {$4$}
\end{overpic}
\caption{Genus One Elliptic Curves and Doubly-Periodic Uniformization.}
\label{fig:ToricRibbons}
\end{center}
\end{figure*}


Unlike their trigonometric counterparts, the shape of $\chi(t)$ or $\psi(t)$ depends 
on the amplitude parameter $\alpha$ via the period ratio $\tau=i\frac{K(1-\alpha)}{2K(\alpha)}$,
now justly derived. Shape variation with $\alpha$ carries over to the Riemann surface,
\begin{eqnarray}
\mathcal{R}_1(\alpha) = \Big\{\big(p,q\big)
=\alpha^{\frac{1}{4}} \big( \psi(t), \; \chi(t)\big) 
\in \mathbb{C}^2 : t \in \mathbb{C} \Big\}, \nonumber
\end{eqnarray}
which limits to a central harmonic hyperboloid when $\alpha \rightarrow 0$ 
and $\mathfrak{I}(t) \rightarrow 2n\; i\; \tau, \; n \in \mathbb{Z}$. Around 
$\mathfrak{I}(t)=0$, surface $\mathcal{R}_1(\alpha)$ looks sufficiently like the 
harmonic hyperboloid to allow a similar embedding\footnote{The map  
$(p,q) \rightarrow (x,y,z)$, from $\mathcal{R}$ to $\mathcal{S}$, 
preserves the euclidean metric: $p^{\star}p+q^{\star}q=x^2+y^2+z^2$.},
\begin{eqnarray}
\mathcal{S}_1(\alpha) = \Big\{\big(x,y,z\big)
=\alpha^{\frac{1}{4}} \big(
\psi_{\mathfrak{R}}(t), \;\chi_{\mathfrak{R}}(t),\;\pm \sqrt{\chi_{\mathfrak{I}}(t)^2+\psi_{\mathfrak{I}}(t)^2} \; \big) 
\in \mathbb{R}^3 : t \in \mathbb{R} \times[0,\pm \tfrac{\tau}{2}] \Big\}. \nonumber
\end{eqnarray}
Subscripts $\mathfrak{R}$ and $\mathfrak{I}$ indicate either the real or the imaginary 
component of complex-valued functions $\psi(t)$ and $\chi(t)$. Hyperbolic sections also 
expand over thin-strip domains of width $\Delta t =1/4$, centered on vertical lines 
$\mathfrak{R}(t)=(2n+1)/4, \; n \in \mathbb{Z}$. Four sections in the range $\mathbb{C}^2$ 
separately approach a harmonic hyperboloid in the limit $\alpha \rightarrow 1$. In the 
right panel of Fig. \ref{fig:ToricRibbons}, the partial domain of all five hyperbolic 
sections covers a full three-quarters of $\mathbb{C}$ while avoiding points 
$(2m+1)\tau + n/2, \; (m,n)\in \mathbb{Z}^2$. At these points, either $\chi(t)$ or $\psi(t)$ 
has a simple pole. Embedding the disjoint pieces of $\mathcal{R}_1(\alpha)$ into three dimensions 
and coloring by the real component of time, we expand from the skeleton of Fig. \ref{fig:ToricSecs} 
to the patchwork surfaces nested in the left of Fig. \ref{fig:ToricRibbons}. 

\begin{figure*}[p!] 
\begin{center}
\vspace{1.5cm}
\begin{overpic}[width=.95\textwidth]{./Figures/ThreeFlows3T.eps}
 \put (21,-2) {\Large$\mathcal{F}_1$} 
 \put (55,-2) {\Large$\mathcal{F}_{\varphi}$} 
 \put (89,-2) {\Large$\mathcal{F}_{\vartheta}$} 

\end{overpic}

\vspace{1.5cm}

\begin{overpic}[width=.95\textwidth]{./Figures/TimeSeries.eps}
 \put (28,22) {$t$} \put (28,5.5) {$t$}
 \put (63.5,22) {$t$} \put (63.5,5.5) {$t$}
 \put (99,22) {$t$} \put (99,5.5) {$t$}

 \put (1,29) {$p$} \put (1,12.5) {$q$}
 \put (36.5,29) {$p$} \put (36.5,14) {$q$}
 \put (72,29) {$p$} \put (72,14.5) {$q$}

 \put (-1.5,27.5) {$1$} \put (-1.5,11) {$1$}
 \put (34,27.5) {$1$} \put (31.5,12) {$\sqrt{2}$}
 \put (69.5,27.5) {$1$} \put (69,13) {$\tfrac{\pi}{2}$}
\end{overpic}
\caption{Hamiltonian Flows and their vertical Projections over one Period.}
\label{fig:ThreeFlows}
\end{center}
\end{figure*}


Depiction of complex tori flexes impressive strength, but actually requires considerably more 
effort than is necessary for applications in classical physics. With only values of $\psi(t)$ above 
the real time axis, we can graph sections of a Hamiltonian flow and make a wide range of experimental 
predictions\textemdash the two tasks are not entirely different. Fig. \ref{fig:ThreeFlows} depicts 
a section of the pendulum's Hamiltonian flow,
\begin{eqnarray}
\mathcal{F}_{\vartheta} = \bigg\{ \Big(\tfrac{1}{2}\theta(t),\tfrac{1}{2}\dot{\theta}(t),2 K(\alpha)t \Big)
 \in \mathbb{R}^3 : t \in \mathbb{R} \bigg\} , \nonumber 
\end{eqnarray}
along with the flows $\mathcal{F}_{\varphi}$ and $\mathcal{F}_{1}$ associated with $H_{\varphi}$ and
$H_{1}$. In terms of textbook expectations, these graphs leave little to be desired, save a few 
numerical double-checks. The following zero sums, 
\begin{eqnarray}
0 = 2 H(q(t),q(t))-\alpha , \;\;\;\;\;\;\;\; 
0 = \dot{p}(t) + \partial_{q} H|_t, \;\;\;\;\;\;\;\; 
0  = \dot{q}(t) - \partial_{p} H|_t, \nonumber
\end{eqnarray}
can be evaluated for real sample points $t \in [0,2]$ and $\alpha \in(0,0.99)$. A truncation of 
$\psi$ after five 
summation terms is sufficient to converge to machine precision, $0 \approx 10^{-15}$. This 
agreement should dispel any remaining doubt about veracity of the solution, but if necessary, 
testing can be extended to higher values $\alpha \in (.99,1)$ and to arbitrary precision 
simply by including more summation terms.

Even then, those more inclined to standard methodology can be expected to issue an obstinate 
argument that "Hamilton's equations have a unique solution. For the simple pendulum, that 
solution is written in terms of the Jacobian Elliptic functions". Certainly this is also true
\cite{WW1902,WHITTAKER1904}, and by comparison of \textit{equally valid} solutions, we find that,
\begin{eqnarray}
\text{sn}\big(2K(\alpha)t\big|\alpha\big) = \alpha^{-\frac{1}{4}}\chi(t)  , \;\;\;\;\;\;\;\;\;\; 
\text{cn}\big(2K(\alpha)t\big|\alpha\big) = \frac{\psi(\tfrac{t}{2})^2 - \chi(\tfrac{t}{2})^2
}{\chi(\tfrac{t}{2})^2 + \psi(\tfrac{t}{2})^2 },\nonumber
\end{eqnarray}
only where $\alpha \in (0,1)$ and $\tau = i\frac{K(1-\alpha)}{2K(\alpha)} \in (0,\infty)$.
These identities imply that any equations solvable in terms of the Jacobian elliptic functions are also solvable 
in terms of $\psi(t)$ with entirely imaginary $\tau$. Allowing that 
$\tau \in \mathbb{C}$, the function $\psi(t)$ becomes fully as powerful as Weierstrass's 
$\wp(t)$, i.e. it is the most general elliptic function possible. More work remains to 
be done. Wherever a standard solutions exists and is known, another alternative solution 
in terms of $\psi(t)$ is waiting to be found. What we have seen so far encourages the 
hope that new derivations will lead to even more new insight!

\pagebreak 
\section{Conclusion}
Hamiltonian mechanics teaches us to appreciate the truth and the beauty of classical mechanics 
through geometric abstractions such as phase portraits, energy landscapes, and time-evolving flows. 
Even more impressive, insightful geometry arises when incorporating complex time. The value inherent 
to one particular solution or sculptural model only increases through real and complex transformation 
theory, which enables proliferation of equivalent forms. In mathematics and physics, choice of 
coordinates always matters, as we have seen yet again in this problem about the simple pendulum. 
Coordinate transformation enables rapid analysis of period integrals in terms of an ordinary 
differential equation paired with coordinate-dependent certificate functions. A wide variety of 
oscillation disks will succumb to similar integral-differential analysis, so it is well-worth 
learning at the start. 

Anharmonic period integrals have long been a topic of interest, not only for theorists, but also for 
experimental scientists. The days of Kepler and Galileo are receding ever farther into the past, and 
with them the frustrations of slow progress and low-precision measurement. Recent digital 
technologies\textemdash including slow-motion video and image processing\textemdash 
potentially enable the commonplace citizen scientist to begin measuring oscillation disks and 
period functions. Non-linearity presents a serious challenge to most student analysts, as they
are over-accustomed to the status quo of linear experiments and linear regressions. In a planned 
followup article, we will actively address this weakness. The next article will describe an 
easily-reproducible, high-precision pendulum experiment, and will use the techniques of phase 
plane geometry to extract a shape parameter, which describes the similarity of the actual period 
function to its predicted form $4K(\alpha)$.
 
The harmonic hyperboloid and the anharmonic tori are only the first few examples on a magnificent 
bridge between physics and mathematics. The dream of "Hamilton-Abel" theory is to see more pedestrian 
and commercial traffic crossing this bridge. The simple pendulum affords one opportunity to exchange 
valuable ideas, but certainly, more opportunity abounds. An exciting frontier grows out of quantum 
Hamiltonian mechanics, and the sort of semiclassical calculations that owe back to the Wiemar time 
of Ehrenfest and Sommerfeld. When we finally get to calculating complex-time tunneling integrals 
along Riemannian tori, Abel's original insight will seem all the more profound and prescient. \pagebreak


\interlinepenalty=10000
\bibliographystyle{unsrtnat}
\bibliography{biblio}

\end{document}
